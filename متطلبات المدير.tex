


ملاحظات مستجدة

ـ أي مكان فيه اليوم يتم اضافة أيام الأسبوع كاملة من السبت وحتى الجمعة قائمة منسدلة ( السبت الأحد الإثنين الثلاثاء الأربعاء الخميس ـ الجمعة )
🎄🎄🎄🎄🎄🎄🎄🎄🎄🎄🎄🎄🎄🎄 فاصل بين الفقرات الرئيسية 🎄🎄🎄🎄🎄🎄🎄🎄🎄🎄🎄🎄🎄

➖➖➖➖➖➖➖➖➖➖➖ فاصل بين الفقرات الفرعية ➖➖➖➖➖➖➖➖➖➖➖➖➖
🔴🔴🔴🔴🔴🔴🔴🔴🔴🔴🔴🔴🔴🔴🔴🔴🔴🔴🔴🔴🔴🔴🔴🔴🔴🔴🔴🔴
بسم الله الرحمن الرحيم
الأخ المهندس/ موسى جميل عبدالله نعمان العواضي الأكرم
تحية طيبة وبعد
نرفع اليكم وثيقة اصلاح موقع مدرسة 
لاصلاح موقع لمدرسة التعاون الأساسية النخلة ـ العدين ـ إب

رفعت هذه الوثيقة يوم السبت ٢٨ / رجب / ١٤٤٧ ﻫ
 ➖➖➖➖➖➖➖➖➖➖➖➖➖➖➖➖➖➖➖

فقرة إدارة المستخدمين يكتبها ويشاهدها المدير 
ـ اسم المستخدم
ـ كلمة السر
ـ اليوم قائمة منسدلة ( السبت الأحد الإثنين الثلاثاء الأربعاء الخميس ـ الجمعة )
ـ التاريخ قائمة منسدلة  ( هجري ميلادي)

ـ الصلاحيات
فقرة ....
🔲 معاينة 🔲 تحرير 🔲 تعديل 🔲 إضافة 🔲 تصدير 🔲 حذف 🔲 فتحPDF 🔲 فتحExcel 🔲 طباعة 🔲 مشاركة

فقرة ......
وهكذا لجميع الفقرات
 ➖➖➖➖➖➖➖➖➖➖➖➖➖➖➖➖➖➖➖
🌹🌹🌹 ملف الملاحظات 
بسم الله الرحمن الرحيم
سعادة المهندس موسى الأكرم
تحية طيبة وبعد:
حفظك الله ورعاك
شوف الملفات التي هي بنظري مهمة
وضروري اطلع عليها مسبقا هي:-
١ـ ملف أسماء التقارير التي ستصدرها كل فقرة
٢ـ ملف شكل التقارير التي ستصدرها كل فقرة لمعرفة هل تكون متوافقة مع نظام المدرسة أولا
٣ـ ملف الصلاحيات لكي اعرف صلاحيات المدير والموظف والمشاهد وأولياء الأمور
٤ـ ملف بكيفية ترتيب فقرات الموقع
٥ـ ملف الاتصال بدون انترنت لمعرفة ماهي الفقرات الممكنة
٦ـ ملف الاضافات لكي أعرف ماهي الاضافات الاخرى التي ستكون غير النقاط المذكورة
٧ـ 


👈  محتوى صفحة الدخول
١ـ ( البيانات الأولية ـ دخول المدير دخول الموظف فقرة أولياء الأمور ـ فقرة المشاهد 
٢ـ يكون به شاشة بأعلى الصفحة الأولى مكتوب فيها اليوم والتاريخ الهجري والميلادي والساعة
٣ـ أكتب بأول الموقع ( أهلا بك في موقع مدرسة التعاون الأساسية النخلة ـ العدين ـ إب) بخط بارز عريض

👈  محتوى الصفحة الأولى 
١ـ وضع فيها جميع الفقرات الرئيسية والفرعية
٢ـ اكتب فيها بشريط يكون ثابت
مؤسس الموقع الأستاذ / خالد شبيطه عام ١٤٤٧ ﻫ ٢٠٢٥ ـ ٢٠٢٦ م
٣ـ عمل محرك بحث عن فقرات وعناصر النظام بمجرد تبحث يوصلك بسرعة للفقرة البحث السريع خطي وصوتي داخل الموقع
٤ـ يكون هناك بالصفحة تحكم بنوع وحجم الخط ويكون من صلاحيات المدير
٥ـ بأول صفحة تظهر للمدير حجم الاستضافة وحجم السعة المخزونة وحجم الفارغة
٦ـ اظهار تلقائي للمدير فقط المتصلين الآن بالموقع 

م	
الاسم	
رقم الجوال	
اليوم	
التأريخ	الزمن من	إلى	
					ثانية	دقيقة	ساعة	ثانية	دقيقة	ساعة	
											
											
											
٧ـ اظهار للمدير الأعمال التي قام بها كل موظف بالموقع هذا يكون يظهر تلقائي للمدير

م	
الاسم	
أعماله	
اليوم	
التأريخ	تم موافقة المدير	ملاحظات
						
						
						
٨ـ يكون برأس الموقع شريط يعرض  للمدير والموظف أذكار وتنبيهات وغيرها ويدخلها المدير فقط مثل 
سبحان الله وبحمده ـ سبحان الله العظيم ـ اللهم صل على محمد ـ لا إله إلا الله محمد رسول الله ـ مدرسة التعاون الأساسية ترحب بكم ـ 
٩ـ يكون برأس الموقع شريط دوار يعرض للموظف تنبيهات وغيرها ويدخلها المدير فقط ويشاهدها الموظف مثل غدا موعد ..... وغيرها

👈 ملاحظات عامة
١ـ إمكانية العمل وتصفح الموقع دون اتصال بالشبكة بمعنى ـ أن يكون المدير والموظف يشتغلوا بالموقع حتى لو الشبكة مغلقة ولما يفتح الشبكة يتحمل ما عمله
١ـ ارسال واستلام الإشعارات من الموقع عبر رقمي 730261508 أو 711963095 برسالة نصية أو واتساب 
٣ـ إضافة أيقونة (واتساب ـ فيس بوك ـ تلجرام ـ ماسنجر وغيرها ) قابلة للنقر في شريط الموقع
٤ـ ادارة التحميلات بشكل سهل باظهار شريط التحميل على الموقع 
٥ـ عمل للمدير قائمة يمكنه من ادخال فيها من يريد حظره عن الموقع
٦ـ امكانية قدرة المدير على ارسال الرسائل القصيرة سواء SMS او واتساب
٧ـ إمكانية إغلاق الموقع من قبل المدير متى شاء
٨ـ تكون فقرات الفقرة مجتمعة كاملة بصفحة وبمكان واحد
٩ـ دخول المدير يكون من أي جهاز وليس جهاز معين 
١٠ـ دخول الموظف يكون من أي جهاز وليس جهاز معين
١١ـ المتصفح يكون أي متصفح وليس متصفح معين
١٢ـ حجب الاعلانات من الظهور
١٣ـ تمكين خيار المزامنة من داخل الموقع بحيث لو كان خيار المزامنة غير مفعل بإعدادات الجوال 
١٤ـ تفعيل خيار المزامنة عند الموظف وعدم السماح للموظف بتوقيفه
١٥ـ أن يكون الموقع مقتصد بالشبكة ولا يصرف
١٦ـ أن يكون الموقع يغلق تلقائي عند الموظف بعد مرور ٢٤ ساعة ويطلب منه ادخال كلمة السر
١٧ـ اضافة اشعار للمدير بالموظفين المستخدمين للموقع وعدم اظهار هذا الاشعار للموظف
١٨ـ يفضل تكون عدد صفحات الموقع قليل
١٩ـ عمل سلة للمحذوفات للموقع ويكون خيارات الاستراد وخيارات المشاهدة وخيارات التفريغ من صلاحية المدير فقط 
٢٠ـ ملاحظة عندما يطلب الموظف قبول ماتم به لابد أن تكون هذه الخيارات تابتة للمدير وهي
مشاهدة العمل ـ تعديل العمل ـ موافق ـ غير موافق ـ ادخال ملاحظات على العمل للموظف
وتكون بهذا الشكل
مشاهدة العمل	تعديل العمل	موافق	غير موافق
ملاحظات	
ولما يطلع الموظف على هذا يكون الموقع يعطي اشعار للمدير بأن الموظف اطلع عليه
 

٢١ـ أن يكون حجم كلمات النص المطلوب ادخاله في (نص) يكون مفتوح وغير مقيد
٢٢ـ ملاحظة عند الضغط على الاشعار يكون الموقع ينقلك لهذه الفقرة مباشرة
٢٣ـ يدعم الموقع خاصية التكبير والتصغير 
٢٤ـ يدعم الموقع خاصية وضع العرض الأفقي والرأسي
٢٥ـ يدعم الموقع حفظ القوائم التي تم فتحها مؤخرًا للرجوع إليها بسرعة.
٢٦ـ يدعم الموقع اطلاع المستخدم بكل جديد من وزارة التربية والتعليم
٢٧ـ يدعم الموقع حفظ البيانات في بيئة مشفّرة وآمنة.
٢٨ـ تدعيم الموقع بحافظة نسخ يكون يحفظ بدخالها كل ما تم نسخه للرجوع له في حال لصقه
٢٩ـ تكون قوائم الموقع نفس حق الكمبيوتر
٣٠ـ تصميم الموقع للكمبيوتر والايفون والجوال 
٣١ـ ظهور اشعار للمدير بمن منحه صلاحية 
٣٢ـ دعم الموقع بقائمة تصميم بوجهة اللغة العربية
٣٣ـ وضع تلقائي ليلي نهاري
٣٤ـ يدعم الموقع على ضبط الوقت بالموقع والجوال على الشبكة تلقائي  
٣٥ـ يدعم الموقع حفظ ما طلب منه تلقائي في مجلد اسمه ( ما تم حفظه من برنامج المدرسة ) وسواء كان هذا المجلد في الذاكرة الداخلية أو الخارجية
٣٦ـ يدعم الموقع عمل للمدير وللموظف تقرير جلسة بما قام به يكون يظهر عند الضغط على زر تقرير جلسة
ويكون كل موظف يطلع على تقرير جلسته فقط والمدير يطلع على تقرير جلسات الجميع
٣٧ـ يدعم الموقع تصدير نسخة للجوال للرجوع اليها وقت الطلب
٣٨ـ لوحة التحكم والاعدادات تكون تظهر للمدير فقط ويعدلها المدير فقط دون الموظف
٣٩ـ يدعم الموقع حساب زمن ووقت اتصال كل موظف بالموقع ويكون يظهره للمدير
٤٠ـ يكون الموقع يظهر اشعار المدير بالاعمال التي نفذت بالموقع والتي لم تنفذ
٤١ـ يكون الموقع يعطي اشعار متصل بالشبكة او غير متصل
٤٢ـ أن يتم اخراج الموقع بطريقة سهلة
٤٣ـ أن يكون الخط في عناوين الفقرات بخط النسخ الواضح والبارز
٤٤ـ أن تكون خيارات فرز الأسماء
بحسب الحروف الابجدية
بحسب الرقم المسلسل
بحسب تأريخ الاضافة
بحسب اسم الطالب الطالبة
بحسب اسم الاب
بحسب اسم الجد
بحسب القرية
بحسب المحلة
ويكون ما سبق 
بحسب طلاب
بحسب طالبات
بحسب طلاب وطالبات
ويكون ما سبق
بحسب صف
بحسب صفوف مختارة
بحسب جميع الصفوف

٤٥ـ بكل مكان يكون الطلاب لوحدهم والطالبات لوحدهن
٤٦ـ يكون الموقع يعطي للمدير اشعار بمن ارتبط بالموقع مباشرة حال ارتباطه
٤٧ـ دعم الموقع ليشتغل بجوال الايفون لان البعض لديهم هذا الجوال
٤٨ـ دعم الموقع ليسمح بالحفظ في الذاكرة الخارجية تلقائي
٤٩ـ دعم الموقع باخراج او مشاهدة تقويم هجري وميلادي ليوم لأسبوع لشهر لعام
٥٠ـ دعم الموقع بمنبه يكون بداخله يكون ينبه متى ما نريد 
العمل اليوم التأريخ الساعة وقت التنبيه ص م تكرار التنبيه   صوت المنبه ادخال 
وهذا يكون للمدير او الموظف بدون اي قيد أو شرط
الفكرة نفس نظام هذ الموقع الموجود بهذا الرابط https://play.google.com/store/apps/details?id=com.sa_firstappalarm_ne.mun.my_alarm_application
 
٥١ـ ادخال تحت كل فقرة التقارير المتاحة لها
٥٢ـ تدعيم الموقع ب مذكرة الأعمال 
يكتب هذه الفقرة الموظف
يشاهدها الموظف
يعدلها الموظف
يضيف لها أو يحذف منها المدير
يطلع عليها ويصدر تقاريرها المدير

وتكون على النحو التالي
مذكرة الأعمال
ـ مسلسل رقم متسلسل تلقائي
ـ الاعمال قائمة منسدلة  ( نص قائمة)
ـ مكان التنفيذ قائمة منسدلة  ( مدرسة إدارة صف راحة منزل اخرى)
ـ اليوم قائمة منسدلة ( السبت الأحد الإثنين الثلاثاء الأربعاء الخميس )
ـ التاريخ قائمة منسدلة  ( هجري ميلادي)
ـ الأسبوع ( الأول  الثاني الثالث الرابع)  
ـ الشهر قائمة منسدلة  ( محرم صفر ربيع أول ربيع ثاني جماد أول جماد ثاني رجب شعبان)
ـ الفصل الدراسي قائمة منسدلة  ( الأول الثاني)
ـ نفذ قائمة منسدلة  ( نعم لا)

تاريخ الإضافة الخيارات التي يتم تثبيتها ( هجري ميلادي) الساعة 
زر ( معاينة ) لعرض العمل والنظر فيه قبل تصديره الخيارات التي يتم تثبيتها ورقة × ورقة ـ أو ورقتين  × ورقة ـ أو ثلاث ورق  × ورقة ـ أو أربع ورق  × ورقة

ـ تقرير لموظف لموظفين مختارين ـ لجميع الموظفين ـ ليوم أسبوع شهر  ـفصل ـ لعام
ـ تصدير التقرير يكون ورقة × ورقة أو ورقتين  × ورقة أو ثلاث ورق  × ورقة أو أربع ورق  × ورقة
ـ تصديرها لملف الخيارات التي يتم تثبيتها (  Excel PDF )
ـ الاحتفاظ بصور الاشعارات للرجوع لها إذا لزم

٥٣ـ تدعيم الموقع بلوحة تصميم شهائد تقدير
١ـ أرشفة شهادة  
 ادخال صورة شهادة
١ـ مسلسل رقم متسلسل تلقائي
٢ـ اسم صاحب الشهادة/ ادخال
٣ـ الصفة قائمة منسدلة (طالب طالبة معلم معلمة صف إدارة اخرى)
٤ـ جهة اصدارها قائمة منسدلة ( المدرسة أخرى)
٥ـ عنوان الشهادة قائمة منسدلة (نص قائمة)
٦ـ اتجاه الشهادة قائمة منسدلة  (عمودي أفقي )
٧ـ الفصل الدراسي قائمة منسدلة  ( الأول الثاني)
٨ـ الشهر قائمة منسدلة  ( محرم صفر ربيع أول ربيع ثاني جماد أول جماد ثاني رجب شعبان)
٩ـ الأسبوع ( الأول  الثاني الثالث الرابع)  
١٠ـ اليوم قائمة منسدلة ( السبت الأحد الإثنين الثلاثاء الأربعاء الخميس )
١١ـ التاريخ قائمة منسدلة  ( هجري ميلادي)
١٢ـ ملاحظات قائمة منسدلة (نص قائمة)

٢ـ تصميم شهادة
ـ اسم صاحب الشهادة/ ادخال
ـ الصفة قائمة منسدلة (طالب طالبة معلم معلمة صف إدارة اخرى)
ـ جهة اصدارها قائمة منسدلة ( المدرسة أخرى)
ـ عنوان الشهادة قائمة منسدلة (نص قائمة صورة )
ـ اتجاه الشهادة قائمة منسدلة (عمودي أفقي )
ـ البسملة قائمة منسدلة ( نص صورة
ـ الاطار  ادخال  ( صورة
ـ نص الشهادة قائمة منسدلة (نص قائمة)

🎄🎄🎄🎄🎄🎄🎄🎄🎄🎄🎄🎄🎄🎄 فاصل بين الفقرات الرئيسية 🎄🎄🎄🎄🎄🎄🎄🎄🎄🎄🎄🎄🎄
👈 اسم الموقع
بالعربي أوالانجليزي 
ـ School أو مدرسة
أو
ـ نظام مدرسة
🎄🎄🎄🎄🎄🎄🎄🎄🎄🎄🎄🎄🎄🎄 فاصل بين الفقرات الرئيسية 🎄🎄🎄🎄🎄🎄🎄🎄🎄🎄🎄🎄🎄

الهدف : نظام متكامل لإدارة بيانات المدرسة والموظفين والطلاب والطالبات

🎄🎄🎄🎄🎄🎄🎄🎄🎄🎄🎄🎄🎄🎄 فاصل بين الفقرات الرئيسية 🎄🎄🎄🎄🎄🎄🎄🎄🎄🎄🎄🎄🎄

واجهة الموقع
العام الدراسي	 ١٤٤٧ ﻫ ٢٠٢٥ ـ ٢٠٢٦ م
اسم المدرسة	التعاون الأساسية
رمز المدرسة	٧١١١٤٠١٠
مؤسس المدرسة	المرحوم بإذن الله صالح حميد شبيطه
تأريخ التأسيس	١٤٠٤ ﻫ ١٩٨٢ ـ ١٩٨٣ م
صورة المدرسة



هذه الواجهة يشاهدها المدير والموظف
ويكون تحريرها من صلاحية المدير فقط وله أن يغير فقرة او صورة ويستبدلها باخرى

ملاحظات 
رمز المدرسة : مثلا رمز المدرسة حقنا ٧١١١٤٠١٠ وربط هذا الرمز بالتقارير ويكون خياري لوضعه في تقارير دون غيرهم
مؤسس المدرسة : قائمة المنسدلة (المرحوم بإذن الله صالح حميد شبيطه ـ أخرى)


المستخدمون 
المدير     كلمة السر 
الموظف  كلمة السر المستلمة من المدير والمدرجة في بيانات الموظف يتم ربط هذا

🎄🎄🎄🎄🎄🎄🎄🎄🎄🎄🎄🎄🎄🎄 فاصل بين الفقرات الرئيسية 🎄🎄🎄🎄🎄🎄🎄🎄🎄🎄🎄🎄🎄

ـ اضف خيارات العرض بالارشيف ( قائمة شبكة تفاصيل)
ـ اضافة جوار أي ارشيف خيارات ( فرز ـ تعديل ـ حذف  )

🎄🎄🎄🎄🎄🎄🎄🎄🎄🎄🎄🎄🎄🎄 فاصل بين الفقرات الرئيسية 🎄🎄🎄🎄🎄🎄🎄🎄🎄🎄🎄🎄🎄

🔴 التقارير :-
من يكتب هذه الفقرة	المدير أو من يكلفه من الموظفين في حينه وموافقة المدير على الادخال أو التعديل
من يشاهد تقارير هذه الفقرة	المدير
من يصدر تقارير هذه الفقرة	المدير
خيارات البحث بهذه الفقرة يكون بواسطة 	فقرة كلمة جملة
التقارير المطلوبة من هذه الفقرة	تقرير ليوم أسبوع شهر فصل عام

👈 هذه الفقرات موضحة بمكانها وكيفيتها ادنى هذا 
ـ كتابة البسملة رأس التقرير بالوسط 
ـ كتابة برأس التقرير اليوم والتاريخ الهجري والميلادي والساعة
ـ كتابة اعلا التقرير بالجهة اليمنى
الجمهورية اليمنية
وزارة التربية والتعليم
مكتب التربية والتعليم بمحافظة إب
الإدارة التعليمية بمديرية العدين
مدرسة التعاون الأساسية
ـ عمل الطير الجمهوري اعلا التقرير بالوسط يمين
ـ عمل شعار المناهج اعلا التقرير بالجهة اليسرى شمال
ـ عمل رأس التقرير رقم الصفحة ( ١ ) من ( هنا العدد الكلي لاوراق التقرير )
ـ عمل رأس التقرير رقم النموذج (عدد متسلسل)
ـ اختياري يكتب اسفل التقرير. بالجهة اليسرى(مدير المدرسة الأستاذ / عبدالقادر أحمد مرشد مثنى أو اخرى
ـ عنوان التقرير يكون بوسط رأس الورقة وقائمة منسدلة  لما تم تحديده أخرى
ـ يكون يكتب تلقائي باسفل الورقة للتقرير بقية بالورقة الثانية إذا كان أكثر من ورقة
ـ يكون يكتب تلقائي بأعلى الورقة الثانية تابع تقرير كذا إذا كان أكثر من ورقة
ـ أن يكتب رأس التقرير تلقائي العام الدراسي مثلا ١٤٤٧ ﻫ ٢٠٢٥ ـ ٢٠٢٦ م
ـ مقدمة التقرير
ـ الخاتمة
ـ جهة التقرير قائمة منسدلة  عام موظف موظفين إدارة المدرسة أخرى

ـ فقرة ختم المدرسة (ملف / صورة) وليس ملف فقط
شكل الورقة الأولى بالتقرير


شكل الورقة الثانية بالتقرير






👈 وهذه الفقرات الأخرى للتقارير
ـ عرض معاينة للتقرير وعدد صفحاته
ـ تصدير التقرير بالورقة بشكل عمودي أفقي
ـ تصدير التقرير ورقة × ورقة أو ورقتين  × ورقة أو ثلاث ورق  × ورقة أو أربع ورق  × ورقة 
ـ هوامش التقرير قائمة منسدلة  افتراضي يحدد بوقته يمين شمال فوق تحت
ـ تكون صفحات التقرير مرقمة تلقائي
ـ أن يكون صلاحية عمل التقرير للمدير فقط
ـ أن يحدد خلفية التقرير بوقته قائمة منسدلة  افتراضي نص صورة
ـ رقم الخط من النظام
ـ اسم الخط من النظام
ـ نوع الخط بارز غامق
ـ يكون التقرير بورق A4 او A3

ـ تقرير لشعبة لصف لصفوف مختارة لجميع الصفوف ويكون يومي أسبوعي شهري فصلي
ـ تقرير لموظف لموظفين مختارين لجميع الموظفين ويكون يومي أسبوعي شهري فصلي
ـ تقرير لحضور وغياب طلاب طالبات صف صفوف مختارة جميع الصفوف
ـ يكون التقرير لطالب لطالبة لطلاب طالبات طلاب وطالبات صف صفوف مختارة جميع الصفوف
ـ يكون التقرير لفقرة لفقرات مختارة لجميع الفقرات
ـ يكون التقرير لشهر لترم لعام
ـ يكون التقرير لما يكتب في حينه ل ..... و .......و ......و ......  

ـ تصدير التقرير (  Excel PDF )
ـ نص التقرير قائمة منسدلة نص بدون اخرى

ـ يكون الموقع يحتفظ بنسخة من التقرير تلقائي لو نرجع نشوف التقارير اللي رفعت نحصلهن
ـ أن يكون الموقع بجميع فقراته يمكننا من عمل تقرير لأي شي

🎄🎄🎄🎄🎄🎄🎄🎄🎄🎄🎄🎄🎄🎄 فاصل بين الفقرات الرئيسية 🎄🎄🎄🎄🎄🎄🎄🎄🎄🎄🎄🎄🎄

🔴 اشعارات الموقع:-

ملاحظة عندما يطلب الموظف قبول ماتم به
يكون هناك خيارات للمدير
مشاهدة العمل ـ تعديل العمل ـ موافق ـ غير موافق ـ ادخال ملاحظات على العمل للموظف
بهذا الشكل
مشاهدة العمل	تعديل العمل	موافق	غير موافق
ملاحظات	
ولما يطلع الموظف على هذا يكون الموقع يعطي اشعار للمدير بأن الموظف اطلع عليه

١٣ـ اشعارات الموقع
مكان ظهورها أعلا الشاشة
مدة ظهورها يحدده المدير
مثلا الموظف محمد قام بعمل كذا
مثلا الموظف محمد متصل بالموقع الآن
مثلا الموظف محمد طلب كذا هل توافق
مثل مطلوب من الموظف محمد عمل كذا
مدة استخدامك للبرنامج كذا
🎄🎄🎄🎄🎄🎄🎄🎄🎄🎄🎄🎄🎄🎄 فاصل بين الفقرات الرئيسية 🎄🎄🎄🎄🎄🎄🎄🎄🎄🎄🎄🎄🎄
🔴 فقرات وعناصر الموقع مرتبة
نظام المدرسة
🔴 ١ ـ المدرسة
١  ـ ١ ـ بيانات المدرسة
١  ـ ٢ ـ وضع الفصول الدراسية في بداية ونهاية العام
١  ـ ٣ ـ الإتلافات
١  ـ ٤ ـ الاصلاحات

🔴 ٢ ـ الخطة
٢ـ ١ـ الخطة العامة لإدارة المدرسة
٢ـ ٢ـ الخطة العامة للعاملين بالمدرسة
٢ـ ٣ـ خطة وأعمال الأسبوع
٢ـ ٤ـ خطة تعليم القراءة
٢ـ ٥ـ الايجابيات السلبيات المقترحات

🔴 ٣ ـ الموظفين
٣  ـ ١ ـ بيانات الموظفين
٣  ـ ٢ ـ جدول الحصص الأسبوعي
٣  ـ ٣ ـ حضور وغياب الموظفين
٣  ـ ٤ ـ تحضيرالموظفين للدروس
٣  ـ ٥ ـ الواجبات المنزلية
٣  ـ ٦ ـ تصحيح الدفاتر
٣  ـ ٧ ـ مخالفات الموظفين

🔴 ٤ ـ الطلاب والطالبات
٤  ـ ١ ـ بيانات الطلاب والطالبات
٤  ـ ٢ ـ حضور وغياب
٤  ـ ٣ ـ كتب الطلاب والطالبات
٤  ـ ٤ ـ أداء الطلاب والطالبات لفقرات طابور الصباح
٤  ـ ٥ ـ حالات الاسعافات الأولية
٤  ـ ٦ ـ الدليل الإرشادي
٤  ـ ٧ ـ الإشعارات الصادرة لأولياء الأمور
٤  ـ ٨ ـ إدارة الصفوف الدراسية
٤  ـ ٩ ـ الوسائل التعليمية

🔴 ٥ ـ الاختبارات
٥  ـ ١ ـ الاختبارات الشهرية
٥  ـ ٢ ـ الامتحانات الفصلية 
٥  ـ ٣ ـ توزيع الطلاب والطالبات على لجان الامتحان

🔴 ٦ ـ الدرجات
٦  ـ ١ ـ المحصلات الشهرية
٦  ـ ٢ ـ درجة الفصل الأول
٦  ـ ٣ ـ درجة الفصل الثاني

🔴 ٧ ـ النتائج
٧  ـ ١ ـ نتائج الفصل الأول
٧  ـ ٢ ـ نتائج الفصل الثاني
٧  ـ ٣ ـ الشهائد

🔴 ٨ ـ الزائرين
🔴 ٩ ـ المشكلات


🔴  ١٠  ـ ١ ـ مجلس الآباء
🔴 ١١  ـ ١ ـ النظافة المدرسية
🔴 ١٢  ـ ١ ـ الإذاعة الصباحية
🔴 ١٣  ـ ١ ـ المسابقات المدرسية
🔴 ١٤  ـ ١ ـ الاجتماعات المدرسية
🔴 ١٥  ـ ١ ـ الأنشطة المدرسية
🔴 ١٦  ـ ١ ـ المساهمة المجتمعية
🔴  ١٧  ـ ١ ـ النشاط المرافق لكل صف لإستغلال عطلة نهاية الفصل الدراسي الأول

🔴 ١٨ ـ صندوق
١٨  ـ ١ ـ الصادر للموظفين
١٨  ـ ٢ ـ الوارد من الموظفين

🔴 ١٩ ـ توثيق أعمال العام
🔴 ٢٠ ـ الملخصات الدراسية المتوفرة



المشاهد
أولياء الأمور


التقارير
الاشعارات

🎄🎄🎄🎄🎄🎄🎄🎄🎄🎄🎄🎄🎄🎄 فاصل بين الفقرات الرئيسية 🎄🎄🎄🎄🎄🎄🎄🎄🎄🎄🎄🎄🎄

نظام المدرسة
🔴 ١ ـ المدرسة
١  ـ ١ ـ بيانات المدرسة
من يكتب هذه الفقرة	المدير أو من يكلفه من الموظفين في حينه
من يشاهد تقارير هذه الفقرة	المدير
من يصدر تقارير هذه الفقرة	المدير
خيارات البحث بهذه الفقرة يكون بواسطة 	كلمة 

نوع التقارير المطلوب من الموقع تصديرها من هذه الفقرة
تصدير تقرير بجميع بيانات المدرسة
تصدير تقرير أدوات اخرى بالمدرسة
تصدير تقرير بالبصمات الثابتة
تصدير تقرير وثائق رفعت للجهة المختصة عن المدرسة

العام الدراسي :مثلا  ١٤٤٧ ﻫ ٢٠٢٥ ـ ٢٠٢٦ م
اسم المدرسة:
رمز المدرسة:
مؤسس المدرسة:
تأريخ التأسيس : هجري ميلادي
المرحلة الدراسية:  أساسي ثانوي كلاهما
مبنى المدرسة : حكومي أهلي مختلط 
نظام التعليم. قائمة منسدلة (طلاب طالبات مختلط )
الفترة: صباحي مسائي كلاهما
عنوان المدرسة : القرية/الحارة العزلة المديرية المحافظة
اسم مدير المدرسة: 

م	عدد المباني	عدد الأدوار	عدد غراف أدارية	عدد الفصول	غرفة معمل	استراحة	حمامات	غرفة حارس	مخزن	غرفة وسائل	غرفة انشطة	قاعة تدريب 	سكن مدرسين
العدد													

م	خزانات ماء	مقاعد	سبورات	دولاب				
	حديد	بلاستيك	مزدوج	فردي	قلمية	عادية	حديد	خشب				
العدد												

ملاحظة :- هناك صورة مرفقة للاطلاع لو حبيت تضيف شي منها


أدوات اخرى بالمدرسة:- 
م	البيان	العدد	ملاحظات
			

بصمات ثابتة 
م	البصمات الثابتة	الاسم	العام	ملاحظات
				

وثائق رفعت للجهة المختصة عن المدرسة لهذا العام ادخال 
م	عنوان الوثيقة	الجهة	اليوم	التاريخ	ملاحظات
					

👈عمل جنب هذه الفقرة مايلي:-
تصدير لعام ـ تصدير لفصل ـ استرداد من عام استرداد من فصل ـ استرداد الفقرة كاملة ـ استرداد فقرة مختارة ـ استرداد جميع الفقرات

عمل جنب هذه الفقرة بيانات الاضافة:-
ـ تاريخ الإضافة قائمة منسدلة  ( هجري ميلادي) الساعة 
ـ اليوم قائمة منسدلة ( السبت الأحد الإثنين الثلاثاء الأربعاء الخميس الجمعة )
ـ اسم الموظف الذي اضاف : 

عمل جنب هذه الفقرة هذه الخيارات:-
ـ زر ( معاينة ) لعرض العمل والنظر فيه قبل تصديره الخيارات التي يتم تثبيتها ورقة × ورقة ـ أو ورقتين  × ورقة ـ أو ثلاث ورق  × ورقة ـ أو أربع ورق  × ورقة
ـ تصدير التقرير يكون ورقة × ورقة. ـ أو ورقتين  × ورقة ـ  أو ثلاث ورق  × ورقة ـ أو أربع ورق  × ورقة
ـ تصدير عمود أعمدة مختارة جميع الأعمدة قائمة منسدلة  (  Excel PDF )
ـ تصديرها لملف قائمة منسدلة  (  Excel PDF )
ـ تصدير للفصل الثاني ـ تصدير للعام القادم 
ـ استيراد من الفصل الأول استيراد من العام الماضي
ـ مشاركة قائمة منسدلة  (  Excel PDF صورة )
ـ الاحتفاظ بصور الاشعارات للرجوع لها إذا لزم

عمل جنب هذه الفقرة هذه الازرار
🔲 معاينة 🔲 تحرير 🔲 تعديل 🔲 إضافة 🔲 تصدير 🔲 حذف 🔲 فتحPDF 🔲 فتحExcel 🔲 طباعة 🔲 مشاركة
➖➖➖➖➖➖➖➖➖➖➖ فاصل بين الفقرات الفرعية ➖➖➖➖➖➖➖➖➖➖➖➖➖
١  ـ ٢ ـ وضع الفصول الدراسية في بداية ونهاية العام
من يكتب هذه الفقرة	مسؤل الصف وموافقة المدير على الادخال أو التعديل
من يشاهد تقارير هذه الفقرة	المدير
من يصدر تقارير هذه الفقرة	المدير
اضافة لهذه الفقرة 	استرداد من العام السابق ـ ترحيل لفصل اخر ـ ترحيل لنهاية العام ـ ترحيل للعام القادم
خيارات البحث بهذه الفقرة يكون بواسطة 	فقرة كلمة جملة

نوع التقارير المطلوب من الموقع تصديرها من هذه الفقرة
تصدير تقرير بداية العام فصل فصول مختارة جميع الفصول
تصدير تقرير نهاية العام فصل فصول مختارة جميع الفصول
تصدير تقرير الأقفال بداية العام نهاية العام لفصل لفصول مختارة لجميع الفصول
تصدير تقرير الأبواب بداية العام نهاية العام لفصل لفصول مختارة لجميع الفصول
تصدير تقرير أدوات الإضاءة بداية العام نهاية العام لفصل لفصول مختارة لجميع الفصول
تصدير تقرير السبورات بداية العام نهاية العام لفصل لفصول مختارة لجميع الفصول
تصدير تقرير النوافذ بداية العام نهاية العام لفصل لفصول مختارة لجميع الفصول
تصدير تقرير القاعات بداية العام نهاية العام لفصل لفصول مختارة لجميع الفصول
تصدير تقرير الجدران بداية العام نهاية العام لفصل لفصول مختارة لجميع الفصول
تصدير تقرير أدوات النظافة بداية العام نهاية العام لفصل لفصول مختارة لجميع الفصول

👈عمل جنب هذه الفقرة مايلي:-
تصدير لعام ـ تصدير لفصل ـ استرداد من عام استرداد من فصل ـ استرداد الفقرة كاملة ـ استرداد فقرة مختارة ـ استرداد جميع الفقرات

فترة التقييم قائمة منسدلة ( بداية العام ـ نهاية العام )
المدرسة قائمة منسدلة ( القديمة الجديدة)
الدور قائمة منسدلة ( الأول ـ الثاني ـ الثالث ـ الرابع الخامس )
رقم الفصل عدد متسلسل

(  ١  ) القفل
موجود قائمة منسدلة ( نعم لا)
حجمه قائمة منسدلة (كبير وسط صغير)
وضعه قائمة منسدلة (جديد مستعمل)
حالته قائمة منسدلة (سليم مصدء)
مكانه قائمة منسدلة (داخلي خارجي) 
يفتح بمفتاح آخر قائمة منسدلة ( نعم لا)

تصدير تقرير الأقفال بداية العام نهاية العام لفصل لفصول مختارة لجميع الفصول  
استرداد من العام السابق ـ ترحيل لفصل اخر ـ ترحيل لنهاية العام ـ ترحيل للعام القادم

(  ٢  ) الباب
مكتوب اسمه فوق الباب قائمة منسدلة نعم لا
نوعه قائمة منسدلة حديد أخرى 
ثباته قائمة منسدلة محكم وسط ضعيف
طلائه الخارجي قائمة منسدلة جيد مستعمل منتهي
طلائه الداخلي قائمة منسدلة جيد مستعمل منتهي
الهندراب قائمة منسدلة خارجي داخلي كلاهما
مسكة قائمة منسدلة خارجي داخلي كلاهما
وضعه قائمة منسدلة سليم مصدء منتهي
وجود فراغ قائمة منسدلة فوق تحت كلاهما

تصدير تقرير الأبواب بداية العام نهاية العام لفصل لفصول مختارة لجميع الفصول  
استرداد من العام السابق ـ ترحيل لفصل اخر ـ ترحيل لنهاية العام ـ ترحيل للعام القادم

(  ٣  ) أدوات الإضاءة
موجودة قائمة منسدلة (لا نعم ) 
عددها (رقم متسلسل)
وضعها قائمة منسدلة سليمة عاطلة
عدد مفاتيح الكهرباء (رقم متسلسل)
عدد فيشات الكهرباء (رقم متسلسل)
وضع مفاتيح الكهرباء قائمة منسدلة سليمة وسط منتهي

تصدير تقرير أدوات الإضاءة بداية العام نهاية العام لفصل لفصول مختارة لجميع الفصول  
استرداد من العام السابق ـ ترحيل لفصل اخر ـ ترحيل لنهاية العام ـ ترحيل للعام القادم

(  ٤  ) السبورة
موجودة قائمة منسدلة (لا نعم ) 
نوعها قائمة منسدلة قلمية طباشير
ثباتها قائمة منسدلة محكم وسط ضعيف
حالتها قائمة منسدلة سليمة مخدوشة ـ عدد الخدش (رقم متسلسل)
حامل السبورة قائمة منسدلة موجود ثابت مرتخي
 
تصدير تقرير السبورات بداية العام نهاية العام لفصل لفصول مختارة لجميع الفصول  
استرداد من العام السابق ـ ترحيل لفصل اخر ـ ترحيل لنهاية العام ـ ترحيل للعام القادم

(  ٥  ) النوافذ
الجهة قائمة منسدلة أمام خلف أخرى
رقم النافذة (رقم متسلسل)
وضعها قائمة منسدلة سليم تالف
زجاجها قائمة منسدلة سليم تالف
اليدات قائمة منسدلة سليمة تالفة
وجود حماية حديد قائمة منسدلة (لا نعم )
حالة الحماية قائمة منسدلةسليم تالف
وجود حماية شبك قائمة منسدلة (لا نعم )
حالة الشبك قائمة منسدلة سليم تالف
نظافتها من الداخل قائمة منسدلة نظيفة وسط متسخة
نظافتها من الخارج قائمة منسدلة نظيفة وسط متسخة

تصدير تقرير النوافذ بداية العام نهاية العام لفصل لفصول مختارة لجميع الفصول  
استرداد من العام السابق ـ ترحيل لفصل اخر ـ ترحيل لنهاية العام ـ ترحيل للعام القادم

(  ٦  ) القاعة
نوعها بلط اسمنت
حالتها قائمة منسدلة سليمة مخدوشة ـ عدد الخدش (رقم متسلسل)
الفراش موجود قائمة منسدلة نعم لا
حالة الفراش قائمة منسدلة جديد مستعمل منتهي
المقاعد موجودة قائمة منسدلة نعم لا نوعها قائمة منسدلة فردي زوجي عددها (رقم متسلسل)
حالة المقاعد قائمة منسدلة سليمة وسط منتهية
المقاعد كافية قائمة منسدلة نعم لا
عدد مقاعد العجز (رقم متسلسل)

تصدير تقرير القاعات بداية العام نهاية العام لفصل لفصول مختارة لجميع الفصول  
استرداد من العام السابق ـ ترحيل لفصل اخر ـ ترحيل لنهاية العام ـ ترحيل للعام القادم


(  ٧  ) الجدران
جهة الجدار قائمة منسدلة جدار الباب جدار خلف جدار يسار جدار أمام سطح
طلاءه قائمة منسدلة سليم مخدوش عدد الخدش (رقم متسلسل)
الكتابة قائمة منسدلة خالي مكتوب عدد الكتابات (رقم متسلسل)
وجود وسائل ثابتة قائمة منسدلة نعم لا
ملصق ثابتة قائمة منسدلة نعم لا
نقش ثابت قائمة منسدلة نعم لا
تسرب ماء من السطح نعم عدد ثقوب(رقم متسلسل) لا
حامل وسائل موجود قائمة منسدلة نعم لا
حالته قائمة منسدلة ثابت مرتخي
حزام البلط قائمة منسدلة سليم تالف
ممر جوار الفصل قائمة منسدلة سليم مخدوش عدد الخدش (رقم متسلسل)

تصدير تقرير الجدران بداية العام نهاية العام لفصل لفصول مختارة لجميع الفصول  
استرداد من العام السابق ـ ترحيل لفصل اخر ـ ترحيل لنهاية العام ـ ترحيل للعام القادم

(  ٨  ) أدوات النظافة
موجودة قائمة منسدلة نعم لا
عدد المكانس (رقم متسلسل)
عدد سلة القمامة (رقم متسلسل)
عدد جاروف (رقم متسلسل)
تصدير تقرير أدوات النظافة بداية العام نهاية العام لفصل لفصول مختارة لجميع الفصول  
استرداد من العام السابق ـ ترحيل لفصل اخر ـ ترحيل لنهاية العام ـ ترحيل للعام القادم

عمل جنب هذه الفقرة بيانات الاضافة:-
ـ تاريخ الإضافة قائمة منسدلة  ( هجري ميلادي) الساعة 
ـ اليوم قائمة منسدلة ( السبت الأحد الإثنين الثلاثاء الأربعاء الخميس الجمعة )
ـ اسم الموظف الذي اضاف : 

عمل جنب هذه الفقرة هذه الخيارات:-
ـ زر ( معاينة ) لعرض العمل والنظر فيه قبل تصديره الخيارات التي يتم تثبيتها ورقة × ورقة ـ أو ورقتين  × ورقة ـ أو ثلاث ورق  × ورقة ـ أو أربع ورق  × ورقة
ـ تصدير التقرير يكون ورقة × ورقة. ـ أو ورقتين  × ورقة ـ  أو ثلاث ورق  × ورقة ـ أو أربع ورق  × ورقة
ـ تصدير عمود أعمدة مختارة جميع الأعمدة قائمة منسدلة  (  Excel PDF )
ـ تصديرها لملف قائمة منسدلة  (  Excel PDF )
ـ تصدير للفصل الثاني ـ تصدير للعام القادم 
ـ استيراد من الفصل الأول استيراد من العام الماضي
ـ مشاركة قائمة منسدلة  (  Excel PDF صورة )
ـ الاحتفاظ بصور الاشعارات للرجوع لها إذا لزم

عمل جنب هذه الفقرة هذه الازرار
🔲 معاينة 🔲 تحرير 🔲 تعديل 🔲 إضافة 🔲 تصدير 🔲 حذف 🔲 فتحPDF 🔲 فتحExcel 🔲 طباعة 🔲 مشاركة

➖➖➖➖➖➖➖➖➖➖➖ فاصل بين الفقرات الفرعية ➖➖➖➖➖➖➖➖➖➖➖➖➖
١  ـ ٣ ـ الإتلافات
من يكتب هذه الفقرة	المدير أو من يكلفه من الموظفين في حينه وموافقة المدير على الادخال أو التعديل
من يشاهد تقارير هذه الفقرة	المدير
من يصدر تقارير هذه الفقرة	المدير
خيارات البحث بهذه الفقرة يكون بواسطة 	فقرة كلمة جملة  

نوع التقارير المطلوب من الموقع تصديرها من هذه الفقرة
تصدير تقرير الإتلافات المدرسية ليوم لأسبوع لشهر لفصل للعام

١ـ مسلسل رقم متسلسل تلقائي
٢ـ بيان ما أتلف (نص قائمة)
٣ـ الفصل الدراسي قائمة منسدلة  ( الأول الثاني)
٤ـ الشهر قائمة منسدلة  ( محرم صفر ربيع أول ربيع ثاني جماد أول جماد ثاني رجب شعبان)
٥ـ الأسبوع ( الأول  الثاني الثالث الرابع)  
٦ـ اليوم قائمة منسدلة ( السبت الأحد الإثنين الثلاثاء الأربعاء الخميس )
٧ـ التاريخ يكتب تلقائي  ( هجري ميلادي)
٨ـ المكان (نص قائمة)
٩ـ التفاصيل (نص قائمة)
١٠ ـ ملاحظات ادخال يدوي قائمة منسدلة نص قائمة

عمل جنب هذه الفقرة بيانات الاضافة:-
ـ تاريخ الإضافة قائمة منسدلة  ( هجري ميلادي) الساعة 
ـ اليوم قائمة منسدلة ( السبت الأحد الإثنين الثلاثاء الأربعاء الخميس الجمعة )
ـ اسم الموظف الذي اضاف : 

عمل جنب هذه الفقرة هذه الخيارات:-
ـ زر ( معاينة ) لعرض العمل والنظر فيه قبل تصديره الخيارات التي يتم تثبيتها ورقة × ورقة ـ أو ورقتين  × ورقة ـ أو ثلاث ورق  × ورقة ـ أو أربع ورق  × ورقة
ـ تصدير التقرير يكون ورقة × ورقة. ـ أو ورقتين  × ورقة ـ  أو ثلاث ورق  × ورقة ـ أو أربع ورق  × ورقة
ـ تصدير عمود أعمدة مختارة جميع الأعمدة قائمة منسدلة  (  Excel PDF )
ـ تصديرها لملف قائمة منسدلة  (  Excel PDF )
ـ تصدير للفصل الثاني ـ تصدير للعام القادم 
ـ استيراد من الفصل الأول استيراد من العام الماضي
ـ مشاركة قائمة منسدلة  (  Excel PDF صورة )
ـ الاحتفاظ بصور الاشعارات للرجوع لها إذا لزم

عمل جنب هذه الفقرة هذه الازرار
🔲 معاينة 🔲 تحرير 🔲 تعديل 🔲 إضافة 🔲 تصدير 🔲 حذف 🔲 فتحPDF 🔲 فتحExcel 🔲 طباعة 🔲 مشاركة
➖➖➖➖➖➖➖➖➖➖➖ فاصل بين الفقرات الفرعية ➖➖➖➖➖➖➖➖➖➖➖➖➖

١  ـ ٤ ـ الاصلاحات
من يكتب هذه الفقرة	المدير أو من يكلفه من الموظفين في حينه وموافقة المدير على الادخال أو التعديل
من يشاهد تقارير هذه الفقرة	المدير
من يصدر تقارير هذه الفقرة	المدير
خيارات البحث بهذه الفقرة يكون بواسطة 	فقرة كلمة جملة


نوع التقارير المطلوب من الموقع تصديرها من هذه الفقرة
تصدير تقرير الاصلاحات المدرسية ليوم لأسبوع لشهر لفصل للعام

١ـ مسلسل رقم متسلسل تلقائي
٢ـ بيان ما اصلح (نص قائمة)
٣ـ الفصل الدراسي قائمة منسدلة  ( الأول الثاني)
٤ـ الشهر قائمة منسدلة  ( محرم صفر ربيع أول ربيع ثاني جماد أول جماد ثاني رجب شعبان)
٥ـ الأسبوع ( الأول  الثاني الثالث الرابع)  
٦ـ اليوم قائمة منسدلة ( السبت الأحد الإثنين الثلاثاء الأربعاء الخميس )
٧ـ التاريخ يكتب تلقائي  ( هجري ميلادي)
٨ـ المكان (نص قائمة)
٩ـ التفاصيل (نص قائمة)
١٠ المبلغ ( رقما حرفا ) ملاحظة هنا لما نكتب المبلغ رقما يكون يكتب المبلغ حرفا تلقائي
١١ ـ ملاحظات ادخال يدوي قائمة منسدلة نص قائمة


عمل جنب هذه الفقرة بيانات الاضافة:-
ـ تاريخ الإضافة قائمة منسدلة  ( هجري ميلادي) الساعة 
ـ اليوم قائمة منسدلة ( السبت الأحد الإثنين الثلاثاء الأربعاء الخميس الجمعة )
ـ اسم الموظف الذي اضاف : 

عمل جنب هذه الفقرة هذه الخيارات:-
ـ زر ( معاينة ) لعرض العمل والنظر فيه قبل تصديره الخيارات التي يتم تثبيتها ورقة × ورقة ـ أو ورقتين  × ورقة ـ أو ثلاث ورق  × ورقة ـ أو أربع ورق  × ورقة
ـ تصدير التقرير يكون ورقة × ورقة. ـ أو ورقتين  × ورقة ـ  أو ثلاث ورق  × ورقة ـ أو أربع ورق  × ورقة
ـ تصدير عمود أعمدة مختارة جميع الأعمدة قائمة منسدلة  (  Excel PDF )
ـ تصديرها لملف قائمة منسدلة  (  Excel PDF )
ـ تصدير للفصل الثاني ـ تصدير للعام القادم 
ـ استيراد من الفصل الأول استيراد من العام الماضي
ـ مشاركة قائمة منسدلة  (  Excel PDF صورة )
ـ الاحتفاظ بصور الاشعارات للرجوع لها إذا لزم

عمل جنب هذه الفقرة هذه الازرار
🔲 معاينة 🔲 تحرير 🔲 تعديل 🔲 إضافة 🔲 تصدير 🔲 حذف 🔲 فتحPDF 🔲 فتحExcel 🔲 طباعة 🔲 مشاركة

🎄🎄🎄🎄🎄🎄🎄🎄🎄🎄🎄🎄🎄🎄 فاصل بين الفقرات الرئيسية 🎄🎄🎄🎄🎄🎄🎄🎄🎄🎄🎄🎄🎄

🔴 ٢ ـ الخطة
٢ـ ١ـ الخطة العامة لإدارة المدرسة
من يكتب هذه الفقرة	المدير أو من يكلفه من الموظفين في حينه وموافقة المدير على الادخال أو التعديل
من يشاهد تقارير هذه الفقرة	المدير
من يصدر تقارير هذه الفقرة	المدير
خيارات البحث بهذه الفقرة يكون بواسطة 	الجانب فقرة كلمة جملة
اضافة لهذه الفقرة	تصدير لعام ـ استرداد من عام ـ استرداد الفقرة كاملة ـ استرداد فقرة مختارة ـ استرداد جميع الفقرات

نوع التقارير المطلوب من الموقع تصديرها من هذه الفقرة
تصدير تقرير أعمال يوم أعمال أسبوع أعمال شهر أعمال فصل أعمال العام

١ـ مسلسل رقم متسلسل تلقائي
٢ـ الأعمال قائمة منسدلة ( نص قائمة )
٣ـ الفصل الدراسي قائمة منسدلة (الأول الثاني ) 
٤ـ الشهر ( محرم صفر ربيع أول ربيع ثاني جماد أول جماد ثاني رجب شعبان)
٥ـ الأسبوع ( الأول  الثاني الثالث الرابع) 
٦ـ اليوم ( السبت الأحد الإثنين ـ الثلاثاء الأربعاء الخميس )
٧ـ التاريخ يكتب تلقائي  ( هجري ميلادي)
٨ـ  ملاحظات ادخال يدوي قائمة منسدلة نص قائمة

👈عمل جنب هذه الفقرة مايلي:-
تصدير لعام ـ تصدير لفصل ـ استرداد من عام استرداد من فصل ـ استرداد الفقرة كاملة ـ استرداد فقرة مختارة ـ استرداد جميع الفقرات

عمل جنب هذه الفقرة بيانات الاضافة:-
ـ تاريخ الإضافة قائمة منسدلة  ( هجري ميلادي) الساعة 
ـ اليوم قائمة منسدلة ( السبت الأحد الإثنين الثلاثاء الأربعاء الخميس الجمعة )
ـ اسم الموظف الذي اضاف : 

عمل جنب هذه الفقرة هذه الخيارات:-
ـ زر ( معاينة ) لعرض العمل والنظر فيه قبل تصديره الخيارات التي يتم تثبيتها ورقة × ورقة ـ أو ورقتين  × ورقة ـ أو ثلاث ورق  × ورقة ـ أو أربع ورق  × ورقة
ـ تصدير التقرير يكون ورقة × ورقة. ـ أو ورقتين  × ورقة ـ  أو ثلاث ورق  × ورقة ـ أو أربع ورق  × ورقة
ـ تصدير عمود أعمدة مختارة جميع الأعمدة قائمة منسدلة  (  Excel PDF )
ـ تصديرها لملف قائمة منسدلة  (  Excel PDF )
ـ تصدير للفصل الثاني ـ تصدير للعام القادم 
ـ استيراد من الفصل الأول استيراد من العام الماضي
ـ مشاركة قائمة منسدلة  (  Excel PDF صورة )
ـ الاحتفاظ بصور الاشعارات للرجوع لها إذا لزم

عمل جنب هذه الفقرة هذه الازرار
🔲 معاينة 🔲 تحرير 🔲 تعديل 🔲 إضافة 🔲 تصدير 🔲 حذف 🔲 فتحPDF 🔲 فتحExcel 🔲 طباعة 🔲 مشاركة
➖➖➖➖➖➖➖➖➖➖➖ فاصل بين الفقرات الفرعية ➖➖➖➖➖➖➖➖➖➖➖➖➖
٢ـ ٢ـ الخطة العامة للعاملين بالمدرسة
من يكتب هذه الفقرة	المدير أو من يكلفه من الموظفين في حينه وموافقة المدير على الادخال أو التعديل
من يشاهد هذه الفقرة	الموظف
من يشاهد تقارير هذه الفقرة	المدير
من يصدر تقارير هذه الفقرة	المدير
خيارات البحث بهذه الفقرة يكون بواسطة 	جانب فقرة كلمة جملة
اضافة لهذه الفقرة	تصدير لعام ـ استرداد من عام ـ استرداد الفقرة كاملة ـ استرداد فقرة مختارة ـ استرداد جميع الفقرات

نوع التقارير المطلوب من الموقع تصديرها من هذه الفقرة
تصدير تقرير أعمال يوم أعمال أسبوع أعمال شهر أعمال فصل أعمال العام

١ـ مسلسل رقم متسلسل تلقائي
٢ـ الأعمال قائمة منسدلة ( نص قائمة )
٣ـ الفصل الدراسي قائمة منسدلة (الأول الثاني ) 
٤ـ الشهر ( محرم صفر ربيع أول ربيع ثاني جماد أول جماد ثاني رجب شعبان)
٥ـ الأسبوع ( الأول  الثاني الثالث الرابع) 
٦ـ اليوم ( السبت الأحد الإثنين ـ الثلاثاء الأربعاء الخميس )
٧ـ التاريخ يكتب تلقائي  ( هجري ميلادي)
٨ ـ ملاحظات ادخال يدوي قائمة منسدلة نص قائمة

👈عمل جنب هذه الفقرة مايلي:-
تصدير لعام ـ تصدير لفصل ـ استرداد من عام استرداد من فصل ـ استرداد الفقرة كاملة ـ استرداد فقرة مختارة ـ استرداد جميع الفقرات

عمل جنب هذه الفقرة بيانات الاضافة:-
ـ تاريخ الإضافة قائمة منسدلة  ( هجري ميلادي) الساعة 
ـ اليوم قائمة منسدلة ( السبت الأحد الإثنين الثلاثاء الأربعاء الخميس الجمعة )
ـ اسم الموظف الذي اضاف : 

عمل جنب هذه الفقرة هذه الخيارات:-
ـ زر ( معاينة ) لعرض العمل والنظر فيه قبل تصديره الخيارات التي يتم تثبيتها ورقة × ورقة ـ أو ورقتين  × ورقة ـ أو ثلاث ورق  × ورقة ـ أو أربع ورق  × ورقة
ـ تصدير التقرير يكون ورقة × ورقة. ـ أو ورقتين  × ورقة ـ  أو ثلاث ورق  × ورقة ـ أو أربع ورق  × ورقة
ـ تصدير عمود أعمدة مختارة جميع الأعمدة قائمة منسدلة  (  Excel PDF )
ـ تصديرها لملف قائمة منسدلة  (  Excel PDF )
ـ تصدير للفصل الثاني ـ تصدير للعام القادم 
ـ استيراد من الفصل الأول استيراد من العام الماضي
ـ مشاركة قائمة منسدلة  (  Excel PDF صورة )
ـ الاحتفاظ بصور الاشعارات للرجوع لها إذا لزم

عمل جنب هذه الفقرة هذه الازرار
🔲 معاينة 🔲 تحرير 🔲 تعديل 🔲 إضافة 🔲 تصدير 🔲 حذف 🔲 فتحPDF 🔲 فتحExcel 🔲 طباعة 🔲 مشاركة
➖➖➖➖➖➖➖➖➖➖➖ فاصل بين الفقرات الفرعية ➖➖➖➖➖➖➖➖➖➖➖➖➖
٢ـ ٣ـ خطة وأعمال الأسبوع
من يكتب هذه الفقرة	المدير أو من يكلفه من الموظفين في حينه وموافقة المدير على الادخال أو التعديل
من يشاهد هذه الفقرة	الموظف
من يشاهد تقارير هذه الفقرة	المدير
من يصدر تقارير هذه الفقرة	المدير
خيارات البحث بهذه الفقرة يكون بواسطة 	فقرة كلمة جملة
اضافة لهذه الفقرة	تصدير لعام ـ استرداد من عام ـ استرداد الفقرة كاملة ـ استرداد فقرة مختارة ـ استرداد جميع الفقرات

نوع التقارير المطلوب من الموقع تصديرها من هذه الفقرة
تصدير تقرير خطة وأعمال الأسبوع ليوم لأسبوع لشهر لفصل لعام

١ـ مسلسل رقم متسلسل تلقائي
٢ـ أعمال خلال الأسبوع كامل قائمة منسدلة ( نص قائمة )
٣ـ الأعمال اليومية قائمة منسدلة ( نص قائمة )
٤ـ المكلف قائمة منسدلة اسم الموظف تلقائي المعلمين المعلمات الجميع أخرى
٥ـ الفصل الدراسي قائمة منسدلة (الأول الثاني ) 
٦ـ الشهر ( محرم صفر ربيع أول ربيع ثاني جماد أول جماد ثاني رجب شعبان)
٧ـ الأسبوع ( الأول  الثاني الثالث الرابع) 
٨ـ اليوم ( السبت الأحد الإثنين ـ الثلاثاء الأربعاء الخميس )
٩ـ التاريخ يكتب تلقائي  ( هجري ميلادي)
١٠ـ ملاحظات ادخال يدوي قائمة منسدلة نص قائمة
وهذا شكل التقرير اللي يكون يظهر لهذه الفقرة
	الاعمال	المكلف
👈 الأسبوع الرابع
ما ينفذ خلال
هذا الأسبوع		
السبت
٢٢ / جماد ثاني ١٤٤٧ ﻫ
١٣ / ديسمبر ١٢ / ٢٠٢٥ م		
الأحد
٢٣ / جماد ثاني ١٤٤٧ ﻫ
١٤ / ديسمبر ١٢ / ٢٠٢٥ م		
الإثنين
٢٤ / جماد ثاني ١٤٤٧ ﻫ
١٥ / ديسمبر ١٢ / ٢٠٢٥ م		
الثلاثاء
٢٥ / جماد ثاني ١٤٤٧ ﻫ
١٦ / ديسمبر ١٢ / ٢٠٢٥ م		
الأربعاء
٢٦ / جماد ثاني ١٤٤٧ ﻫ
١٧ / ديسمبر ١٢ / ٢٠٢٥ م		

👈عمل جنب هذه الفقرة مايلي:-
تصدير لعام ـ تصدير لفصل ـ استرداد من عام استرداد من فصل ـ استرداد الفقرة كاملة ـ استرداد فقرة مختارة ـ استرداد جميع الفقرات

عمل جنب هذه الفقرة بيانات الاضافة:-
ـ تاريخ الإضافة قائمة منسدلة  ( هجري ميلادي) الساعة 
ـ اليوم قائمة منسدلة ( السبت الأحد الإثنين الثلاثاء الأربعاء الخميس الجمعة )
ـ اسم الموظف الذي اضاف : 

عمل جنب هذه الفقرة هذه الخيارات:-
ـ زر ( معاينة ) لعرض العمل والنظر فيه قبل تصديره الخيارات التي يتم تثبيتها ورقة × ورقة ـ أو ورقتين  × ورقة ـ أو ثلاث ورق  × ورقة ـ أو أربع ورق  × ورقة
ـ تصدير التقرير يكون ورقة × ورقة. ـ أو ورقتين  × ورقة ـ  أو ثلاث ورق  × ورقة ـ أو أربع ورق  × ورقة
ـ تصدير عمود أعمدة مختارة جميع الأعمدة قائمة منسدلة  (  Excel PDF )
ـ تصديرها لملف قائمة منسدلة  (  Excel PDF )
ـ تصدير للفصل الثاني ـ تصدير للعام القادم 
ـ استيراد من الفصل الأول استيراد من العام الماضي
ـ مشاركة قائمة منسدلة  (  Excel PDF صورة )
ـ الاحتفاظ بصور الاشعارات للرجوع لها إذا لزم

عمل جنب هذه الفقرة هذه الازرار
🔲 معاينة 🔲 تحرير 🔲 تعديل 🔲 إضافة 🔲 تصدير 🔲 حذف 🔲 فتحPDF 🔲 فتحExcel 🔲 طباعة 🔲 مشاركة
➖➖➖➖➖➖➖➖➖➖➖ فاصل بين الفقرات الفرعية ➖➖➖➖➖➖➖➖➖➖➖➖➖
٢ـ ٤ـ خطة تعليم القراءة
من يكتب هذه الفقرة	المدير أو من يكلفه من الموظفين في حينه وموافقة المدير على الادخال أو التعديل
من يشاهد هذه الفقرة	الموظف
من يشاهد تقارير هذه الفقرة	المدير
من يصدر تقارير هذه الفقرة	المدير
خيارات البحث بهذه الفقرة يكون بواسطة 	فقرة كلمة جملة
اضافة لهذه الفقرة	تصدير لعام ـ استرداد من عام ـ استرداد الفقرة كاملة ـ استرداد فقرة مختارة ـ استرداد جميع الفقرات

نوع التقارير المطلوب من الموقع تصديرها من هذه الفقرة
تصدير تقرير بخطة القراءة لصف لصفوف مختارة ـ لجميع الصفوف ـ لفصل ـ لعام

١ـ الفصل الدراسي قائمة منسدلة  ( الأول الثاني)
٢ـ الشهر قائمة منسدلة  ( محرم صفر ربيع أول ربيع ثاني جماد أول جماد ثاني رجب شعبان)
٣ـ الأسبوع ( الأول  الثاني الثالث الرابع)  
٤ـ اليوم قائمة منسدلة ( السبت الأحد الإثنين الثلاثاء الأربعاء الخميس )
٥ـ التاريخ يكتب تلقائي  ( هجري ميلادي)
٦ـ الموقع : نص
٧ـ  تم تنفيذ الموقع قائمة منسدلة  ( نعم ـ لا )
٨ـ ـ ملاحظات ادخال يدوي قائمة منسدلة نص قائمة

👈عمل جنب هذه الفقرة مايلي:-
تصدير لعام ـ تصدير لفصل ـ استرداد من عام استرداد من فصل ـ استرداد الفقرة كاملة ـ استرداد فقرة مختارة ـ استرداد جميع الفقرات

عمل جنب هذه الفقرة بيانات الاضافة:-
ـ تاريخ الإضافة قائمة منسدلة  ( هجري ميلادي) الساعة 
ـ اليوم قائمة منسدلة ( السبت الأحد الإثنين الثلاثاء الأربعاء الخميس الجمعة )
ـ اسم الموظف الذي اضاف : 

عمل جنب هذه الفقرة هذه الخيارات:-
ـ زر ( معاينة ) لعرض العمل والنظر فيه قبل تصديره الخيارات التي يتم تثبيتها ورقة × ورقة ـ أو ورقتين  × ورقة ـ أو ثلاث ورق  × ورقة ـ أو أربع ورق  × ورقة
ـ تصدير التقرير يكون ورقة × ورقة. ـ أو ورقتين  × ورقة ـ  أو ثلاث ورق  × ورقة ـ أو أربع ورق  × ورقة
ـ تصدير عمود أعمدة مختارة جميع الأعمدة قائمة منسدلة  (  Excel PDF )
ـ تصديرها لملف قائمة منسدلة  (  Excel PDF )
ـ تصدير للفصل الثاني ـ تصدير للعام القادم 
ـ استيراد من الفصل الأول استيراد من العام الماضي
ـ مشاركة قائمة منسدلة  (  Excel PDF صورة )
ـ الاحتفاظ بصور الاشعارات للرجوع لها إذا لزم

عمل جنب هذه الفقرة هذه الازرار
🔲 معاينة 🔲 تحرير 🔲 تعديل 🔲 إضافة 🔲 تصدير 🔲 حذف 🔲 فتحPDF 🔲 فتحExcel 🔲 طباعة 🔲 مشاركة
➖➖➖➖➖➖➖➖➖➖➖ فاصل بين الفقرات الفرعية ➖➖➖➖➖➖➖➖➖➖➖➖➖
٢ـ ٥ـ الايجابيات السلبيات المقترحات

من يكتب هذه الفقرة	المدير أو من يكلفه من الموظفين في حينه وموافقة المدير على الادخال أو التعديل
من يشاهد هذه الفقرة	الموظف
من يشاهد تقارير هذه الفقرة	المدير
من يصدر تقارير هذه الفقرة	المدير
خيارات البحث بهذه الفقرة يكون بواسطة 	الجانب الموضوع ايجابيات سلبيات مقترحات فقرة كلمة جملة
اضافة لهذه الفقرة	تصدير لعام ـ استرداد من عام ـ استرداد الفقرة كاملة ـ استرداد فقرة مختارة ـ استرداد جميع الفقرات

نوع التقارير المطلوب من الموقع تصديرها من هذه الفقرة
تصدير تقرير الايجابيات لموضوع لمواضيع مختارة لجميع المواضيع
تصدير تقرير السلبيات لموضوع لمواضيع مختارة لجميع المواضيع
تصدير تقرير المقترحات لموضوع لمواضيع مختارة لجميع المواضيع
تصدير تقرير الايجابيات السلبيات المقترحات لموضوع لمواضيع مختارة لجميع المواضيع

ـ مسلسل رقم متسلسل تلقائي
ـ الجانب قائمة منسدلة نص قائمة
ـ الموضوع قائمة منسدلة نص قائمة
١ـ الايجابيات 
مسلسل رقم متسلسل تلقائي
النص قائمة منسدلة نص قائمة
٢ـ السلبيات مسلسل النص
مسلسل رقم متسلسل تلقائي
النص قائمة منسدلة نص قائمة
٣ـ المقترحات مسلسل النص
مسلسل رقم متسلسل تلقائي
النص قائمة منسدلة نص قائمة

👈👈👈 وهذا مثال لهذه الفقرة
مسلسل ( ١ ) الجانب/ عام الموضوع/ ماء المدرسة
* أولا الإيجابيات
1.	تعاون ومساهمة الشيخ فؤاد فارس بتملئة خزانات المدرسة بالماء
2.	تفاعل ومساهمة الأخ زياد صاحب الوائت باحضار الماء لخزانات المدرسة
3.	وجود خزانات الماء المناسبة بسطح المدرسة ودورة المياه
4.	إغلاق خزانات الماء بطريقة صحيحة
5.	قيام الإدارة بشراء أقفال ووضعها على حنفيات الشرب
6.	تفاعل فاعل خير مع الإدارة وشراء حنفي للشرب مكان المكسور من عزاء الشهيد محمد عبدالله لطف مع قفل صغير ١٤٤٦ ﻫ
7.	تفاعل فاعل خير مع الإدارة بإصلاح الحنفي الثاني لماء الشرب للطلاب/الطالبات ١٤٤٦ ﻫ
8.	تفاعل فاعل خير مع الإدارة بإحضار خزان الماء مع قاعدته ووضعه بدرجة المدرسة القديمة ١٤٤٦ ﻫ
9.	متابعة خزانات المياه بسطح المدرسة أسبوعيًا حتى لا تتعطل بشكل مفاجىء
10.	قيام الإدارة بشراء الأدوات وتفاعل إحدى المعلمات بعمل الرسم المعبر جوار حنفيات ماء الشرب
11.	إصلاح غطاءات خزانات البلاستيك لإحكام إغلاقها
12.	تواجد دوح الماء بإدارة المدرسة
13.	تكليف المعلمة سرور بتملئة الدوح صباح يوم السبت من كل أسبوع
14.	وجود إثنين حنفيات لماء الشرب للطلاب/الطالبات
15.	وجود نافورة مصغرة بجوار حنفي ماء الشرب
16.	وجود حابس في خزان الماء بسطح المدرسة لإغلاق الماء هناك وقت الطلب
17.	وجود حابس بالدور الثاني بالمدرسة القديمة لإغلاق الماء خارج الدوام
18.	وجود خزانات الماء المناسبة بسطح المدرسة ودورة المياه
19.	قيام المعلم/المعلمة بتوجيه الطلاب/الطالبات بعدم الإسراف بالماء
👈 [[ ١ ]]ـ توجيه المربين/الرواد للطلاب/الطالبات التعليمات الخاصة بالماء وهي:-
1.	أن أصحطب معي زمزمية ماء أو دبة صحي حتى لا ألجئ للخروج
2.	أن أحافظ على وعاء شربي وعلى سلامته
3.	أن اصطحب بطاقة الإذن عند خروجي للشرب
4.	أن ألتزم بدوري أثناء الإنتظار دون التقدم على أحد احترام لنفسي وللآخرين
5.	أن أفتح حنفي الماء بطريقة صحيحة وسليمة
6.	أن أحافظ على الماء وأقتصد فيه
7.	أن أفتح حنفي الماء بقدر حاجتي فقط
8.	أن أغسل اليدين قبل أن أشرب
9.	أن أغسل أوعية الماء قبل أن أشرب بها
10.	أن أقتصد بالماء ولا أسرف فيه
11.	أن أسمي الله قبل أن أشرب وأقول بسم الله
12.	أن أحمد الله و أشكره على نعمة الماء
13.	أن أغلق حنفي الماء بعد شربي اغلاقا تاما
14.	أن أتاكد من إغلاق حنفي الماء بعد استخدامه 
15.	أن أحافظ على حنفي الماء وعلى سلامته
16.	أن أحافظ على أوعية الشرب وعلى سلامتها
17.	أن لا أشرب من الحنفي إذا كان حارًا
* ثانيًا السلبيات
لا يوجد
* ثالثًا المقترحات للعمل بها بإذن الله
1.	عدم تواجد الوائت بوقت الطلب لنقل الماء وإيصاله للمدرسة 
2.	عدم وجود خزان كبير بلاستيك لوضعه لشرب الطلاب بدل خزان الحديد الموجود حاليًا
3.	عدم وجود صندوق حماية لحنفيات ماء الشرب من لعب الآخرين به خارج الدوام
4.	طرح لفاعل خير شراء الأدوات المطلوبة وإيصال خط الماء إلى المدرسة بقصب من أقرب خط الشيخ فؤاد
5.	طرح لفاعل خير شراء الأدوات المطلوبة وإيصال خط الماء من السطح بقصب بلاستيك ثابتة لخزان الماء البلاستيك بأسفل الدرجة
6.	طرح لفاعل خير شراء الأدوات المطلوبة وتحويل حنفيات الشرب تجنبًا لوقوع الماء لممر المدرسة

👈عمل جنب هذه الفقرة مايلي:-
تصدير لعام ـ تصدير لفصل ـ استرداد من عام استرداد من فصل ـ استرداد الفقرة كاملة ـ استرداد فقرة مختارة ـ استرداد جميع الفقرات

عمل جنب هذه الفقرة بيانات الاضافة:-
ـ تاريخ الإضافة قائمة منسدلة  ( هجري ميلادي) الساعة 
ـ اليوم قائمة منسدلة ( السبت الأحد الإثنين الثلاثاء الأربعاء الخميس الجمعة )
ـ اسم الموظف الذي اضاف : 

عمل جنب هذه الفقرة هذه الخيارات:-
ـ زر ( معاينة ) لعرض العمل والنظر فيه قبل تصديره الخيارات التي يتم تثبيتها ورقة × ورقة ـ أو ورقتين  × ورقة ـ أو ثلاث ورق  × ورقة ـ أو أربع ورق  × ورقة
ـ تصدير التقرير يكون ورقة × ورقة. ـ أو ورقتين  × ورقة ـ  أو ثلاث ورق  × ورقة ـ أو أربع ورق  × ورقة
ـ تصدير عمود أعمدة مختارة جميع الأعمدة قائمة منسدلة  (  Excel PDF )
ـ تصديرها لملف قائمة منسدلة  (  Excel PDF )
ـ تصدير للفصل الثاني ـ تصدير للعام القادم 
ـ استيراد من الفصل الأول استيراد من العام الماضي
ـ مشاركة قائمة منسدلة  (  Excel PDF صورة )
ـ الاحتفاظ بصور الاشعارات للرجوع لها إذا لزم

عمل جنب هذه الفقرة هذه الازرار
🔲 معاينة 🔲 تحرير 🔲 تعديل 🔲 إضافة 🔲 تصدير 🔲 حذف 🔲 فتحPDF 🔲 فتحExcel 🔲 طباعة 🔲 مشاركة
🎄🎄🎄🎄🎄🎄🎄🎄🎄🎄🎄🎄🎄🎄 فاصل بين الفقرات الرئيسية 🎄🎄🎄🎄🎄🎄🎄🎄🎄🎄🎄🎄🎄


🔴 ٣ ـ الموظفين

٣  ـ ١ ـ بيانات الموظفين
من يكتب هذه الفقرة	المدير أو من يكلفه من الموظفين في حينه وموافقة المدير على الادخال أو التعديل
من يشاهد هذه الفقرة	المدير
من يشاهد تقارير هذه الفقرة	المدير
من يصدر تقارير هذه الفقرة	المدير
خيارات البحث بهذه الفقرة يكون بواسطة 	موظف فقرة كلمة جملة
اضافة لهذه الفقرة	تصدير لعام ـ استرداد من عام ـ استرداد موظف مختار ـ استرداد موظفين مختارين ـ استرداد جميع الموظفين

نوع التقارير المطلوب من الموقع تصديرها من هذه الفقرة
تصدير تقرير موظف موظفين مختارين جميع الموظفين


١ـ مسلسل رقم متسلسل تلقائي
٢ـ اسم الموظف حسب الهوية
٣ـ الجنس قائمة منسدلة ( ذكر أنثى)
٤ـ الرقم الوظيفي
٥ـ الرقم المالي
٦ـ المؤهل قائمة منسدلة ( خبرة أساسي ثانوي دبلوم معلمين دبلوم عالي جامعي أخرى)
٧ـ تأريخ المؤهل مثلا ١٩٩٤ ـ ١٩٩٥ م
٨ـ صورة المؤهل ادخال من ملف
٩ـ التخصص قائمة منسدلة ( عام أخرى) 
١٠ـ رقم الهاتف قائمة منسدلة ( الأول الثاني) ملاحظة اعمل ادخال من سجل الهاتف
١١ـ تاريخ الميلاد. مثلا انا أكتب هنا ١٩٧٢ فيطلع بالعمر ٥٣  والعكس صحيح
١٢ـ العمر ٥٣
١٣ـ نوع الهوية قائمة منسدلة ( شخصية جواز عائلية أخرى) 
١٤ـ رقم الهوية
١٥ـ صورة الهوية ادخال من ملف
١٦ـ تأريخ انتهاء الهوية مثلا ٣١ / ١ / ٢٠٢٥ م
١٧ـ عدد سنوات الخبرة. يطلع رقم تلقائي وفقا لتاريخ المؤهل مع امكانية تعديله إذا لزم
١٨ـ الحالة قائمة منسدلة ( ثابت متطوع )  بزيادة تاء التأنيث تلقائي وفقا للجنس
١٩ـ فترة الدوام قائمة منسدلة ( صباحي مسائي كلاهما)
٢٠ـ الحالة الإجتماعية عازب  أخرى متزوج بزيادة تاء التأنيث تلقائي وفقا للجنس وبعد متزوج قائمة منسدلة عدد الأولاد (ذكور إناث المجموع ) وهذا المجموع يكون يكتب ويطلع تلقائي ومحمي من التعديل
٢١ـ العنوان  (القرية قائمة منسدلة (وادي النخلة ـ النخلة ـ بردان ـ المفرق ـ المسرب ـ بيت الشبر ـ العوارض ـ اخرى ) المحلة قائمة منسدلة (اسماء المحلات لكل قرية ) ( العزلة قائمة منسدلة ( منهات ـ بني عواض ـ أخرى ) المديرية قائمة منسدلة ( العدين ـ أخرى ) المحافظة قائمة منسدلة ( إب ـ أخرى )
٢٢ـ تاريخ ابتداء العمل بالمدرسة مثلا ١٤٤٧ ﻫ ٢٠٢٥ ـ ٢٠٢٦ م
٢٣ـ عدد سنوات العمل بالمدرسة يطلع تلقائي من خلال كتابة تاريخ ابتداء العمل بالمدرسة مع امكانية تعديله إذا لزم
٢٤ـ العمل القائم به في المدرسة قائمة منسدلة ( مدير مشرف مشرفة معلم معلمة أخرى )
٢٥ـ المهامات الأخرى المكلف بها. مسلسل تلقائي ـ العمل ـ  ادخال تاريخ تكليفه بالعمل (هجري وميلاد) اليوم قائمة منسدلة ( السبت الأحد الإثنين ـ الثلاثاء الأربعاء الخميس )(ملاحظة اعمل للواحد أكثر من عمل
٢٦ـ الصف المكلف به قائمة منسدلة ( أول أساسي ثاني أساسي ثالث أساسي رابع أساسي خامس أساسي سادس أساسي سابع أساسي ثامن أساسي تاسع أساسي ـ أول ثانوي ـ ثاني ثانوي ـ ثالث ثانوي بدون)
٢٧ ـ الشعبة قائمة منسدلة ( أ ب ج بدون)
٢٨ـ مهامه تجاه صفه قائمة منسدلة (مربي مربية رائد رائدة معلم معلمة)
٢٩ـ الصفوف التي يدرسها (ملاحظة اعمل للواحد أكثر من أربعة صفوف
٣٠ـ المادة / المواد التي يدرسها (ملاحظة اعمل للواحد أكثر من ٤ مواد) والمواد في قائمة منسدلة ( القرآن الإسلامية  اللغة العربية اللغة الانجليزية الرياضيات العلوم الاجتماعيات أخرى )
٣١ـ عدد الحصص رقم مثلا ٢٢ وهذا الرقم يكون يكتب تلقائي وفقا لما كتب بفقرة جدول الحصص الأسبوعي
٣٢ـ حالته وفيها سليم ( نعم لا ) مريض ( نعم لا ) تفاصيل مرضه ( ادخال ) معاق( نعم لا ) يتيم( نعم لا )
٣٣ـ أبنائه وبناته بالمدرسة يطلع تلقائي من خلال اسم الطالب المطابق لاسم ابوه مع امكانية تعديله إذا لزم
٣٤ـ الدورات الحاصل عليها مسلسل اسم الدورة صورة الشهادة (ملاحظة اعمل للواحد أكثر من دورة
٣٥ـ مواهبه المتميز بها قائمة منسدلة ( الخط الرسم الإلقاء التمثيل الحفظ الإنشاد محاكات الأصوات ترتيل القرآن شاعر حرفة يدوية سائق حركات نادرة رياضة سباحة تطريز نقش خياطة صناعة تصاميم تصوير أخرى ) 
ملاحظة نعمل للواحد أكثر من موهبة ١ـ.    ٢ـ.   ٣ـ.   
٣٦ـ انتظم بعمله لهذا العام  قائمة منسدلة ( حتى نهاية الفصل الدراسي ( الأول الثاني أخرى) مع امكانية تعديله إذا لزم
٣٧ـ معه واتس قائمة منسدلة ( نعم لا )
٣٨ـ نزل اعتماده قائمة منسدلة ( نعم لا )
٣٩ـ المدرسة التي عمل بها سابقا ادخال
٤٠ـ كلمة السر الخاصة به لاستخدام الموقع ( ملاحظة يحددها المدير ولكل موظف كلمة مختلفة وللمدير صلاحية تعديلها أو تغيرها) واربط لا يقبل من المستخدم الا الكلمة المكتوبة هنا
٤١ـ استخدامه للبرنامج  قائمة منسدلة ( منح الصلاحية  توقيف الصلاحية) واربط فلا يكون يسمح للمستخدم استخدام الموقع الا وفق ما كتب هنا
٤٢ـ ملاحظات ادخال يدوي قائمة منسدلة نص قائمة

👈عمل جنب هذه الفقرة مايلي:-
تصدير لعام ـ تصدير لفصل ـ استرداد من عام استرداد من فصل ـ استرداد الفقرة كاملة ـ استرداد فقرة مختارة ـ استرداد جميع الفقرات

عمل جنب هذه الفقرة بيانات الاضافة:-
ـ تاريخ الإضافة قائمة منسدلة  ( هجري ميلادي) الساعة 
ـ اليوم قائمة منسدلة ( السبت الأحد الإثنين الثلاثاء الأربعاء الخميس الجمعة )
ـ اسم الموظف الذي اضاف : 

عمل جنب هذه الفقرة هذه الخيارات:-
ـ زر ( معاينة ) لعرض العمل والنظر فيه قبل تصديره الخيارات التي يتم تثبيتها ورقة × ورقة ـ أو ورقتين  × ورقة ـ أو ثلاث ورق  × ورقة ـ أو أربع ورق  × ورقة
ـ تصدير التقرير يكون ورقة × ورقة. ـ أو ورقتين  × ورقة ـ  أو ثلاث ورق  × ورقة ـ أو أربع ورق  × ورقة
ـ تصدير عمود أعمدة مختارة جميع الأعمدة قائمة منسدلة  (  Excel PDF )
ـ تصديرها لملف قائمة منسدلة  (  Excel PDF )
ـ تصدير للفصل الثاني ـ تصدير للعام القادم 
ـ استيراد من الفصل الأول استيراد من العام الماضي
ـ مشاركة قائمة منسدلة  (  Excel PDF صورة )
ـ الاحتفاظ بصور الاشعارات للرجوع لها إذا لزم

عمل جنب هذه الفقرة هذه الازرار
🔲 معاينة 🔲 تحرير 🔲 تعديل 🔲 إضافة 🔲 تصدير 🔲 حذف 🔲 فتحPDF 🔲 فتحExcel 🔲 طباعة 🔲 مشاركة
➖➖➖➖➖➖➖➖➖➖➖ فاصل بين الفقرات الفرعية ➖➖➖➖➖➖➖➖➖➖➖➖➖
٣  ـ ٢ ـ جدول الحصص الأسبوعي
من يكتب هذه الفقرة	المدير أو من يكلفه من الموظفين في حينه وموافقة المدير على الادخال أو التعديل
من يشاهد هذه الفقرة	الموظف يشاهد جدوله فقط ومسؤل الصف يشاهد جدول صفه فقط
من يشاهد تقارير هذه الفقرة	المدير
من يصدر تقارير هذه الفقرة	المدير
خيارات البحث بهذه الفقرة يكون بواسطة 	موظف مادة فقرة كلمة 
اضافة لهذه الفقرة	تصدير لعام ـ استرداد من عام ـ استرداد الفقرة كاملة ـ استرداد فقرة مختارة ـ استرداد جميع الفقرات

نوع التقارير المطلوب من الموقع تصديرها من هذه الفقرة
تصدير تقرير جدول موظف لموظفين مختارين لجميع الموظفين  
تصدير تقرير جدول مادة مواد مختارة لجميع المواد
تصدير تقرير جدول يوم لصف صفوف مختارة جميع الصفوف
تصدير تقرير جدول أسبوع لصف صفوف مختارة جميع الصفوف
تصدير تقرير بالحصص الفارغة لموظف لموظفين مختارين لجميع الموظفين



١ـ مسلسل رقم متسلسل تلقائي
٢ـ اسم الموظف
٣ـ الفصل الدراسي قائمة منسدلة  ( الأول الثاني)
٤ـ اليوم قائمة منسدلة ( السبت الأحد الإثنين الثلاثاء الأربعاء الخميس )
٥ـ التاريخ يكتب تلقائي  ( هجري ميلادي)
٦ـ الصف ( أول أساسي ثاني أساسي ثالث أساسي رابع أساسي خامس أساسي سادس أساسي سابع أساسي ثامن أساسي تاسع أساسي ـ أول ثانوي ـ ثاني ثانوي ـ ثالث ثانوي)
٧ـ الشعبة قائمة منسدلة ( أ ب ج بدون)
٨ـ الحصة قائمة منسدلة ( الأولى الثانية الثالثة الرابعة الخامسة السادسة السابعة)
٩ـ المادة قائمة منسدلة القرآن الإسلامية  اللغة العربية اللغة الانجليزية الرياضيات العلوم الاجتماعيات أخرى
١٠ـ عدد حصص المادة بالأسبوع القرآن ادخال يدوي الإسلامية ادخال يدوي اللغة العربية ادخال يدوي اللغة الانجليزية ادخال يدوي الرياضيات ادخال يدوي العلوم ادخال يدوي الاجتماعيات ادخال يدوي أخرى ادخال يدوي
١١ـ المادة المحددة عليه قائمة منسدلة القرآن الإسلامية  اللغة العربية اللغة الانجليزية الرياضيات العلوم الاجتماعيات أخرى. 
١٢ـ عدد حصصه اليومية والأسبوعية هذا يكون يكتب تلقائي
١٣ـ ملاحظات ادخال يدوي قائمة منسدلة نص قائمة
ملاحظة يكون الموقع يعد الجدول تلقائي دون تضارب
ملاحظة اضافة لجدول الحصص أن يقوم الموقع بحساب الحصص و توزيعها حسب أيام الاسبوع

ويكون شكل اخراج جدول صف مثل هذا
الأيام	الحصة 
( ١ )	الحصة 
( ٢ )	الحصة 
( ٣ )	الحصة 
( ٤ )	الحصة 
( ٥ )	الحصة 
( ٦ )	الحصة 
( ٧ )
السبت	لغة	لغة	إسلامية	قرآن	رياضيات	فنية	بدنية
الأحد	لغة	لغة	قرآن	رياضيات	علوم	فنية	بدنية
الإثنين	لغة	إسلامية	قرآن	رياضيات	لغة	فنية	بدنية
الثلاثاء	لغة	لغة	رياضيات	إسلامية	لغة	بدنية	بدنية
الأربعاء	علوم	قرآن	قرآن	لغة	بدنية	بدنية	بدنية
ويكون شكل اخراج جدول موظف مثل هذا
الأيام	الحصة 
( ١ )	الحصة 
( ٢ )	الحصة 
( ٣ )	الحصة 
( ٤ )	الحصة 
( ٥ )	الحصة 
( ٦ )	الحصة 
( ٧ )
السبت	الصف رابع
المادة لغة						
الأحد	الصف رابع
المادة لغة						
الإثنين	الصف رابع
المادة لغة						
الثلاثاء	الصف رابع
المادة لغة						
الأربعاء	الصف رابع
المادة لغة						

👈عمل جنب هذه الفقرة مايلي:-
تصدير لعام ـ تصدير لفصل ـ استرداد من عام استرداد من فصل ـ استرداد الفقرة كاملة ـ استرداد فقرة مختارة ـ استرداد جميع الفقرات

👈 مقترح لاصلاح جدول صف 
الصف
الشعبة
نظام الصف حصص بدون
أول أيام الأسبوع
آخر أيام الأسبوع
عدد أيام الأسبوع
أسماء المواد الدراسية للصف
عدد المواد الدراسية للصف
عدد حصص كل مادة
عدد مدرسين الصف
أسماء مدرسين الصف
توزيع مواد الصف على المدرسين
عدد مواد كل مدرس
عدد حصص كل مدرس بالأسبوع
اسم مربي / رائد الصف
اجمالي حصص الصف الأسبوع
اجمالي حصص الصف باليوم
من يتصدر للحصة 
رقم الحصة ١ و٢ و٣ و٤ 
اليوم
المادة 
وبعدها نضغط للنظام لصنع الجدول



ملاحظات عامة حول الجدول

ـ ألا يكون للمعلم/المعلمة في الفصل الواحد اكثر من ثنتين حصص متصلة 
ـ لا تكرر حصص مادة في يوم أكثر من حصتين الا إذا كانت حصص المادة أكثر من أيام الأسبوع .
ـ توازن جدول المعلم/المعلمة الواحد خلال ايام الأسبوع بحيث لا يظهر مزدحمًا في يوم وقليل الحصص في يوم آخر.
ـ مراعاة عدم التعارض أو التضارب بين الحصص الدراسية
ـ أن تكون الحصة الأولى لرائد الصف بصفه
ـ أن تكون حصة الفراغ للمعلم/المعلمة بعد أداءه لحصتين ما أمكن
ـ أن تكون عدد حصص جميع الصفوف متحدة
ـ أن يوزع الجدول بحسب عدد حصص كل صف
جدولة ذكية:
• محرك تحقق متقدم يكشف التعارضات تلقائياً
• منع تعارض المعلمين (عدم وجود معلم في مكانين في نفس الوقت)
• منع تعارض الفصول 
• قواعد خاصة للمعلمين
• قواعد مخصصة للفصول والمواد

توزيع تلقائي:
• نظام توزيع تلقائي ذكي يحترم جميع القواعد والقيود
• توزيع متوازن لأحمال المعلمين
• تجنب التعارضات تلقائياً
• توزيع متساوي عبر أيام الأسبوع

إدارة المعلمين المتقدمة
• إدارة المعلمين مع إمكانية إضافة بفلاتر ذكية
• قوالب جاهزة للإجازات والاستراحات
• وجود قوالب خاصة (إجازة وضع، إجازة أمومة، إجازة دراسية، تخفيف طبي و المزيد)

عروض متعددة
• عرض الجدول حسب المادة والفصل والمعلم
• عرض حصص محددة مع جميع التفاصيل
• عرض شامل لجميع المواد في مكان واحد
• جداول الفصول الكاملة

تصدير احترافي
• تصدير الجداول بصيغة PDF عالية الجودة
• خيارات تصدير مرنة (مادة واحدة، جميع المواد، فصل محدد)
انشاء جدول حصص اسبوعي مرتب.
تعديل اسماء المواد او الصفوف بسهولة.
حفظ التغييرات تلقائيا حتى بعد اغلاق الموقع.
واجهة بسيطة وسهلة الاستخدام.
يدعم اللغة العربية بشكل كامل.
يقوم بحساب الحصص و توزيعها حسب أيام الاسبوع
اضافة عدد لا محدود من الجداول 
 جدول استعمال زمن منظم و سهل 
 اضافة الالوان الي الجداول 
طباعة الجداول و تحويلها ل pdf  
حفظ الجدول على شكل صورة 
مشاركة الجدول مع المعلمين 
بسيط و سهل الاستخدام

عمل جنب هذه الفقرة بيانات الاضافة:-
ـ تاريخ الإضافة قائمة منسدلة  ( هجري ميلادي) الساعة 
ـ اليوم قائمة منسدلة ( السبت الأحد الإثنين الثلاثاء الأربعاء الخميس الجمعة )
ـ اسم الموظف الذي اضاف : 

عمل جنب هذه الفقرة هذه الخيارات:-
ـ زر ( معاينة ) لعرض العمل والنظر فيه قبل تصديره الخيارات التي يتم تثبيتها ورقة × ورقة ـ أو ورقتين  × ورقة ـ أو ثلاث ورق  × ورقة ـ أو أربع ورق  × ورقة
ـ تصدير التقرير يكون ورقة × ورقة. ـ أو ورقتين  × ورقة ـ  أو ثلاث ورق  × ورقة ـ أو أربع ورق  × ورقة
ـ تصدير عمود أعمدة مختارة جميع الأعمدة قائمة منسدلة  (  Excel PDF )
ـ تصديرها لملف قائمة منسدلة  (  Excel PDF )
ـ تصدير للفصل الثاني ـ تصدير للعام القادم 
ـ استيراد من الفصل الأول استيراد من العام الماضي
ـ مشاركة قائمة منسدلة  (  Excel PDF صورة )
ـ الاحتفاظ بصور الاشعارات للرجوع لها إذا لزم

عمل جنب هذه الفقرة هذه الازرار
🔲 معاينة 🔲 تحرير 🔲 تعديل 🔲 إضافة 🔲 تصدير 🔲 حذف 🔲 فتحPDF 🔲 فتحExcel 🔲 طباعة 🔲 مشاركة
➖➖➖➖➖➖➖➖➖➖➖ فاصل بين الفقرات الفرعية ➖➖➖➖➖➖➖➖➖➖➖➖➖
٣  ـ ٣ ـ حضور وغياب الموظفين
من يكتب هذه الفقرة	المدير أو من يكلفه من الموظفين في حينه وموافقة المدير على الادخال أو التعديل
من يشاهد هذه الفقرة	الموظف يشاهد علاماته فقط
من يشاهد تقارير هذه الفقرة	المدير
من يصدر تقارير هذه الفقرة	المدير
خيارات البحث بهذه الفقرة يكون بواسطة 	فقرة كلمة جملة

نوع التقارير المطلوب من الموقع تصديرها من هذه الفقرة
تصدير تقرير بالحضور اليومي لموظف لموظفين مختارين لجميع الموظفين
تصدير تقرير بالحضور الأسبوعي لموظف لموظفين مختارين لجميع الموظفين
تصدير تقرير بالحضور الشهري لموظف لموظفين مختارين لجميع الموظفين
تصدير تقرير بالحضور الفصلي لموظف لموظفين مختارين لجميع الموظفين
تصدير تقرير بالحضور للعام لموظف لموظفين مختارين لجميع الموظفين

تصدير تقرير بالغياب اليومي لموظف لموظفين مختارين لجميع الموظفين
تصدير تقرير بالغياب الأسبوعي لموظف لموظفين مختارين لجميع الموظفين
تصدير تقرير بالغياب الشهري لموظف لموظفين مختارين لجميع الموظفين
تصدير تقرير بالغياب الفصلي لموظف لموظفين مختارين لجميع الموظفين
تصدير تقرير بالغياب للعام لموظف لموظفين مختارين لجميع الموظفين

تصدير تقرير بالتأخر اليومي لموظف لموظفين مختارين لجميع الموظفين
تصدير تقرير بالتأخر الأسبوعي لموظف لموظفين مختارين لجميع الموظفين
تصدير تقرير بالتأخر الشهري لموظف لموظفين مختارين لجميع الموظفين
تصدير تقرير بالتأخر الفصلي لموظف لموظفين مختارين لجميع الموظفين
تصدير تقرير بالتأخر للعام لموظف لموظفين مختارين لجميع الموظفين

تصدير تقرير غياب حصص اليومي لموظف لموظفين مختارين لجميع الموظفين
تصدير تقرير غياب حصص الأسبوعي لموظف لموظفين مختارين لجميع الموظفين
تصدير تقرير غياب حصص الشهري لموظف لموظفين مختارين لجميع الموظفين
تصدير تقرير غياب حصص الفصلي لموظف لموظفين مختارين لجميع الموظفين
تصدير تقرير غياب حصص للعام لموظف لموظفين مختارين لجميع الموظفين



١ـ مسلسل رقم متسلسل تلقائي
٢ـ اسم الموظف
٣ـ الجنس قائمة منسدلة  ( ذكر أنثى)
٤ـ الفصل الدراسي قائمة منسدلة  ( الأول الثاني)
٥ـ الشهر قائمة منسدلة  ( محرم صفر ربيع أول ربيع ثاني جماد أول جماد ثاني رجب شعبان)
٦ـ الأسبوع ( الأول  الثاني الثالث الرابع)  
٧ـ اليوم قائمة منسدلة ( السبت الأحد الإثنين الثلاثاء الأربعاء الخميس )
٨ـ التاريخ يكتب تلقائي  ( هجري ميلادي)
٩ـ الصف ( أول أساسي ثاني أساسي ثالث أساسي رابع أساسي خامس أساسي سادس أساسي سابع أساسي ثامن أساسي تاسع أساسي ـ أول ثانوي ـ ثاني ثانوي ـ ثالث ثانوي)
١٠ـ الشعبة قائمة منسدلة ( أ ب ج بدون)

١١ـ العلامات 
متأخر	نعم	بعذر ـ بدون عذر	
	لا		
حاضر	نعم		
	لا		
غائب يوم كامل	نعم	بإذن ـ بدون إذن	خيارات بدون إذن (بعذر ـ بدون عذر
	لا		
غائب حصة رقم ١ أو ٢و٣ حصص	نعم	بإذن ـ بدون إذن	خيارات بدون إذن (بعذر ـ بدون عذر
	لا		
أخرى			
١٢ـ ملاحظات ادخال يدوي قائمة منسدلة نص قائمة

ملاحظة تحكم بكتابة وقت التأخر. للمتاخرين مدة التأخر : مثلا ساعة دقيقة 6 : 1

عمل جنب هذه الفقرة بيانات الاضافة:-
ـ تاريخ الإضافة قائمة منسدلة  ( هجري ميلادي) الساعة 
ـ اليوم قائمة منسدلة ( السبت الأحد الإثنين الثلاثاء الأربعاء الخميس الجمعة )
ـ اسم الموظف الذي اضاف : 

عمل جنب هذه الفقرة هذه الخيارات:-
ـ زر ( معاينة ) لعرض العمل والنظر فيه قبل تصديره الخيارات التي يتم تثبيتها ورقة × ورقة ـ أو ورقتين  × ورقة ـ أو ثلاث ورق  × ورقة ـ أو أربع ورق  × ورقة
ـ تصدير التقرير يكون ورقة × ورقة. ـ أو ورقتين  × ورقة ـ  أو ثلاث ورق  × ورقة ـ أو أربع ورق  × ورقة
ـ تصدير عمود أعمدة مختارة جميع الأعمدة قائمة منسدلة  (  Excel PDF )
ـ تصديرها لملف قائمة منسدلة  (  Excel PDF )
ـ تصدير للفصل الثاني ـ تصدير للعام القادم 
ـ استيراد من الفصل الأول استيراد من العام الماضي
ـ مشاركة قائمة منسدلة  (  Excel PDF صورة )
ـ الاحتفاظ بصور الاشعارات للرجوع لها إذا لزم

عمل جنب هذه الفقرة هذه الازرار
🔲 معاينة 🔲 تحرير 🔲 تعديل 🔲 إضافة 🔲 تصدير 🔲 حذف 🔲 فتحPDF 🔲 فتحExcel 🔲 طباعة 🔲 مشاركة
➖➖➖➖➖➖➖➖➖➖➖ فاصل بين الفقرات الفرعية ➖➖➖➖➖➖➖➖➖➖➖➖➖
٣  ـ ٤ ـ تحضيرالموظفين للدروس
من يكتب هذه الفقرة	الموظف ومسؤل الصف  
وفي حالة قيام الموظف بالتعديل لا بد من الموافقة من المدير في حينه
من يشاهد هذه الفقرة	الموظف لدروسه فقط وفي حالة قيام الموظف بالتعديل لا بد من الموافقة من المدير في حينه
من يشاهد تقارير هذه الفقرة	المدير
من يصدر تقارير هذه الفقرة	المدير
خيارات البحث بهذه الفقرة يكون بواسطة 	موظف مادة فقرة كلمة جملة

نوع التقارير المطلوب من الموقع تصديرها من هذه الفقرة
تصدير تقرير تحضير موظف ليوم لأسبوع لشهر لفصل لعام
تصدير تقرير تحضير لموظفين مختارين ليوم لأسبوع لشهر لفصل لعام
تصدير تقرير تحضير لجميع الموظفين ليوم لأسبوع لشهر لفصل لعام 
تصدير تقرير تحضير مادة ليوم لأسبوع لشهر لفصل لعام
تصدير تقرير تحضير مواد مختارة ليوم لأسبوع لشهر لفصل لعام
تصدير تقرير تحضير لجميع المواد ليوم لأسبوع لشهر لفصل لعام


١ـ مسلسل رقم متسلسل تلقائي
٢ـ اسم الموظف
٣ـ الجنس قائمة منسدلة  ( ذكر أنثى)
٤ـ الفصل الدراسي قائمة منسدلة  ( الأول الثاني)
٥ـ الشهر قائمة منسدلة  ( محرم صفر ربيع أول ربيع ثاني جماد أول جماد ثاني رجب شعبان)
٦ـ الأسبوع ( الأول  الثاني الثالث الرابع)  
٧ـ اليوم قائمة منسدلة ( السبت الأحد الإثنين الثلاثاء الأربعاء الخميس )
٨ـ التاريخ يكتب تلقائي  ( هجري ميلادي)
٩ـ الصف ( أول أساسي ثاني أساسي ثالث أساسي رابع أساسي خامس أساسي سادس أساسي سابع أساسي ثامن أساسي تاسع أساسي ـ أول ثانوي ـ ثاني ثانوي ـ ثالث ثانوي)
١٠ـ الشعبة قائمة منسدلة ( أ ب ج بدون)
١١ـ رقم الحصة قائمة منسدلة ( الأولى الثانية الثالثة الرابعة الخامسة السادسة السابعة)
١٢ـ المادة قائمة منسدلة ( القرآن الإسلامية  اللغة العربية اللغة الانجليزية الرياضيات العلوم الاجتماعيات أخرى )
١٣ـ عنوان الدرس ادخال يدوي قائمة منسدلة نص قائمة
١٤ـ رقم الصفحة مثلا ٥٠ أو بدون
١٥ـ خطوات التحضير ادخال يدوي قائمة منسدلة نص قائمة
١٦ـ صورة التحضير ادخال من ملف
١٧ـ صورة ورقة الخطة الأسبوعية ادخال من ملف
١٨ـ وسيلة الدرس قائمة منسدلة موجودة بدون ـ اخرى 
١٩ـ نفذ الدرس قائمة منسدلة نعم لا
٢٠ـ ملاحظات ادخال يدوي قائمة منسدلة نص قائمة

عمل جنب هذه الفقرة بيانات الاضافة:-
ـ تاريخ الإضافة قائمة منسدلة  ( هجري ميلادي) الساعة 
ـ اليوم قائمة منسدلة ( السبت الأحد الإثنين الثلاثاء الأربعاء الخميس الجمعة )
ـ اسم الموظف الذي اضاف : 

عمل جنب هذه الفقرة هذه الخيارات:-
ـ زر ( معاينة ) لعرض العمل والنظر فيه قبل تصديره الخيارات التي يتم تثبيتها ورقة × ورقة ـ أو ورقتين  × ورقة ـ أو ثلاث ورق  × ورقة ـ أو أربع ورق  × ورقة
ـ تصدير التقرير يكون ورقة × ورقة. ـ أو ورقتين  × ورقة ـ  أو ثلاث ورق  × ورقة ـ أو أربع ورق  × ورقة
ـ تصدير عمود أعمدة مختارة جميع الأعمدة قائمة منسدلة  (  Excel PDF )
ـ تصديرها لملف قائمة منسدلة  (  Excel PDF )
ـ تصدير للفصل الثاني ـ تصدير للعام القادم 
ـ استيراد من الفصل الأول استيراد من العام الماضي
ـ مشاركة قائمة منسدلة  (  Excel PDF صورة )
ـ الاحتفاظ بصور الاشعارات للرجوع لها إذا لزم

عمل جنب هذه الفقرة هذه الازرار
🔲 معاينة 🔲 تحرير 🔲 تعديل 🔲 إضافة 🔲 تصدير 🔲 حذف 🔲 فتحPDF 🔲 فتحExcel 🔲 طباعة 🔲 مشاركة
➖➖➖➖➖➖➖➖➖➖➖ فاصل بين الفقرات الفرعية ➖➖➖➖➖➖➖➖➖➖➖➖➖
٣  ـ ٥ ـ الواجبات المنزلية
من يكتب هذه الفقرة	الموظف وموافقة المدير على الادخال أو التعديل
من يشاهد هذه الفقرة	الموظف لوجباته مشاهدة فقط واذا اراد أن يعدل يطلب من المدير الموافقة او غير موافق
من يشاهد تقارير هذه الفقرة	المدير
من يصدر تقارير هذه الفقرة	المدير
خيارات البحث بهذه الفقرة يكون بواسطة 	مادة موظف فقرة كلمة جملة

نوع التقارير المطلوب من الموقع تصديرها من هذه الفقرة
تصدير تقرير بالواجبات اليومية لموظف لموظفين مختارين لجميع الموظفين
تصدير تقرير بالواجبات الأسبوعية لموظف لموظفين مختارين لجميع الموظفين
تصدير تقرير بالواجبات الشهرية لموظف لموظفين مختارين لجميع الموظفين
تصدير تقرير بالواجبات الفصلية لموظف لموظفين مختارين لجميع الموظفين
تصدير تقرير بالواجبات للعام لموظف لموظفين مختارين لجميع الموظفين

تصدير تقرير بالواجبات اليومية لصف لصفوف مختارة لجميع الصفوف
تصدير تقرير بالواجبات الأسبوعية لصف لصفوف مختارة لجميع الصفوف
تصدير تقرير بالواجبات الشهرية لصف لصفوف مختارة لجميع الصفوف
تصدير تقرير بالواجبات الفصلية لصف لصفوف مختارة لجميع الصفوف
تصدير تقرير بالواجبات للعام لصف لصفوف مختارة لجميع الصفوف

تصدير تقرير بالواجبات اليومية لمادة لمواد مختارة لجميع المواد
تصدير تقرير بالواجبات الأسبوعية لمادة لمواد مختارة لجميع المواد
تصدير تقرير بالواجبات الشهرية لمادة لمواد مختارة لجميع المواد
تصدير تقرير بالواجبات الفصلية لمادة لمواد مختارة لجميع المواد
تصدير تقرير بالواجبات للعام لمادة لمواد مختارة لجميع المواد



١ـ مسلسل رقم متسلسل تلقائي
٢ـ اسم الموظف
٣ـ الجنس قائمة منسدلة  ( ذكر أنثى)
٤ـ الفصل الدراسي قائمة منسدلة  ( الأول الثاني)
٥ـ الشهر قائمة منسدلة  ( محرم صفر ربيع أول ربيع ثاني جماد أول جماد ثاني رجب شعبان)
٦ـ الأسبوع ( الأول  الثاني الثالث الرابع)  
٧ـ اليوم قائمة منسدلة ( السبت الأحد الإثنين الثلاثاء الأربعاء الخميس )
٨ـ التاريخ يكتب تلقائي  ( هجري ميلادي)
٩ـ الصف ( أول أساسي ثاني أساسي ثالث أساسي رابع أساسي خامس أساسي سادس أساسي سابع أساسي ثامن أساسي تاسع أساسي ـ أول ثانوي ـ ثاني ثانوي ـ ثالث ثانوي)
١٠ـ الشعبة قائمة منسدلة ( أ ب ج بدون)
١١ـ الحصة قائمة منسدلة ( الأولى الثانية الثالثة الرابعة الخامسة السادسة السابعة)
١٢ـ المادة قائمة منسدلة ( القرآن الإسلامية  اللغة العربية اللغة الانجليزية الرياضيات العلوم الاجتماعيات أخرى )
١٣ـ نوع الواجب قائمة منسدلة ( عام ـ سلوك ـ قراءة ـ كتابة ـ حفظ ـ حساب ـ رسم ـ نشاط ـ اخرى )
١٤ـ الواجب قائمة منسدلة نص قائمة
١٥ـ رقم الصفحة مثلا ٤٥ او ٥١ و ٥٢ و٥٣ أو بدون
١٦ـ المدة المقترحة لكتابة الواجب قائمة منسدلة ( ربع ساعة نصف ساعة ساعة ساعة ونصف ساعتين أخرى )
١٧ـ نفذ قائمة منسدلة ( نعم لا )
١٨ـ توبع قائمة منسدلة ( نعم لا )
١٩ـ صحح قائمة منسدلة ( نعم لا )
٢٠ـ ملاحظات ادخال يدوي قائمة منسدلة نص قائمة

عمل جنب هذه الفقرة بيانات الاضافة:-
ـ تاريخ الإضافة قائمة منسدلة  ( هجري ميلادي) الساعة 
ـ اليوم قائمة منسدلة ( السبت الأحد الإثنين الثلاثاء الأربعاء الخميس الجمعة )
ـ اسم الموظف الذي اضاف : 

عمل جنب هذه الفقرة هذه الخيارات:-
ـ زر ( معاينة ) لعرض العمل والنظر فيه قبل تصديره الخيارات التي يتم تثبيتها ورقة × ورقة ـ أو ورقتين  × ورقة ـ أو ثلاث ورق  × ورقة ـ أو أربع ورق  × ورقة
ـ تصدير التقرير يكون ورقة × ورقة. ـ أو ورقتين  × ورقة ـ  أو ثلاث ورق  × ورقة ـ أو أربع ورق  × ورقة
ـ تصدير عمود أعمدة مختارة جميع الأعمدة قائمة منسدلة  (  Excel PDF )
ـ تصديرها لملف قائمة منسدلة  (  Excel PDF )
ـ تصدير للفصل الثاني ـ تصدير للعام القادم 
ـ استيراد من الفصل الأول استيراد من العام الماضي
ـ مشاركة قائمة منسدلة  (  Excel PDF صورة )
ـ الاحتفاظ بصور الاشعارات للرجوع لها إذا لزم

عمل جنب هذه الفقرة هذه الازرار
🔲 معاينة 🔲 تحرير 🔲 تعديل 🔲 إضافة 🔲 تصدير 🔲 حذف 🔲 فتحPDF 🔲 فتحExcel 🔲 طباعة 🔲 مشاركة
➖➖➖➖➖➖➖➖➖➖➖ فاصل بين الفقرات الفرعية ➖➖➖➖➖➖➖➖➖➖➖➖➖
٣  ـ ٦ ـ تصحيح الدفاتر
من يكتب هذه الفقرة	الموظف وموافقة المدير على الادخال أو التعديل
ولا يعدلها الموظف الا بموافقة المدير في حينه
من يشاهد هذه الفقرة	الموظف يشاهد عمله مشاهدة فقط
من يشاهد تقارير هذه الفقرة	المدير
من يصدر تقارير هذه الفقرة	المدير
خيارات البحث بهذه الفقرة يكون بواسطة 	موظف مادة فقرة كلمة جملة


نوع التقارير المطلوب من الموقع تصديرها من هذه الفقرة
تصدير تقرير تصحيح الدفاتر اليومية لموظف لموظفين مختارين لجميع الموظفين
تصدير تقرير تصحيح الدفاتر الأسبوعية لموظف لموظفين مختارين لجميع الموظفين
تصدير تقرير تصحيح الدفاتر الشهرية لموظف لموظفين مختارين لجميع الموظفين
تصدير تقرير تصحيح الدفاتر الفصلية لموظف لموظفين مختارين لجميع الموظفين
تصدير تقرير تصحيح الدفاتر للعام لموظف لموظفين مختارين لجميع الموظفين

تصدير تقرير تصحيح الدفاتر اليومية لصف لصفوف مختارة لجميع الصفوف
تصدير تقرير تصحيح الدفاتر الأسبوعية لصف لصفوف مختارة لجميع الصفوف
تصدير تقرير تصحيح الدفاتر الشهرية لصف لصفوف مختارة لجميع الصفوف
تصدير تقرير تصحيح الدفاتر الفصلية لصف لصفوف مختارة لجميع الصفوف
تصدير تقرير تصحيح الدفاتر للعام لصف لصفوف مختارة لجميع الصفوف


تصدير تقرير تصحيح الدفاتر اليومية لمادة لمواد مختارة لجميع المواد
تصدير تقرير تصحيح الدفاتر الأسبوعية لمادة لمواد مختارة لجميع المواد
تصدير تقرير تصحيح الدفاتر الشهرية لمادة لمواد مختارة لجميع المواد
تصدير تقرير تصحيح الدفاتر الفصلية لمادة لمواد مختارة لجميع المواد
تصدير تقرير تصحيح الدفاتر للعام لمادة لمواد مختارة لجميع المواد

تصدير تقرير تصحيح الدفاتر اليومية لطالب لطلاب مختارين لجميع الطلاب
تصدير تقرير تصحيح الدفاتر الأسبوعية لطالب لطلاب مختارين لجميع الطلاب
تصدير تقرير تصحيح الدفاتر الشهرية لطالب لطلاب مختارين لجميع الطلاب
تصدير تقرير تصحيح الدفاتر الفصلية لطالب لطلاب مختارين لجميع الطلاب
تصدير تقرير تصحيح الدفاتر للعام لطالب لطلاب مختارين لجميع الطلاب

١ـ مسلسل رقم متسلسل تلقائي
٢ـ اسم الموظف 
٣ـ الجنس قائمة منسدلة  ( ذكر أنثى)
٤ـ الفصل الدراسي قائمة منسدلة  ( الأول الثاني)
٥ـ الشهر قائمة منسدلة  ( محرم صفر ربيع أول ربيع ثاني جماد أول جماد ثاني رجب شعبان)
٦ـ الأسبوع ( الأول  الثاني الثالث الرابع)  
٧ـ اليوم قائمة منسدلة ( السبت الأحد الإثنين الثلاثاء الأربعاء الخميس )
٨ـ التاريخ يكتب تلقائي  ( هجري ميلادي)
٩ـ الصف ( أول أساسي ثاني أساسي ثالث أساسي رابع أساسي خامس أساسي سادس أساسي سابع أساسي ثامن أساسي تاسع أساسي ـ أول ثانوي ـ ثاني ثانوي ـ ثالث ثانوي)
١٠ـ الشعبة قائمة منسدلة ( أ ب ج بدون)
١١ـ دفتر المادة المصحح  قائمة منسدلة (القرآن الإسلامية  اللغة العربية اللغة الانجليزية الرياضيات العلوم الاجتماعيات أخرى )
١٢ـ ملاحظات ادخال يدوي قائمة منسدلة نص قائمة

عمل جنب هذه الفقرة بيانات الاضافة:-
ـ تاريخ الإضافة قائمة منسدلة  ( هجري ميلادي) الساعة 
ـ اليوم قائمة منسدلة ( السبت الأحد الإثنين الثلاثاء الأربعاء الخميس الجمعة )
ـ اسم الموظف الذي اضاف : 

عمل جنب هذه الفقرة هذه الخيارات:-
ـ زر ( معاينة ) لعرض العمل والنظر فيه قبل تصديره الخيارات التي يتم تثبيتها ورقة × ورقة ـ أو ورقتين  × ورقة ـ أو ثلاث ورق  × ورقة ـ أو أربع ورق  × ورقة
ـ تصدير التقرير يكون ورقة × ورقة. ـ أو ورقتين  × ورقة ـ  أو ثلاث ورق  × ورقة ـ أو أربع ورق  × ورقة
ـ تصدير عمود أعمدة مختارة جميع الأعمدة قائمة منسدلة  (  Excel PDF )
ـ تصديرها لملف قائمة منسدلة  (  Excel PDF )
ـ تصدير للفصل الثاني ـ تصدير للعام القادم 
ـ استيراد من الفصل الأول استيراد من العام الماضي
ـ مشاركة قائمة منسدلة  (  Excel PDF صورة )
ـ الاحتفاظ بصور الاشعارات للرجوع لها إذا لزم

عمل جنب هذه الفقرة هذه الازرار
🔲 معاينة 🔲 تحرير 🔲 تعديل 🔲 إضافة 🔲 تصدير 🔲 حذف 🔲 فتحPDF 🔲 فتحExcel 🔲 طباعة 🔲 مشاركة
➖➖➖➖➖➖➖➖➖➖➖ فاصل بين الفقرات الفرعية ➖➖➖➖➖➖➖➖➖➖➖➖➖

٣  ـ ٧ ـ مخالفات الموظفين
من يكتب هذه الفقرة	المدير أو من يكلفه من الموظفين في حينه وموافقة المدير على الادخال أو التعديل
من يشاهد هذه الفقرة	الموظف يشاهد مخالفاته مشاهدة فقط دون ادخال أو تعديل
من يشاهد تقارير هذه الفقرة	المدير
من يصدر تقارير هذه الفقرة	المدير
خيارات البحث بهذه الفقرة يكون بواسطة 	موظف فقرة كلمة جملة

نوع التقارير المطلوب من الموقع تصديرها من هذه الفقرة
تصدير تقرير بالمخالفات اليومية لموظف ـ لموظفين مختارين ـ لجميع الموظفين 
تصدير تقرير بالمخالفات الأسبوعية لموظف ـ لموظفين مختارين ـ لجميع الموظفين
تصدير تقرير بالمخالفات الشهرية لموظف ـ لموظفين مختارين ـ لجميع الموظفين
تصدير تقرير بالمخالفات الفصلية لموظف ـ لموظفين مختارين ـ لجميع الموظفين
تصدير تقرير بالمخالفات للعام لموظف ـ لموظفين مختارين ـ لجميع الموظفين

١ـ مسلسل رقم متسلسل تلقائي
٢ـ اسم الموظف تلقائي
٣ـ الجنس قائمة منسدلة  ( ذكر أنثى)
٤ـ الفصل الدراسي قائمة منسدلة  ( الأول الثاني)
٥ـ الشهر قائمة منسدلة  ( محرم صفر ربيع أول ربيع ثاني جماد أول جماد ثاني رجب شعبان)
٦ـ الأسبوع ( الأول  الثاني الثالث الرابع)  
٧ـ اليوم قائمة منسدلة ( السبت الأحد الإثنين الثلاثاء الأربعاء الخميس )
٨ـ التاريخ يكتب تلقائي  ( هجري ميلادي)  
٩ـ جانب المخالفة قائمة منسدلة نص قائمة
١٠ـ نص المخالفة قائمة منسدلة نص قائمة
١١ـ رقم المخالفة يكتب تلقائي في حالة كونها اول مخالفة يكون الرقم ١ وفي حالة كونها الثانية يكتب الرقم ٢
١٢ـ ملاحظات ادخال يدوي قائمة منسدلة نص قائمة

عمل جنب هذه الفقرة بيانات الاضافة:-
ـ تاريخ الإضافة قائمة منسدلة  ( هجري ميلادي) الساعة 
ـ اليوم قائمة منسدلة ( السبت الأحد الإثنين الثلاثاء الأربعاء الخميس الجمعة )
ـ اسم الموظف الذي اضاف : 

عمل جنب هذه الفقرة هذه الخيارات:-
ـ زر ( معاينة ) لعرض العمل والنظر فيه قبل تصديره الخيارات التي يتم تثبيتها ورقة × ورقة ـ أو ورقتين  × ورقة ـ أو ثلاث ورق  × ورقة ـ أو أربع ورق  × ورقة
ـ تصدير التقرير يكون ورقة × ورقة. ـ أو ورقتين  × ورقة ـ  أو ثلاث ورق  × ورقة ـ أو أربع ورق  × ورقة
ـ تصدير عمود أعمدة مختارة جميع الأعمدة قائمة منسدلة  (  Excel PDF )
ـ تصديرها لملف قائمة منسدلة  (  Excel PDF )
ـ تصدير للفصل الثاني ـ تصدير للعام القادم 
ـ استيراد من الفصل الأول استيراد من العام الماضي
ـ مشاركة قائمة منسدلة  (  Excel PDF صورة )
ـ الاحتفاظ بصور الاشعارات للرجوع لها إذا لزم

عمل جنب هذه الفقرة هذه الازرار
🔲 معاينة 🔲 تحرير 🔲 تعديل 🔲 إضافة 🔲 تصدير 🔲 حذف 🔲 فتحPDF 🔲 فتحExcel 🔲 طباعة 🔲 مشاركة

🎄🎄🎄🎄🎄🎄🎄🎄🎄🎄🎄🎄🎄🎄 فاصل بين الفقرات الرئيسية 🎄🎄🎄🎄🎄🎄🎄🎄🎄🎄🎄🎄🎄

🔴 ٤ ـ الطلاب والطالبات
٤  ـ ١ ـ بيانات الطلاب والطالبات

من يكتب هذه الفقرة	مسؤل الصف أو من يكلفه المدير من الموظفين في حينه وموافقة المدير على الادخال أو التعديل
ولا يعدلها الموظف أو مسؤل الصف الا بموافقة المدير في حينه
من يشاهد هذه الفقرة	مسؤل الصف يشاهد صفه فقط
من يشاهد تقارير هذه الفقرة	المدير
من يصدر تقارير هذه الفقرة	المدير
خيارات البحث بهذه الفقرة يكون بواسطة 	طالب طالبة فقرة كلمة جملة

نوع التقارير المطلوب من الموقع تصديرها من هذه الفقرة
تصدير تقرير بيانات لطالب لطلاب مختارين لجميع الطلاب
تصدير تقرير بيانات صف لصفوف مختارة لجميع الصفوف
تصدير تقرير بيانات القرى والمحلات لصف لصفوف مختارة لجميع الصفوف
تصدير تقرير بيانات الحالة المرضية لصف لصفوف مختارة لجميع الصفوف
تصدير تقرير بيانات قديم ـ جديد ـ اسم المدرسة المنتقل منها
تصدير تقرير بيانات من يذاكر له بالبيت
تصدير تقرير بيانات الاخوة والأخوات
تصدير تقرير بيانات مكان جلوسه بالصف لصف صفوف مختارة جميع الصفوف
تصدير تقرير بيانات مكان وقوفه بالطابور لصف صفوف مختارة جميع الصفوف
تصدير تقرير بيانات الزي المدرسي لصف صفوف مختارة جميع الصفوف
تصدير تقرير بيانات الأيتام لصف صفوف مختارة جميع الصفوف
تصدير تقرير بيانات المواهب المتميز بها لصف صفوف مختارة جميع الصفوف
تصدير تقرير بيانات القادرين على القراءة لصف صفوف مختارة جميع الصفوف
تصدير تقرير بيانات القادرين على الكتابة لصف صفوف مختارة جميع الصفوف
تصدير تقرير بيانات الوالد لصف صفوف مختارة جميع الصفوف
تصدير تقرير بيانات الأم لصف صفوف مختارة جميع الصفوف
تصدير تقرير بيانات بمن سلموا رسوم الامتحان للفصل الأول لصف صفوف مختارة جميع الصفوف
تصدير تقرير بيانات بمن لم يسلموا رسوم الامتحان للفصل الأول لصف صفوف مختارة جميع الصفوف
تصدير تقرير بيانات بمن سلموا رسوم الامتحان للفصل الثاني لصف صفوف مختارة جميع الصفوف
تصدير تقرير بيانات بمن لم يسلموا رسوم الامتحان للفصل الثاني لصف صفوف مختارة جميع الصفوف
تصدير تقرير بيانات عدد طلاب وطالبات لصف صفوف مختارة جميع الصفوف

مسلسل رقم متسلسل تلقائي
الاسم
الجنس قائمة منسدلة  ( ذكر أنثى)
الصف قائمة منسدلة ( أول أساسي ثاني أساسي ثالث أساسي رابع أساسي خامس أساسي سادس أساسي سابع أساسي ثامن أساسي تاسع أساسي ـ أول ثانوي ـ ثاني ثانوي ـ ثالث ثانوي)
الشعبة قائمة منسدلة ( أ ب ج بدون)
تاريخ الميلاد. مثلا انا أكتب هنا ٢٠١٥ فيطلع بالعمر ١٠   العمر والعكس مثلا اكتب ١٠ يطلع تلقائي ٢٠١٥
القرية قائمة منسدلة  وادي النخلة ـ النخلة ـ بردان ـ المفرق ـ المسرب ـ بيت الشبر ـ العوارض ـ اخرى
المحلة قائمة منسدلة  أسماء القرى والمحلات التابعة لكل قرية
١ـ قرية وادي النخلة المحلات التابعة لها ( الطوائل ـ شعب الكبير  ـ وادي النخلة ـ شرى النقيل ـ الخراشب ـ الجحارة ـ الدمن ـ أخرى)
٢ـ النخلة المحلات التابعة لها ( الصلبة ـ مجهم العرم ـ المراديح ـ رأس القرية ـ القرية ـ حول أسامة ـ عرض فراص ـ  بيت قضابة ـ المرخامة ـ البرحة ـ المجهم ـ دي سويد ـ الراحبة ـ شرى الكدف ـ حول مطيع ـ شرى البارقة ـ رأس البارقة ـ الدرب ـ الهياج ـ بيت الحسام ـ المجاهم ـ ابعرات ـ الحرة ـ الصلاو ـ أخرى  )
٣ـ بردان المحلات التابعة لها ( المقبرة ـ شرى بردان ـ  بردان ـ الشعب ـ قرية العارضة ـ باب الهيجة ـ أخرى )
٤ـ المفرق المحلات التابعة لها ( القرف ـ دعاشب ـ وادي بردان ـ ضحمه ـ أخرى)
٥ـ المسرب 
٦ـ بيت الشبر
٧ـ العوارض 
٨ـ اخرى

العزلة قائمة منسدلة  ( منهات ـ بني عواض ـ أخرى )
المديرية قائمة منسدلة  ( العدين ـ أخرى )
المحافظة قائمة منسدلة  ( إب ـ أخرى )
ترتيبه بالأسرة عدد متسلسل من ١ إلى ٢٠
الحالة الصحية قائمة منسدلة  ( سليم )
الحالة المرضية قائمة منسدلة  (مريض ـ تفاصيل مرضه )
وهذا شكل التقرير اللي يكون يظهر لهذه الفقرة الحالات المرضية بالمدرسة
م	الاسم	الصف	الجنس	مريض	تفاصيل مرضه	ملاحظات
						
						

دراسته بالمدرسة قائمة منسدلة  ( قديم ـ جديد ـ اسم المدرسة المنتقل منها قائمة منسدلة  ( أبي عبيدة ـ أم سلمة ـ عثمان بن عفان ـ أخرى )
وهذا شكل التقرير اللي يكون يظهر لهذه الفقرة كشف بأسماء الطلاب والطالبات الجدد
م	الاسم	الجنس	الصف 	المدرسة المنتقل منها	ملاحظات 
					
					

من يذاكر له بالبيت  الاسم  صفة القرابة خيارات صفة القرابة التي يتم تثبيتها( أب أم أخ أخت عم عمة خال خالة جد جدة أخرى) المؤهل قائمة منسدلة  ( بدون خبرة أساسي ثانوي دبلوم معلمين دبلوم عالي جامعي)
وهذا شكل التقرير لهذه الفقرة
م	اسم الطالب / الطالبة	الصف	الجنس	اسم من يذاكر له	صفة القرابة	المؤهل
						


أسماء اخوانه/ أخواته بالمدرسة
وهذ شكل التقرير لهذه الفقرة
م	اسم الطالب/ الطالبة	الصف	أول	ثاني	ثالث	رابع	خامس	سادس	سابع	ثامن	تاسع	١/ ث	٢/ ث	٣/ ث
		أسماء اخوانه	١ـ
٢ـ 	١ـ
٢ـ	١ـ
٢ـ	١ـ
٢ـ	١ـ
٢ـ	١ـ
٢ـ	١ـ
٢ـ	١ـ
٢ـ	١ـ
٢ـ	١ـ
٢ـ	١ـ
٢ـ	١ـ
٢ـ
		أسماء أخواته	١ـ
٢ـ	١ـ
٢ـ	١ـ
٢ـ	١ـ
٢ـ	١ـ
٢ـ	١ـ
٢ـ	١ـ
٢ـ	١ـ
٢ـ	١ـ 
٢ـ	١ـ
٢ـ	١ـ
٢ـ	١ـ
٢ـ

النتيجة قائمة منسدلة  ( ناجح ناجحة راسب راسبة معيد معيدة )
وهذا شكل التقرير اللي يكون يظهر لهذه الفقرة كشف بأسماء الطلاب والطالبات المعيدين
م	الاسم	الجنس	الصف الناجح له	الصف المعاد إليه	ملاحظات 
					
					

مكان جلوسه بالصف 
اطار قائمة منسدلة ( أول ثاني ثالث رابع خامس) 
مقعد عدد متسلسل من ١ إلى ٣٠
مكانه بالمقعد قائمة منسدلة ( يمين وسط شمال)
وهذا شكل التقرير اللي يكون يظهر لهذه الفقرة لصف لصفوف مختارة لجميع الصفوف

مكان وقوفه بالطابور 
اطار قائمة منسدلة ( أول ثاني ثالث رابع خامس) 
رقمه عدد متسلسل تلقائي
اسم الذي قبله
اسم الذي بعده
 وهذا شكل التقرير لهذه الفقرة👇 لصف لصفوف مختارة لجميع الصفوف
الصف:(ــــــــــــــــــــــــــــــــــــــــ ) مربي/ رائد  الصف الأستاذ    (ـــــــــــــــــــــــــــــــــــــــــــــــــــــــــــــــــــــــــ )

طلاب		طالبات
م	الاسم		م	الاسم
				
				
				
				
				
				
				
				
				
				
				
				
				
				
				
				
				
				
				
				
				
				
				
				
				
				
				
				
				
				
				
				
				
				
				
				
				
				
				
				
				
				

ارتداء الزي المدرسي قائمة منسدلة  ( نعم وسط لا)
وهذا شكل التقرير اللي يكون يظهر لهذه الفقرة كشف باسماء الطلاب الذين لم يرتدوا الزي المدرسي
م	الاسم	الجنس	الصف	القرية
				
				

يتيم قائمة منسدلة ( لا ـ نعم) خيارات نعم قائمة منسدلة ( أب ـ أم ـ أب وأم)
وهذا شكل التقرير اللي يكون يظهر لهذه الفقرة كشف الأيتام
م	الاسم	الجنس	الصف	القرية	ت ميلاد	يتيم 	المسؤل عليه	ملاحظات
						أب	أم	أب وأم
										

المواهب المتميز بها قائمة منسدلة  ( التفوق الدراسي الذكاء القراءة الخط الرسم الإلقاء التمثيل الحفظ الإنشاد محاكات الأصوات ترتيل القرآن شاعر حرفة يدوية سائق حركات نادرة رياضة سباحة تطريز نقش خياطة صناعة)
اعمل للواحد الموهبة الأولى الموهبة الثانية الموهبة الثالثة
وهذا شكل التقرير اللي يكون يظهر لهذه الفقرة
م	الاسم	الجنس	الصف	القرية	المواهب المتميز بها	ملاحظات
					الأولى	الثانية	الثالثة	
								
وتقرير هذا يكون تصدير موهبة ـ  مواهب مختارة ـ جميع المواهب لصف لصفوف مختارة لجميع الصفوف

قدرته على القراءة قائمة منسدلة  ( يقرأ وسط  ضعيف)
وهذا شكل التقرير اللي يكون يظهر لهذه الفقرة
م	الاسم	الجنس	الصف	القرية	تقييم جانب القراءة	ملاحظات
					يقرأ	وسط	ضعيف
								
وتقرير هذا يكون تصدير القادرين الوسط الضعاف لصف لصفوف مختارة لجميع الصفوف

قدرته على الكتابة قائمة منسدلة  ( خطاط وسط  ضعيف)
وهذا شكل التقرير اللي يكون يظهر لهذه الفقرة
م	الاسم	الجنس	الصف	القرية	تقييم جانب الخط	ملاحظات
					خطاط	وسط	ضعيف
								
وتقرير هذا يكون تصدير الخطاطين الوسط الضعاف لصف لصفوف مختارة لجميع الصفوف

والده كما يلي 
الاسم يظهر تلقائي من اسم الطالب
مقيم قائمة منسدلة وادي النخلة ـ النخلة ـ بردان ـ المفرق ـ المسرب ـ بيت الشبر ـ العوارض ـ اخرى
مغترب قائمة منسدلة ( السعودية البحرين  عمان أخرى)
المهنة (المهنة الأولى المهنة الثانية المهنة الثالثة 
خيارات المهنة قائمة منسدلة ( معلم دكتور عامل سائق بناء تاجر مزارع بدون أخرى )
وهذا شكل التقرير اللي يكون يظهر لهذه الفقرة
م	اسم الطالب / الطالبة	اسم والده	مقيم	مغترب	المهنة ١	المهنة ٢	المهنة ٣	ملاحظات
								

ولي الأمر في حالة الاتصال به رقم يمني 
م	الاسم	صفة القرابة	رقم جواله	الإقامة	ملاحظات
					
					

الاقامة قائمة منسدلة  وادي النخلة ـ النخلة ـ بردان ـ المفرق ـ المسرب ـ بيت الشبر ـ العوارض ـ اخرى
ونفس الجدول السابق يكون شكل التقرير لهذه الفقرة لصف لصفوف مختارة لجميع الصفوف

ولي الأمر في حالة التواصل معه بالواتس 
م	الاسم	صفة القرابة	رقم جواله	الإقامة	ملاحظات
					
					

الاقامة قائمة منسدلة  وادي النخلة ـ النخلة ـ بردان ـ المفرق ـ المسرب ـ بيت الشبر ـ العوارض ـ اخرى
ونفس الجدول السابق يكون شكل التقرير لهذه الفقرة لصف لصفوف مختارة لجميع الصفوف

بيانات الأم
وهذ شكل التقرير لهذه الفقرة لصف لصفوف مختارة لجميع الصفوف
م	الاسم	رقم جوال للاتصال	رقم جوال للواتس	ملاحظات
				
				

عمل الطالب خارج الدوام قائمة منسدلة  ( مذاكرة رعي بيع لعب بدون أخرى)

عمل الطالبة خارج الدوام قائمة منسدلة  ( مذاكرة رعي عمل البيت لعب بدون أخرى)
يكون شكل التقرير لهذه الفقرة لصف لصفوف مختارة لجميع الصفوف
م	الاسم	الجنس	الصف	العمل خارج الدوام	ملاحظات
					

الحالة قائمة منسدلة مستمر منقول
الانتظام قائمة منسدلة (منتظم ـ منتسب ـ اخرى )
رسوم امتحان الفصل الأول
المبلغ المقرر رقم
المبلغ المستلم رقم
رسوم امتحان الفصل الثاني
المبلغ المقرر رقم
المبلغ المستلم رقم

يكون يخرج تقرير بمن سلموا رسوم الامتحان للفصل الأول
يكون يخرج تقرير بمن لم يسلموا رسوم الامتحان للفصل الأول
يكون يخرج تقرير بمن سلموا رسوم الامتحان للفصل الثاني
يكون يخرج تقرير بمن لم يسلموا رسوم الامتحان للفصل الثاني
ملاحظة ان يكون الموقع يخرج لنا 


تقرير واحصاء رقمي بعدد طلاب وطالبات المدرسة
وهذا شكله
الصف	ذكور	إناث	المجموع	ملاحظات
الأول الأساسي				
الثاني الأساسي				
الثالث الأساسي				
الرابع الأساسي				
الخامس الأساسي				
السادس الأساسي				
السابع الأساسي 				
الثامن الأساسي				
الإجمالي				

مجموع الذكور والإناث يكون يكتب تلقائي
الاجمالي اخر الورقة يكون يكتب تلقائي

ويكون التقرير لصف لصفوف مختارة ـ لجميع الصفوف ـ لفصل ـ لعام

👈عمل جنب هذه الفقرة مايلي:-
تصدير لعام ـ تصدير لفصل ـ استرداد من عام استرداد من فصل ـ استرداد الفقرة كاملة ـ استرداد فقرة مختارة ـ استرداد جميع الفقرات

عمل جنب هذه الفقرة بيانات الاضافة:-
ـ تاريخ الإضافة قائمة منسدلة  ( هجري ميلادي) الساعة 
ـ اليوم قائمة منسدلة ( السبت الأحد الإثنين الثلاثاء الأربعاء الخميس الجمعة )
ـ اسم الموظف الذي اضاف : 

عمل جنب هذه الفقرة هذه الخيارات:-
ـ زر ( معاينة ) لعرض العمل والنظر فيه قبل تصديره الخيارات التي يتم تثبيتها ورقة × ورقة ـ أو ورقتين  × ورقة ـ أو ثلاث ورق  × ورقة ـ أو أربع ورق  × ورقة
ـ تصدير التقرير يكون ورقة × ورقة. ـ أو ورقتين  × ورقة ـ  أو ثلاث ورق  × ورقة ـ أو أربع ورق  × ورقة
ـ تصدير عمود أعمدة مختارة جميع الأعمدة قائمة منسدلة  (  Excel PDF )
ـ تصديرها لملف قائمة منسدلة  (  Excel PDF )
ـ تصدير للفصل الثاني ـ تصدير للعام القادم 
ـ استيراد من الفصل الأول استيراد من العام الماضي
ـ مشاركة قائمة منسدلة  (  Excel PDF صورة )
ـ الاحتفاظ بصور الاشعارات للرجوع لها إذا لزم

عمل جنب هذه الفقرة هذه الازرار
🔲 معاينة 🔲 تحرير 🔲 تعديل 🔲 إضافة 🔲 تصدير 🔲 حذف 🔲 فتحPDF 🔲 فتحExcel 🔲 طباعة 🔲 مشاركة

ملاحظة هامة : ادخال بيانات الطلاب والطالبات يتم من قبل الموظفين المسؤولين عن الصفوف فقط أو من يكلفه المدير في حينه




ملاحظة :- المعلومات المطلوب من الموقع
اخراجها عن طالب أو طالبة
الاسم
الجنس
تاريخ الميلاد
العمر
القرية
المحلة
العزلة
المديرية
المحافظة
ترتيبه بالأسرة
الحالة الصحية
الحالة المرضية
دراسته بالمدرسة
من يذاكر له بالبيت
أسماء اخوانه/ أخواته بالمدرسة
النتيجة
مكان جلوسه بالصف
مكان وقوفه بالطابور
ارتداء الزي المدرسي
يتيم
المواهب المتميز بها
قدرته على القراءة
قدرته على الكتابة
والده
ولي الأمر في حالة الاتصال به رقم يمني
ولي الأمر في حالة التواصل معه بالواتس
بيانات الأم
عمل الطالب خارج الدوام
حضور وغياب
تقيد مشكلات الطلاب والطالبات
درجاته
كتبه
أدائه لفقرات طابور الصباح
تفوقه بالمسابقات المدرسية
➖➖➖➖➖➖➖➖➖➖➖ فاصل بين الفقرات الفرعية ➖➖➖➖➖➖➖➖➖➖➖➖➖
٤  ـ ٢ ـ حضور وغياب
من يكتب هذه الفقرة	مسؤل الصف أو من يكلفه المدير من الموظفين في حينه وموافقة المدير على الادخال أو التعديل
ولا يعدلها الموظف أو مسؤل الصف الا بموافقة المدير في حينه
من يشاهد هذه الفقرة	مسؤل الصف يشاهد صفه فقط
من يشاهد تقارير هذه الفقرة	المدير
من يصدر تقارير هذه الفقرة	المدير
خيارات البحث بهذه الفقرة يكون بواسطة 	طالب طالبة فقرة كلمة جملة

نوع التقارير المطلوب من الموقع تصديرها من هذه الفقرة
تصدير تقرير بالحضور اليومي لطالب لطلاب مختارين لجميع الطلاب
تصدير تقرير بالحضور الأسبوعي لطالب لطلاب مختارين لجميع الطلاب
تصدير تقرير بالحضور الشهري لطالب لطلاب مختارين لجميع الطلاب
تصدير تقرير بالحضور الفصلي لطالب لطلاب مختارين لجميع الطلاب
تصدير تقرير بالحضور للعام لطالب لطلاب مختارين لجميع الطلاب

تصدير تقرير بالغياب اليومي لطالب لطلاب مختارين لجميع الطلاب
تصدير تقرير بالغياب الأسبوعي لطالب لطلاب مختارين لجميع الطلاب
تصدير تقرير بالغياب الشهري لطالب لطلاب مختارين لجميع الطلاب
تصدير تقرير بالغياب الفصلي لطالب لطلاب مختارين لجميع الطلاب
تصدير تقرير بالغياب للعام لطالب لطلاب مختارين لجميع الطلاب

تصدير تقرير بالتأخر اليومي لطالب لطلاب مختارين لجميع الطلاب
تصدير تقرير بالتأخر الأسبوعي لطالب لطلاب مختارين لجميع الطلاب
تصدير تقرير بالتأخر الشهري لطالب لطلاب مختارين لجميع الطلاب
تصدير تقرير بالتأخر الفصلي لطالب لطلاب مختارين لجميع الطلاب
تصدير تقرير بالتأخر للعام لطالب لطلاب مختارين لجميع الطلاب

تصدير تقرير بالحضور اليومي لصف لصفوف مختارة لجميع الصفوف
تصدير تقرير بالحضور الأسبوعي لصف لصفوف مختارة لجميع الصفوف
تصدير تقرير بالحضور الشهري لصف لصفوف مختارة لجميع الصفوف
تصدير تقرير بالحضور الفصلي لصف لصفوف مختارة لجميع الصفوف
تصدير تقرير بالحضور للعام لصف لصفوف مختارة لجميع الصفوف

تصدير تقرير بالغياب اليومي لصف لصفوف مختارة لجميع الصفوف
تصدير تقرير بالغياب الأسبوعي لصف لصفوف مختارة لجميع الصفوف
تصدير تقرير بالغياب الشهري لصف لصفوف مختارة لجميع الصفوف
تصدير تقرير بالغياب الفصلي لصف لصفوف مختارة لجميع الصفوف
تصدير تقرير بالغياب للعام لصف لصفوف مختارة لجميع الصفوف

تصدير تقرير بالتأخر اليومي لصف لصفوف مختارة لجميع الصفوف
تصدير تقرير بالتأخر الأسبوعي لصف لصفوف مختارة لجميع الصفوف
تصدير تقرير بالتأخر الشهري لصف لصفوف مختارة لجميع الصفوف
تصدير تقرير بالتأخر الفصلي لصف لصفوف مختارة لجميع الصفوف
تصدير تقرير بالتأخر للعام لصف لصفوف مختارة لجميع الصفوف

تصدير تقرير بالمؤدبين اليومي لطالب لطلاب مختارين لجميع الطلاب
تصدير تقرير بالمؤدبين الأسبوعي لطالب لطلاب مختارين لجميع الطلاب
تصدير تقرير بالمؤدبين الشهري لطالب لطلاب مختارين لجميع الطلاب
تصدير تقرير بالمؤدبين الفصلي لطالب لطلاب مختارين لجميع الطلاب
تصدير تقرير بالمؤدبين للعام لطالب لطلاب مختارين لجميع الطلاب

تصدير تقرير بالمؤدبين اليومي لصف لصفوف مختارة لجميع الصفوف
تصدير تقرير بالمؤدبين الأسبوعي لصف لصفوف مختارة لجميع الصفوف
تصدير تقرير بالمؤدبين الشهري لصف لصفوف مختارة لجميع الصفوف
تصدير تقرير بالمؤدبين الفصلي لصف لصفوف مختارة لجميع الصفوف
تصدير تقرير بالمؤدبين للعام لصف لصفوف مختارة لجميع الصفوف

تصدير تقرير بالمزعجين اليومي لطالب لطلاب مختارين لجميع الطلاب
تصدير تقرير بالمزعجين الأسبوعي لطالب لطلاب مختارين لجميع الطلاب
تصدير تقرير بالمزعجين الشهري لطالب لطلاب مختارين لجميع الطلاب
تصدير تقرير بالمزعجين الفصلي لطالب لطلاب مختارين لجميع الطلاب
تصدير تقرير بالمزعجين للعام لطالب لطلاب مختارين لجميع الطلاب

تصدير تقرير بالمزعجين اليومي لصف لصفوف مختارة لجميع الصفوف
تصدير تقرير بالمزعجين الأسبوعي لصف لصفوف مختارة لجميع الصفوف
تصدير تقرير بالمزعجين الشهري لصف لصفوف مختارة لجميع الصفوف
تصدير تقرير بالمزعجين الفصلي لصف لصفوف مختارة لجميع الصفوف
تصدير تقرير بالمزعجين للعام لصف لصفوف مختارة لجميع الصفوف



١ـ مسلسل رقم متسلسل تلقائي
٢ـ الاسم
٣ـ الجنس قائمة منسدلة  ( ذكر أنثى)
٤ـ الفصل الدراسي قائمة منسدلة  ( الأول الثاني)
٥ـ الشهر قائمة منسدلة  ( محرم صفر ربيع أول ربيع ثاني جماد أول جماد ثاني رجب شعبان)
٦ـ الأسبوع ( الأول  الثاني الثالث الرابع)  
٧ـ اليوم قائمة منسدلة ( السبت الأحد الإثنين الثلاثاء الأربعاء الخميس )
٨ـ التاريخ يكتب تلقائي  ( هجري ميلادي)
٩ـ الصف ( أول أساسي ثاني أساسي ثالث أساسي رابع أساسي خامس أساسي سادس أساسي سابع أساسي ثامن أساسي تاسع أساسي ـ أول ثانوي ـ ثاني ثانوي ـ ثالث ثانوي)
١٠ـ الشعبة قائمة منسدلة ( أ ب ج بدون)
١١ـ العلامات قائمة منسدلة كما في الجدول التالي 
متأخر	نعم	بعذر ـ بدون عذر	
	لا		
حاضر	نعم		
	لا		
غائب يوم كامل	نعم	بإذن ـ بدون إذن	خيارات بدون إذن (بعذر ـ بدون عذر
	لا		
منصرف بالراحة	نعم	بإذن ـ بدون إذن	خيارات بدون إذن (بعذر ـ بدون عذر
	لا		
مزعج	نعم		
	لا		
مؤدب	نعم		
	لا		
أخرى			
الدرجة 			
١٢ـ ملاحظات ادخال يدوي قائمة منسدلة نص قائمة

ملاحظة تحكم بكتابة وقت التأخر. للمتاخرين مدة التأخر : مثلا ساعة دقيقة 6 : 1
 
ملاحظة مهمة
درجة المواظبة تكون تكتب تلقائي من الحضور والغياب كون ايام الدراسة بالشهر ٢٠ ÷ ٤= ٥ وهذا الناتج هو الدرجة المقررة لجانب المواظبة في كشوفات الدرجات الشهرية ويكون يكتب بعمود الدرجة تلقائي مع امكانية التعديل إذا لزم ومنه يكتب تلقائي في عمود المواظبة في الدرجات الشهرية


سؤال في حالة كان هناك يوم عطلة من ال ٢٠ حق الشهر كيف شيكون يعمل الموقع
انا رأي اضبط الموقع على أنه يكون ينزل العلامات بيوم العطلة كاملة للجميع 

عمل جنب هذه الفقرة بيانات الاضافة:-
ـ تاريخ الإضافة قائمة منسدلة  ( هجري ميلادي) الساعة 
ـ اليوم قائمة منسدلة ( السبت الأحد الإثنين الثلاثاء الأربعاء الخميس الجمعة )
ـ اسم الموظف الذي اضاف : 

عمل جنب هذه الفقرة هذه الخيارات:-
ـ زر ( معاينة ) لعرض العمل والنظر فيه قبل تصديره الخيارات التي يتم تثبيتها ورقة × ورقة ـ أو ورقتين  × ورقة ـ أو ثلاث ورق  × ورقة ـ أو أربع ورق  × ورقة
ـ تصدير التقرير يكون ورقة × ورقة. ـ أو ورقتين  × ورقة ـ  أو ثلاث ورق  × ورقة ـ أو أربع ورق  × ورقة
ـ تصدير عمود أعمدة مختارة جميع الأعمدة قائمة منسدلة  (  Excel PDF )
ـ تصديرها لملف قائمة منسدلة  (  Excel PDF )
ـ تصدير للفصل الثاني ـ تصدير للعام القادم 
ـ استيراد من الفصل الأول استيراد من العام الماضي
ـ مشاركة قائمة منسدلة  (  Excel PDF صورة )
ـ الاحتفاظ بصور الاشعارات للرجوع لها إذا لزم

عمل جنب هذه الفقرة هذه الازرار
🔲 معاينة 🔲 تحرير 🔲 تعديل 🔲 إضافة 🔲 تصدير 🔲 حذف 🔲 فتحPDF 🔲 فتحExcel 🔲 طباعة 🔲 مشاركة
➖➖➖➖➖➖➖➖➖➖➖ فاصل بين الفقرات الفرعية ➖➖➖➖➖➖➖➖➖➖➖➖➖
٤  ـ ٣ ـ كتب الطلاب والطالبات
من يكتب هذه الفقرة	مسؤل الصف أو من يكلفه المدير من الموظفين في حينه وموافقة المدير على الادخال أو التعديل
ولا يعدلها الموظف أو مسؤل الصف الا بموافقة المدير في حينه
من يشاهد هذه الفقرة	مسؤل الصف يشاهد صفه فقط
من يشاهد تقارير هذه الفقرة	المدير
من يصدر تقارير هذه الفقرة	المدير
خيارات البحث بهذه الفقرة يكون بواسطة 	طالب طالبة فقرة كلمة جملة

نوع التقارير المطلوب من الموقع تصديرها من هذه الفقرة
تصدير تقرير بالكتب الموجودة لمادة لمواد مختارة لجميع المواد
تصدير تقرير بالكتب الموجودة لصف لصفوف مختارة لجميع الصفوف
تصدير تقرير بالكتب الموجودة لطالب لطلاب مختارين لجميع الطلاب

تصدير تقرير بالكتب الغير موجودة لمادة لمواد مختارة لجميع المواد
تصدير تقرير بالكتب الغير موجودة لصف لصفوف مختارة لجميع الصفوف
تصدير تقرير بالكتب الغير موجودة لطالب لطلاب مختارين لجميع الطلاب

تصدير تقرير بالكتب المملوكة لمادة لمواد مختارة لجميع المواد
تصدير تقرير بالكتب المملوكة لصف لصفوف مختارة لجميع الصفوف
تصدير تقرير بالكتب المملوكة لطالب لطلاب مختارين لجميع الطلاب

تصدير تقرير بالكتب التابعة للمدرسة لمادة لمواد مختارة لجميع المواد
تصدير تقرير بالكتب التابعة للمدرسة لصف لصفوف مختارة لجميع الصفوف
تصدير تقرير بالكتب التابعة للمدرسة لطالب لطلاب مختارين لجميع الطلاب



١ـ مسلسل رقم متسلسل تلقائي
٢ـ الاسم
٣ـ الجنس قائمة منسدلة  ( ذكر أنثى)
٤ـ الفصل الدراسي قائمة منسدلة  ( الأول الثاني)
٥ـ الشهر قائمة منسدلة  ( محرم صفر ربيع أول ربيع ثاني جماد أول جماد ثاني رجب شعبان)
٦ـ الأسبوع ( الأول  الثاني الثالث الرابع)  
٧ـ اليوم قائمة منسدلة ( السبت الأحد الإثنين الثلاثاء الأربعاء الخميس )
٨ـ التاريخ يكتب تلقائي  ( هجري ميلادي)
٩ـ الصف ( أول أساسي ثاني أساسي ثالث أساسي رابع أساسي خامس أساسي سادس أساسي سابع أساسي ثامن أساسي تاسع أساسي ـ أول ثانوي ـ ثاني ثانوي ـ ثالث ثانوي)
١٠ـ الشعبة قائمة منسدلة ( أ ب ج بدون)
١١ـ أسماء الكتب قائمة منسدلة  ( القرآن الإسلامية  اللغة العربية E حصة E واجب الرياضيات العلوم تاريخ جغرافيا وطنية أخرى )
١٢ـ الجزء قائمة منسدلة  ( ج١ ـ ج٢ ـ جزئين )
١٣ـ جهة الكتاب قائمة منسدلة ( م ملك ـ م مدرسة ـ لا ) ملاحظة م ملك تعني موجود وهو ملك الطالب بمعنى اشتروه هو
ـ م مدرسة تعني موجود وهو مستلم من المدرسة ـ لا تعني غير موجود
١٤ـ تسليم كتب المدرسة قائمة منسدلة ( سلم ـ لا)
١٥ـ اسم معلم المادة
١٦ـ ملاحظات ادخال يدوي قائمة منسدلة نص قائمة

عمل جنب هذه الفقرة بيانات الاضافة:-
ـ تاريخ الإضافة قائمة منسدلة  ( هجري ميلادي) الساعة 
ـ اليوم قائمة منسدلة ( السبت الأحد الإثنين الثلاثاء الأربعاء الخميس الجمعة )
ـ اسم الموظف الذي اضاف : 

عمل جنب هذه الفقرة هذه الخيارات:-
ـ زر ( معاينة ) لعرض العمل والنظر فيه قبل تصديره الخيارات التي يتم تثبيتها ورقة × ورقة ـ أو ورقتين  × ورقة ـ أو ثلاث ورق  × ورقة ـ أو أربع ورق  × ورقة
ـ تصدير التقرير يكون ورقة × ورقة. ـ أو ورقتين  × ورقة ـ  أو ثلاث ورق  × ورقة ـ أو أربع ورق  × ورقة
ـ تصدير عمود أعمدة مختارة جميع الأعمدة قائمة منسدلة  (  Excel PDF )
ـ تصديرها لملف قائمة منسدلة  (  Excel PDF )
ـ تصدير للفصل الثاني ـ تصدير للعام القادم 
ـ استيراد من الفصل الأول استيراد من العام الماضي
ـ مشاركة قائمة منسدلة  (  Excel PDF صورة )
ـ الاحتفاظ بصور الاشعارات للرجوع لها إذا لزم

عمل جنب هذه الفقرة هذه الازرار
🔲 معاينة 🔲 تحرير 🔲 تعديل 🔲 إضافة 🔲 تصدير 🔲 حذف 🔲 فتحPDF 🔲 فتحExcel 🔲 طباعة 🔲 مشاركة
➖➖➖➖➖➖➖➖➖➖➖ فاصل بين الفقرات الفرعية ➖➖➖➖➖➖➖➖➖➖➖➖➖

٤  ـ ٤ ـ أداء الطلاب والطالبات لفقرات طابور الصباح

من يكتب هذه الفقرة	مسؤل الصف أو من يكلفه المدير من الموظفين في حينه وموافقة المدير على الادخال أو التعديل
ولا يعدلها الموظف أو مسؤل الصف الا بموافقة المدير في حينه
من يشاهد هذه الفقرة	مسؤل الصف يشاهد صفه فقط
من يشاهد تقارير هذه الفقرة	المدير
من يصدر تقارير هذه الفقرة	المدير
خيارات البحث بهذه الفقرة يكون بواسطة 	طالب طالبة فقرة كلمة جملة
التقارير المطلوبة من هذه الفقرة	تصدير تقرير الفقرات المنفذة لفقرة لفقرات مختارة لجميع الفقرات ـ لفصل ـ لعام
تصدير تقرير القاء الفقرات لطالب لطلاب مختارين ـ لجميع الطلاب  ـ لفصل ـ لعام
تصدير تقرير عدم القاء الفقرات لطالب لطلاب مختارين ـ لجميع الطلاب  ـ لفصل ـ لعام
تصدير تقرير القاء الفقرات لصف لصفوف مختارة ـ لجميع الصفوف ـ لفصل ـ لعام
تصدير تقرير تقييم الطابور لصف لصفوف مختارة ـ لجميع الصفوف لفصل ـ لعام
تصدير تقرير لمقيم الطابور لموظف لموظفين مختارين ـ لجميع الموظفين لفصل ـ لعام


١ـ مسلسل رقم متسلسل تلقائي
٢ـ الاسم
٣ـ الجنس قائمة منسدلة  ( ذكر أنثى)
٤ـ الفصل الدراسي قائمة منسدلة  ( الأول الثاني)
٥ـ الشهر قائمة منسدلة  ( محرم صفر ربيع أول ربيع ثاني جماد أول جماد ثاني رجب شعبان)
٦ـ الأسبوع ( الأول  الثاني الثالث الرابع)  
٧ـ اليوم قائمة منسدلة ( السبت الأحد الإثنين الثلاثاء الأربعاء الخميس )
٨ـ التاريخ يكتب تلقائي  ( هجري ميلادي)
٩ـ الصف ( أول أساسي ثاني أساسي ثالث أساسي رابع أساسي خامس أساسي سادس أساسي سابع أساسي ثامن أساسي تاسع أساسي ـ أول ثانوي ـ ثاني ثانوي ـ ثالث ثانوي)
١٠ـ الشعبة قائمة منسدلة ( أ ب ج بدون)
١١ـ اسم الفقرة قائمة منسدلة ( المقدم قرآن حديث حكم هل تعلم الكلمة الشعر يعجنني ولا يعجبني نصائح وفوائد فقرة أوائل الأسئلة الثقافية فقرة المسابقة النشيد الشعار قرآن الختام أخرى )
١٢ـ تم القاء الفقرة قائمة منسدلة  ( نعم ـ لا )
١٣ـ درجة تقييم الفقرة الدرجة المقررة (  رقم متسلسل ) الدرجة المستحقة ( رقم متسلسل  )
١٤ـ اسم المكلف بالتقييم :
١٥ـ ملاحظات ادخال يدوي قائمة منسدلة نص قائمة

عمل جنب هذه الفقرة بيانات الاضافة:-
ـ تاريخ الإضافة قائمة منسدلة  ( هجري ميلادي) الساعة 
ـ اليوم قائمة منسدلة ( السبت الأحد الإثنين الثلاثاء الأربعاء الخميس الجمعة )
ـ اسم الموظف الذي اضاف : 

عمل جنب هذه الفقرة هذه الخيارات:-
ـ زر ( معاينة ) لعرض العمل والنظر فيه قبل تصديره الخيارات التي يتم تثبيتها ورقة × ورقة ـ أو ورقتين  × ورقة ـ أو ثلاث ورق  × ورقة ـ أو أربع ورق  × ورقة
ـ تصدير التقرير يكون ورقة × ورقة. ـ أو ورقتين  × ورقة ـ  أو ثلاث ورق  × ورقة ـ أو أربع ورق  × ورقة
ـ تصدير عمود أعمدة مختارة جميع الأعمدة قائمة منسدلة  (  Excel PDF )
ـ تصديرها لملف قائمة منسدلة  (  Excel PDF )
ـ تصدير للفصل الثاني ـ تصدير للعام القادم 
ـ استيراد من الفصل الأول استيراد من العام الماضي
ـ مشاركة قائمة منسدلة  (  Excel PDF صورة )
ـ الاحتفاظ بصور الاشعارات للرجوع لها إذا لزم

عمل جنب هذه الفقرة هذه الازرار
🔲 معاينة 🔲 تحرير 🔲 تعديل 🔲 إضافة 🔲 تصدير 🔲 حذف 🔲 فتحPDF 🔲 فتحExcel 🔲 طباعة 🔲 مشاركة
➖➖➖➖➖➖➖➖➖➖➖ فاصل بين الفقرات الفرعية ➖➖➖➖➖➖➖➖➖➖➖➖➖

٤  ـ ٥ ـ حالات الاسعافات الأولية

من يكتب هذه الفقرة	مسؤل الصف أو من يكلفه المدير من الموظفين في حينه وموافقة المدير على الادخال أو التعديل
ولا يعدلها الموظف أو مسؤل الصف الا بموافقة المدير في حينه
من يشاهد هذه الفقرة	مسؤل الصف يشاهد صفه فقط
من يشاهد تقارير هذه الفقرة	المدير
من يصدر تقارير هذه الفقرة	المدير
خيارات البحث بهذه الفقرة يكون بواسطة 	طالب طالبة فقرة كلمة جملة
التقارير المطلوبة من هذه الفقرة	تصدير تقرير بالحالات المنفذة لطالب لطلاب مختارين ـ لجميع الطلاب
تصدير تقرير بالحالات المنفذة لصف لصفوف مختارة ـ لجميع الصفوف
تصدير تقرير بالحالات المنفذة اليومية الأسبوعية الشهرية لفصل لعام

١ـ مسلسل رقم متسلسل تلقائي
٢ـ الاسم
٣ـ الجنس قائمة منسدلة  ( ذكر أنثى)
٤ـ الفصل الدراسي قائمة منسدلة  ( الأول الثاني)
٥ـ الشهر قائمة منسدلة  ( محرم صفر ربيع أول ربيع ثاني جماد أول جماد ثاني رجب شعبان)
٦ـ الأسبوع ( الأول  الثاني الثالث الرابع)  
٧ـ اليوم قائمة منسدلة ( السبت الأحد الإثنين الثلاثاء الأربعاء الخميس )
٨ـ التاريخ يكتب تلقائي  ( هجري ميلادي)
٩ـ الصف ( أول أساسي ثاني أساسي ثالث أساسي رابع أساسي خامس أساسي سادس أساسي سابع أساسي ثامن أساسي تاسع أساسي ـ أول ثانوي ـ ثاني ثانوي ـ ثالث ثانوي)
١٠ـ الشعبة قائمة منسدلة ( أ ب ج بدون)
١١ـ تفاصيل الحالة (نص قائمة)
١٢ـ الاجراءات المنفذة (نص قائمة)
١٣ـ اسم المنفذ (نص)
١٤ـ ملاحظات ادخال يدوي قائمة منسدلة نص قائمة

عمل جنب هذه الفقرة بيانات الاضافة:-
ـ تاريخ الإضافة قائمة منسدلة  ( هجري ميلادي) الساعة 
ـ اليوم قائمة منسدلة ( السبت الأحد الإثنين الثلاثاء الأربعاء الخميس الجمعة )
ـ اسم الموظف الذي اضاف : 

عمل جنب هذه الفقرة هذه الخيارات:-
ـ زر ( معاينة ) لعرض العمل والنظر فيه قبل تصديره الخيارات التي يتم تثبيتها ورقة × ورقة ـ أو ورقتين  × ورقة ـ أو ثلاث ورق  × ورقة ـ أو أربع ورق  × ورقة
ـ تصدير التقرير يكون ورقة × ورقة. ـ أو ورقتين  × ورقة ـ  أو ثلاث ورق  × ورقة ـ أو أربع ورق  × ورقة
ـ تصدير عمود أعمدة مختارة جميع الأعمدة قائمة منسدلة  (  Excel PDF )
ـ تصديرها لملف قائمة منسدلة  (  Excel PDF )
ـ تصدير للفصل الثاني ـ تصدير للعام القادم 
ـ استيراد من الفصل الأول استيراد من العام الماضي
ـ مشاركة قائمة منسدلة  (  Excel PDF صورة )
ـ الاحتفاظ بصور الاشعارات للرجوع لها إذا لزم

عمل جنب هذه الفقرة هذه الازرار
🔲 معاينة 🔲 تحرير 🔲 تعديل 🔲 إضافة 🔲 تصدير 🔲 حذف 🔲 فتحPDF 🔲 فتحExcel 🔲 طباعة 🔲 مشاركة
➖➖➖➖➖➖➖➖➖➖➖ فاصل بين الفقرات الفرعية ➖➖➖➖➖➖➖➖➖➖➖➖➖
٤  ـ ٦ ـ الدليل الإرشادي

من يكتب هذه الفقرة	المدير أو من يكلفه المدير من الموظفين في حينه وموافقة المدير على الادخال أو التعديل
من يشاهد هذه الفقرة	الموظف ومسؤل الصف
من يشاهد تقارير هذه الفقرة	المدير
من يصدر تقارير هذه الفقرة	المدير
خيارات البحث بهذه الفقرة يكون بواسطة 	فقرة كلمة جملة
التقارير المطلوبة من هذه الفقرة	تصدير تقرير المنفذ لموظف لموظفين مختارين لجميع الموظفين 
تصدير تقرير ما نفذ لموضوع لمواضيع مختارة لجميع المواضيع
تصدير تقرير بالمواضيع المنفذة اليومية الأسبوعية الشهرية لفصل لعام

١ـ مسلسل رقم متسلسل تلقائي
٢ـ الجانب ( نص)
٣ـ الموضوع قائمة منسدلة ( نص قائمة
٤ـ مسلسل رقم متسلسل تلقائي
٥ـ الارشادات قائمة منسدلة ( نص قائمة
٦ـ طرحت للطلاب قائمة منسدلة ( نعم لا)
٧ـ اسم الموظف : يعني الذي طرحها للطلاب
٨ـ الفصل الدراسي قائمة منسدلة  ( الأول الثاني)
٩ـ الشهر قائمة منسدلة  ( محرم صفر ربيع أول ربيع ثاني جماد أول جماد ثاني رجب شعبان)
١٠ـ الأسبوع ( الأول  الثاني الثالث الرابع)  
١١ـ اليوم قائمة منسدلة ( السبت الأحد الإثنين الثلاثاء الأربعاء الخميس )
١٢ـ التاريخ يكتب تلقائي  ( هجري ميلادي)
١٣ـ الصف ( أول أساسي ثاني أساسي ثالث أساسي رابع أساسي خامس أساسي سادس أساسي سابع أساسي ثامن أساسي تاسع أساسي ـ أول ثانوي ـ ثاني ثانوي ـ ثالث ثانوي)
١٤ـ الشعبة قائمة منسدلة ( أ ب ج بدون)
 
👈عمل جنب هذه الفقرة مايلي:-
تصدير لعام ـ تصدير لفصل ـ استرداد من عام استرداد من فصل ـ استرداد الفقرة كاملة ـ استرداد فقرة مختارة ـ استرداد جميع الفقرات

مثال لهذه الفقرة
الجانب الأدوات المدرسية الموضوع أدوات الأخرين :
1.	أن أحرص ما أمكن بأن لا أستلف شي من أدوات الآخرين 
2.	أن لا ألمس شيء من أدوات الآخرين إلا بإذن
3.	أن لا أستخدم أدوات الآخرين إلا بإذن منهم
4.	أن أتعامل مع أدواتي بحرص ومع أدوات الآخرين بحرص أكبر
5.	أن أحافظ على أدوات الآخرين كما أحافظ على أدواتي
6.	أن أحافظ على الأدوات الدراسية المستعارة من الآخرين .
7.	أن أعيد ما استعرته من أدوات الآخرين كما هي
8.	أن أعيد ما أستعرته من أدوات الآخرين مباشرة 
9.	أن أرد أي شيء وجدته مفقودا من أدوات الآخرين إلى أصحابها
10.	أن أدرك أن محافظتي على أدوات الآخرين دليل على ذوقي وإهتمامي
11.	أن أتحمل مسؤولية المحافظة على أدوات الآخرين من أي شيء يلحق بها مني

عمل جنب هذه الفقرة بيانات الاضافة:-
ـ تاريخ الإضافة قائمة منسدلة  ( هجري ميلادي) الساعة 
ـ اليوم قائمة منسدلة ( السبت الأحد الإثنين الثلاثاء الأربعاء الخميس الجمعة )
ـ اسم الموظف الذي اضاف : 

عمل جنب هذه الفقرة هذه الخيارات:-
ـ زر ( معاينة ) لعرض العمل والنظر فيه قبل تصديره الخيارات التي يتم تثبيتها ورقة × ورقة ـ أو ورقتين  × ورقة ـ أو ثلاث ورق  × ورقة ـ أو أربع ورق  × ورقة
ـ تصدير التقرير يكون ورقة × ورقة. ـ أو ورقتين  × ورقة ـ  أو ثلاث ورق  × ورقة ـ أو أربع ورق  × ورقة
ـ تصدير عمود أعمدة مختارة جميع الأعمدة قائمة منسدلة  (  Excel PDF )
ـ تصديرها لملف قائمة منسدلة  (  Excel PDF )
ـ تصدير للفصل الثاني ـ تصدير للعام القادم 
ـ استيراد من الفصل الأول استيراد من العام الماضي
ـ مشاركة قائمة منسدلة  (  Excel PDF صورة )
ـ الاحتفاظ بصور الاشعارات للرجوع لها إذا لزم

عمل جنب هذه الفقرة هذه الازرار
🔲 معاينة 🔲 تحرير 🔲 تعديل 🔲 إضافة 🔲 تصدير 🔲 حذف 🔲 فتحPDF 🔲 فتحExcel 🔲 طباعة 🔲 مشاركة
➖➖➖➖➖➖➖➖➖➖➖ فاصل بين الفقرات الفرعية ➖➖➖➖➖➖➖➖➖➖➖➖➖

٤  ـ ٧ ـ الإشعارات الصادرة لأولياء الأمور
من يكتب هذه الفقرة	الموظف ومسؤل الصف أو من يكلفه المدير من الموظفين في حينه وموافقة المدير على الادخال أو التعديل
ولا يعدلها الموظف أو مسؤل الصف الا بموافقة المدير في حينه
من يشاهد هذه الفقرة	الموظف ومسؤل الصف كل واحد يشاهد عمله فقط
من يشاهد تقارير هذه الفقرة	المدير
من يصدر تقارير هذه الفقرة	المدير
خيارات البحث بهذه الفقرة يكون بواسطة 	طالب طالبة فقرة كلمة جملة
التقارير المطلوبة من هذه الفقرة	تصدير تقرير بالاشعارات الايجابي المنفذة لطالب لطلاب مختارين ـ لجميع الطلاب
تصدير تقرير بالاشعارات الايجابي المنفذة لصف لصفوف مختارة ـ لجميع الصفوف
تصدير تقرير بالاشعارات الايجابي المنفذة لموظف لموظفين مختارين لجميع الموظفين
تصدير تقرير بالاشعارات الايجابي المنفذة ليوم لأسبوع لشهر لفصل لعام

تصدير تقرير بالاشعارات السلبي المنفذة لطالب لطلاب مختارين ـ لجميع الطلاب
تصدير تقرير بالاشعارات السلبي المنفذة لصف لصفوف مختارة ـ لجميع الصفوف
تصدير تقرير بالاشعارات السلبي المنفذة لموظف لموظفين مختارين لجميع الموظفين
تصدير تقرير بالاشعارات السلبي المنفذة ليوم لأسبوع لشهر لفصل لعام

تصدير تقرير بالاشعارات الايجابي الغير منفذ لطالب لطلاب مختارين ـ لجميع الطلاب
تصدير تقرير بالاشعارات الايجابي الغير منفذ لصف لصفوف مختارة ـ لجميع الصفوف
تصدير تقرير بالاشعارات الايجابي الغير منفذ لموظف لموظفين مختارين لجميع الموظفين
تصدير تقرير بالاشعارات الايجابي الغير منفذ ليوم لأسبوع لشهر لفصل لعام

تصدير تقرير بالاشعارات السلبي الغير منفذ لطالب لطلاب مختارين ـ لجميع الطلاب
تصدير تقرير بالاشعارات السلبي الغير منفذ لصف لصفوف مختارة ـ لجميع الصفوف
تصدير تقرير بالاشعارات السلبي الغير منفذ لموظف لموظفين مختارين لجميع الموظفين
تصدير تقرير بالاشعارات السلبي الغير منفذ ليوم لأسبوع لشهر لفصل لعام

١ـ مسلسل رقم متسلسل تلقائي
٢ـ الاسم
٣ـ الجنس قائمة منسدلة  ( ذكر أنثى)
٤ـ الفصل الدراسي قائمة منسدلة  ( الأول الثاني)
٥ـ الشهر قائمة منسدلة  ( محرم صفر ربيع أول ربيع ثاني جماد أول جماد ثاني رجب شعبان)
٦ـ الأسبوع ( الأول  الثاني الثالث الرابع)  
٧ـ اليوم قائمة منسدلة ( السبت الأحد الإثنين الثلاثاء الأربعاء الخميس )
٨ـ التاريخ يكتب تلقائي  ( هجري ميلادي)
٩ـ الصف ( أول أساسي ثاني أساسي ثالث أساسي رابع أساسي خامس أساسي سادس أساسي سابع أساسي ثامن أساسي تاسع أساسي ـ أول ثانوي ـ ثاني ثانوي ـ ثالث ثانوي)
١٠ـ الشعبة قائمة منسدلة ( أ ب ج بدون)
١١ـ موضوع الاشعار ادخال يدوي قائمة منسدلة نص قائمة
١٢ـ رقم الاشعار عدد متسلسل يكتب تلقائي اذا كان هذا الاشعار اول اشعار يكتب واحد وإذا قد سبق وكتب له اشعار يكتب اثنين ولو سبق كتب له ثلاثة يكتب ثلاثة مع امكانية التعديل إذا لزم
١٣ـ نوع الاشعار قائمة منسدلة ( ايجابي ـ سلبي )

١٤ـ نص الاشعار
الأخ  قائمة منسدلة ( الأستاذ الشيخ الدكتور أخرى ) ــــــ. وهنا اسم ولي الأمرـــــــــ      الأكرم 
تحية طيبة وبعد :-
تهديكم إدارة مدرسة التعاون الأساسية أطيب التحايا ونضع بين يديكم هذا الرسالة 
بخصوص الطالب/ الطالبة  / ــــــــــ
 الصف / ــــــــــــــــــــــــــ.     
حيث لديه / لديها السلوكيات قائمة منسدلة ( الحميدة ـ المخالفة ) الآتية:- 
١ـ 
٢ـ 
٣ـ 
٤ـ 

وعليه : وهنا يكتب المطلوب من ولي الأمر 

١٥ـ اسم الموظف محرر الاشعار يظهر تلقائي 
١٦ـ طريقة ارساله ورقي ( نعم لا ) ـ عبر الواتس ( نعم لا )
١٧ـ اسم الرسول ادخال نص/ قائمة
١٨ـ تم ارساله قائمة منسدلة ( نعم لا )
١٩ـ النتائج قائمة منسدلة نص/ قائمة
٢٠ـ ملاحظات ادخال يدوي قائمة منسدلة نص قائمة


عمل جنب هذه الفقرة بيانات الاضافة:-
ـ تاريخ الإضافة قائمة منسدلة  ( هجري ميلادي) الساعة 
ـ اليوم قائمة منسدلة ( السبت الأحد الإثنين الثلاثاء الأربعاء الخميس الجمعة )
ـ اسم الموظف الذي اضاف : 

عمل جنب هذه الفقرة هذه الخيارات:-
ـ زر ( معاينة ) لعرض العمل والنظر فيه قبل تصديره الخيارات التي يتم تثبيتها ورقة × ورقة ـ أو ورقتين  × ورقة ـ أو ثلاث ورق  × ورقة ـ أو أربع ورق  × ورقة
ـ تصدير التقرير يكون ورقة × ورقة. ـ أو ورقتين  × ورقة ـ  أو ثلاث ورق  × ورقة ـ أو أربع ورق  × ورقة
ـ تصدير عمود أعمدة مختارة جميع الأعمدة قائمة منسدلة  (  Excel PDF )
ـ تصديرها لملف قائمة منسدلة  (  Excel PDF )
ـ تصدير للفصل الثاني ـ تصدير للعام القادم 
ـ استيراد من الفصل الأول استيراد من العام الماضي
ـ مشاركة قائمة منسدلة  (  Excel PDF صورة )
ـ الاحتفاظ بصور الاشعارات للرجوع لها إذا لزم

عمل جنب هذه الفقرة هذه الازرار
🔲 معاينة 🔲 تحرير 🔲 تعديل 🔲 إضافة 🔲 تصدير 🔲 حذف 🔲 فتحPDF 🔲 فتحExcel 🔲 طباعة 🔲 مشاركة
➖➖➖➖➖➖➖➖➖➖➖ فاصل بين الفقرات الفرعية ➖➖➖➖➖➖➖➖➖➖➖➖➖

٤  ـ ٨ ـ إدارة الصفوف الدراسية

من يكتب هذه الفقرة	مسؤل الصف أو من يكلفه المدير من الموظفين في حينه وموافقة المدير على الادخال أو التعديل
ولا يعدلها مسؤل الصف الا بموافقة المدير في حينه
من يشاهد هذه الفقرة	مسؤل الصف يشاهد عمله فقط
من يشاهد تقارير هذه الفقرة	المدير
من يصدر تقارير هذه الفقرة	المدير
خيارات البحث بهذه الفقرة يكون بواسطة 	طالب طالبة فقرة كلمة جملة
التقارير المطلوبة من هذه الفقرة	تصدير تقرير صف لصفوف لجميع الصفوف لفصل لعام

 الفصل الدراسي قائمة منسدلة  ( الأول الثاني)
وهذا شكل كل فصل دراسي


































عمل جنب هذه الفقرة بيانات الاضافة:-
ـ تاريخ الإضافة قائمة منسدلة  ( هجري ميلادي) الساعة 
ـ اليوم قائمة منسدلة ( السبت الأحد الإثنين الثلاثاء الأربعاء الخميس الجمعة )
ـ اسم الموظف الذي اضاف : 

عمل جنب هذه الفقرة هذه الخيارات:-
ـ زر ( معاينة ) لعرض العمل والنظر فيه قبل تصديره الخيارات التي يتم تثبيتها ورقة × ورقة ـ أو ورقتين  × ورقة ـ أو ثلاث ورق  × ورقة ـ أو أربع ورق  × ورقة
ـ تصدير التقرير يكون ورقة × ورقة. ـ أو ورقتين  × ورقة ـ  أو ثلاث ورق  × ورقة ـ أو أربع ورق  × ورقة
ـ تصدير عمود أعمدة مختارة جميع الأعمدة قائمة منسدلة  (  Excel PDF )
ـ تصديرها لملف قائمة منسدلة  (  Excel PDF )
ـ تصدير للفصل الثاني ـ تصدير للعام القادم 
ـ استيراد من الفصل الأول استيراد من العام الماضي
ـ مشاركة قائمة منسدلة  (  Excel PDF صورة )
ـ الاحتفاظ بصور الاشعارات للرجوع لها إذا لزم

عمل جنب هذه الفقرة هذه الازرار
🔲 معاينة 🔲 تحرير 🔲 تعديل 🔲 إضافة 🔲 تصدير 🔲 حذف 🔲 فتحPDF 🔲 فتحExcel 🔲 طباعة 🔲 مشاركة

➖➖➖➖➖➖➖➖➖➖➖ فاصل بين الفقرات الفرعية ➖➖➖➖➖➖➖➖➖➖➖➖➖
٤  ـ ٩ ـ الوسائل التعليمية

من يكتب هذه الفقرة	الموظف ومسؤل الصف أو من يكلفه المدير من الموظفين في حينه وموافقة المدير على الادخال أو التعديل
ولا يعدلها الموظف أو مسؤل الصف الا بموافقة المدير في حينه
من يشاهد هذه الفقرة	الموظف ومسؤل الصف كل واحد يشاهد عمله فقط
من يشاهد تقارير هذه الفقرة	المدير
من يصدر تقارير هذه الفقرة	المدير
خيارات البحث بهذه الفقرة يكون بواسطة 	طالب طالبة فقرة كلمة جملة
التقارير المطلوبة من هذه الفقرة	تصدير تقرير بالوسائل المنفذة لطالب لطلاب مختارين ـ لجميع الطلاب لفصل لعام
تصدير تقرير بالوسائل المنفذة لمادة لمواد مختارة لجميع المواد لفصل لعام
تصدير تقرير بالوسائل المنفذة لصف لصفوف مختارة ـ لجميع الصفوف لفصل لعام
تصدير تقرير بالوسائل المنفذة لموظف لموظفين مختارين لجميع الموظفين لفصل لعام
تصدير تقرير بالوسائل المنفذة ليوم لأسبوع لشهر لفصل لعام

تصدير تقرير بالوسائل الغير منفذة لطالب لطلاب مختارين ـ لجميع الطلاب لفصل لعام
تصدير تقرير بالوسائل الغير منفذة لمادة لمواد مختارة لجميع المواد لفصل لعام
تصدير تقرير بالوسائل الغير منفذة لصف لصفوف مختارة ـ لجميع الصفوف لفصل لعام
تصدير تقرير بالوسائل الغير منفذة لموظف لموظفين مختارين لجميع الموظفين لفصل لعام
تصدير تقرير بالوسائل الغير منفذة ليوم لأسبوع لشهر لفصل لعام



١ـ مسلسل رقم متسلسل تلقائي
٢ـ الاسم
٣ـ الجنس قائمة منسدلة  ( ذكر أنثى)
٤ـ الفصل الدراسي قائمة منسدلة  ( الأول الثاني)
٥ـ الشهر قائمة منسدلة  ( محرم صفر ربيع أول ربيع ثاني جماد أول جماد ثاني رجب شعبان)
٦ـ الأسبوع ( الأول  الثاني الثالث الرابع)  
٧ـ اليوم قائمة منسدلة ( السبت الأحد الإثنين الثلاثاء الأربعاء الخميس )
٨ـ التاريخ يكتب تلقائي  ( هجري ميلادي)
٩ـ الصف ( أول أساسي ثاني أساسي ثالث أساسي رابع أساسي خامس أساسي سادس أساسي سابع أساسي ثامن أساسي تاسع أساسي ـ أول ثانوي ـ ثاني ثانوي ـ ثالث ثانوي)
١٠ـ الشعبة قائمة منسدلة ( أ ب ج بدون)
١١ـ مادة الوسيلة قائمة منسدلة  ( القرآن الإسلامية  اللغة العربية E حصة E واجب الرياضيات العلوم تاريخ جغرافيا وطنية أخرى )
١٢ـ الجزء قائمة منسدلة  ( ج١ ـ ج٢ ـ جزئين )
١٣ـ نوع الوسيلة قائمة منسدلة ( مجلة ـ كروت ـ أخرى )
١٤ـ حجم الوسيلة قائمة منسدلة  ( كبيرة ـ وسط ـ صغيرة) اخرى
١٥ـ الوسيلة قائمة منسدلة  ( جديدة ـ مكتوبة )
١٦ـ مقدم الوسيلة قائمة منسدلة  ( فردي ـ مشترك )
١٧ـ اسماء المشتركين في حالة وجود أكثر من واحد ١ـ ..................... ٢ ـ .................... ٣ـ ..............
١٨ـ فيها أخطاء قائمة منسدلة  ( نعم ـ لا )
١٩ـ علقت الوسيلة قائمة منسدلة  ( نعم ـ لا )
٢٠ـ اسم معلم المادة
٢١ـ ملاحظات ادخال يدوي قائمة منسدلة نص قائمة

عمل جنب هذه الفقرة بيانات الاضافة:-
ـ تاريخ الإضافة قائمة منسدلة  ( هجري ميلادي) الساعة 
ـ اليوم قائمة منسدلة ( السبت الأحد الإثنين الثلاثاء الأربعاء الخميس الجمعة )
ـ اسم الموظف الذي اضاف : 

عمل جنب هذه الفقرة هذه الخيارات:-
ـ زر ( معاينة ) لعرض العمل والنظر فيه قبل تصديره الخيارات التي يتم تثبيتها ورقة × ورقة ـ أو ورقتين  × ورقة ـ أو ثلاث ورق  × ورقة ـ أو أربع ورق  × ورقة
ـ تصدير التقرير يكون ورقة × ورقة. ـ أو ورقتين  × ورقة ـ  أو ثلاث ورق  × ورقة ـ أو أربع ورق  × ورقة
ـ تصدير عمود أعمدة مختارة جميع الأعمدة قائمة منسدلة  (  Excel PDF )
ـ تصديرها لملف قائمة منسدلة  (  Excel PDF )
ـ تصدير للفصل الثاني ـ تصدير للعام القادم 
ـ استيراد من الفصل الأول استيراد من العام الماضي
ـ مشاركة قائمة منسدلة  (  Excel PDF صورة )
ـ الاحتفاظ بصور الاشعارات للرجوع لها إذا لزم

عمل جنب هذه الفقرة هذه الازرار
🔲 معاينة 🔲 تحرير 🔲 تعديل 🔲 إضافة 🔲 تصدير 🔲 حذف 🔲 فتحPDF 🔲 فتحExcel 🔲 طباعة 🔲 مشاركة
➖➖➖➖➖➖➖➖➖➖➖ فاصل بين الفقرات الفرعية ➖➖➖➖➖➖➖➖➖➖➖➖➖

🎄🎄🎄🎄🎄🎄🎄🎄🎄🎄🎄🎄🎄🎄 فاصل بين الفقرات الرئيسية 🎄🎄🎄🎄🎄🎄🎄🎄🎄🎄🎄🎄🎄

🔴 ٥ ـ الاختبارات
٥  ـ ١ ـ الاختبارات الشهرية
من يكتب هذه الفقرة	الموظف ومسؤل الصف أو من يكلفه المدير من الموظفين في حينه وموافقة المدير على الادخال أو التعديل
ولا يعدلها الموظف أو مسؤل الصف الا بموافقة المدير في حينه
من يشاهد هذه الفقرة	الموظف ومسؤل الصف كل واحد يشاهد عمله فقط
من يشاهد تقارير هذه الفقرة	المدير
من يصدر تقارير هذه الفقرة	المدير
خيارات البحث بهذه الفقرة يكون بواسطة 	مادة موظف فقرة كلمة جملة
التقارير المطلوبة من هذه الفقرة	تصدير تقرير بجدول الاختبارات لصف لصفوف مختارة لجميع الصفوف 
تصدير تقرير بالاختبارات المنفذة لمادة لمواد مختارة لجميع المواد 
تصدير تقرير بالاختبارات المنفذة لصف لصفوف لجميع الصفوف
تصدير تقرير بالاختبارات المنفذة لموظف لموظفين مختارين لجميع الموظفين 
تصدير تقرير بالاختبارات المنفذة لشهر لشهور مختارة لفصل لعام


١ـ مسلسل رقم متسلسل تلقائي
٢ـ اسم الموظف
٣ـ الجنس قائمة منسدلة  ( ذكر أنثى)
٤ـ الفصل الدراسي قائمة منسدلة  ( الأول الثاني)
٥ـ الشهر قائمة منسدلة  ( محرم صفر ربيع أول ربيع ثاني جماد أول جماد ثاني رجب شعبان)
٦ـ الأسبوع ( الأول  الثاني الثالث الرابع)  
٧ـ اليوم قائمة منسدلة ( السبت الأحد الإثنين الثلاثاء الأربعاء الخميس )
٨ـ التاريخ يكتب تلقائي  ( هجري ميلادي)
٩ـ الصف ( أول أساسي ثاني أساسي ثالث أساسي رابع أساسي خامس أساسي سادس أساسي سابع أساسي ثامن أساسي تاسع أساسي ـ أول ثانوي ـ ثاني ثانوي ـ ثالث ثانوي)
١٠ـ الشعبة قائمة منسدلة ( أ ب ج بدون)
١١ـ المادة قائمة منسدلة القرآن الإسلامية  اللغة العربية اللغة الانجليزية الرياضيات العلوم الاجتماعيات أخرى
١٢ـ الجزء قائمة منسدلة  ( ج١ ـ ج٢ ـ جزئين )
١٣ـ رقم السؤال قائمة منسدلة ( الأول الثاني الثالث الرابع الخامس السادس السابع الثامن أخرى )
١٤ـ فقرات السؤال قائمة منسدلة ( عدد متسلسل )
١٥ـ نص السؤال قائمة منسدلة نص قائمة
١٦ـ درجة السؤال ادخال يدوي قائمة منسدلة نص رقم
١٧ـ تم موافقة المدير على الاختبار قائمة منسدلة نعم لا

ملاحظة : ضف رابط الموقع حق كتابة الاختبار اللي كلمكني عليه


عمل جنب هذه الفقرة بيانات الاضافة:-
ـ تاريخ الإضافة قائمة منسدلة  ( هجري ميلادي) الساعة 
ـ اليوم قائمة منسدلة ( السبت الأحد الإثنين الثلاثاء الأربعاء الخميس الجمعة )
ـ اسم الموظف الذي اضاف : 

عمل جنب هذه الفقرة هذه الخيارات:-
ـ زر ( معاينة ) لعرض العمل والنظر فيه قبل تصديره الخيارات التي يتم تثبيتها ورقة × ورقة ـ أو ورقتين  × ورقة ـ أو ثلاث ورق  × ورقة ـ أو أربع ورق  × ورقة
ـ تصدير التقرير يكون ورقة × ورقة. ـ أو ورقتين  × ورقة ـ  أو ثلاث ورق  × ورقة ـ أو أربع ورق  × ورقة
ـ تصدير عمود أعمدة مختارة جميع الأعمدة قائمة منسدلة  (  Excel PDF )
ـ تصديرها لملف قائمة منسدلة  (  Excel PDF )
ـ تصدير للفصل الثاني ـ تصدير للعام القادم 
ـ استيراد من الفصل الأول استيراد من العام الماضي
ـ مشاركة قائمة منسدلة  (  Excel PDF صورة )
ـ الاحتفاظ بصور الاشعارات للرجوع لها إذا لزم

عمل جنب هذه الفقرة هذه الازرار
🔲 معاينة 🔲 تحرير 🔲 تعديل 🔲 إضافة 🔲 تصدير 🔲 حذف 🔲 فتحPDF 🔲 فتحExcel 🔲 طباعة 🔲 مشاركة
➖➖➖➖➖➖➖➖➖➖➖ فاصل بين الفقرات الفرعية ➖➖➖➖➖➖➖➖➖➖➖➖➖
٥  ـ ٢ ـ الامتحانات الفصلية 
من يكتب هذه الفقرة	الموظف ومسؤل الصف أو من يكلفه المدير من الموظفين في حينه وموافقة المدير على الادخال أو التعديل
ولا يعدلها الموظف أو مسؤل الصف الا بموافقة المدير في حينه
من يشاهد هذه الفقرة	الموظف ومسؤل الصف كل واحد يشاهد عمله فقط
من يشاهد تقارير هذه الفقرة	المدير
من يصدر تقارير هذه الفقرة	المدير
خيارات البحث بهذه الفقرة يكون بواسطة 	مادة موظف فقرة كلمة جملة
التقارير المطلوبة من هذه الفقرة	تصدير تقرير بجدول الامتحانات لصف لصفوف مختارة لجميع الصفوف 
تصدير تقرير بالامتحانات المنفذة لمادة لمواد مختارة لجميع المواد 
تصدير تقرير بالامتحانات المنفذة لصف لصفوف لجميع الصفوف
تصدير تقرير بالامتحانات المنفذة لموظف لموظفين مختارين لجميع الموظفين 
تصدير تقرير بالامتحانات المنفذة لفصل لعام





١ـ مسلسل رقم متسلسل تلقائي
٢ـ اسم الموظف
٣ـ الجنس قائمة منسدلة  ( ذكر أنثى)
٤ـ الفصل الدراسي قائمة منسدلة  ( الأول الثاني)
٥ـ الشهر قائمة منسدلة  ( محرم صفر ربيع أول ربيع ثاني جماد أول جماد ثاني رجب شعبان)
٦ـ الأسبوع ( الأول  الثاني الثالث الرابع)  
٧ـ اليوم قائمة منسدلة ( السبت الأحد الإثنين الثلاثاء الأربعاء الخميس )
٨ـ التاريخ يكتب تلقائي  ( هجري ميلادي)
٩ـ الصف ( أول أساسي ثاني أساسي ثالث أساسي رابع أساسي خامس أساسي سادس أساسي سابع أساسي ثامن أساسي تاسع أساسي ـ أول ثانوي ـ ثاني ثانوي ـ ثالث ثانوي)
١٠ـ الشعبة قائمة منسدلة ( أ ب ج بدون)
١١ـ المادة قائمة منسدلة القرآن الإسلامية  اللغة العربية اللغة الانجليزية الرياضيات العلوم الاجتماعيات أخرى
١٢ـ الجزء قائمة منسدلة  ( ج١ ـ ج٢ ـ جزئين )
١٣ـ رقم السؤال قائمة منسدلة ( الأول الثاني الثالث الرابع الخامس السادس السابع الثامن أخرى )
١٤ـ فقرات السؤال قائمة منسدلة ( عدد متسلسل )
١٥ـ نص السؤال قائمة منسدلة نص قائمة
١٦ـ درجة السؤال ادخال يدوي قائمة منسدلة نص رقم
١٧ـ تم موافقة المدير على الامتحان قائمة منسدلة نعم لا

ملاحظة : ضف رابط الموقع حق كتابة الاختبار اللي كلمكني عليه

عمل جنب هذه الفقرة بيانات الاضافة:-
ـ تاريخ الإضافة قائمة منسدلة  ( هجري ميلادي) الساعة 
ـ اليوم قائمة منسدلة ( السبت الأحد الإثنين الثلاثاء الأربعاء الخميس الجمعة )
ـ اسم الموظف الذي اضاف : 

عمل جنب هذه الفقرة هذه الخيارات:-
ـ زر ( معاينة ) لعرض العمل والنظر فيه قبل تصديره الخيارات التي يتم تثبيتها ورقة × ورقة ـ أو ورقتين  × ورقة ـ أو ثلاث ورق  × ورقة ـ أو أربع ورق  × ورقة
ـ تصدير التقرير يكون ورقة × ورقة. ـ أو ورقتين  × ورقة ـ  أو ثلاث ورق  × ورقة ـ أو أربع ورق  × ورقة
ـ تصدير عمود أعمدة مختارة جميع الأعمدة قائمة منسدلة  (  Excel PDF )
ـ تصديرها لملف قائمة منسدلة  (  Excel PDF )
ـ تصدير للفصل الثاني ـ تصدير للعام القادم 
ـ استيراد من الفصل الأول استيراد من العام الماضي
ـ مشاركة قائمة منسدلة  (  Excel PDF صورة )
ـ الاحتفاظ بصور الاشعارات للرجوع لها إذا لزم

عمل جنب هذه الفقرة هذه الازرار
🔲 معاينة 🔲 تحرير 🔲 تعديل 🔲 إضافة 🔲 تصدير 🔲 حذف 🔲 فتحPDF 🔲 فتحExcel 🔲 طباعة 🔲 مشاركة
➖➖➖➖➖➖➖➖➖➖➖ فاصل بين الفقرات الفرعية ➖➖➖➖➖➖➖➖➖➖➖➖➖
٥  ـ ٣ ـ توزيع الطلاب والطالبات على لجان الامتحان

من يكتب هذه الفقرة	تلقائي
من يشاهد هذه الفقرة	المدير ومن يسمح له المدير في حينه
من يشاهد تقارير هذه الفقرة	المدير
من يصدر تقارير هذه الفقرة	المدير
خيارات البحث بهذه الفقرة يكون بواسطة 	اسم فقرة كلمة جملة
التقارير المطلوبة من هذه الفقرة	تصدير تقرير بتوزيع الطلاب والطالبات على لجان الامتحان
تصدير تقرير لجنة رقم

١ـ مسلسل رقم متسلسل تلقائي
٢ـ الاسم يكتب تلقائي
٣ـ الجنس قائمة منسدلة  ( ذكر أنثى)
٤ـ الفصل الدراسي قائمة منسدلة  ( الأول الثاني)
٥ـ الصف ( أول أساسي ثاني أساسي ثالث أساسي رابع أساسي خامس أساسي سادس أساسي سابع أساسي ثامن أساسي تاسع أساسي ـ أول ثانوي ـ ثاني ثانوي ـ ثالث ثانوي)
٦ـ الشعبة قائمة منسدلة ( أ ب ج بدون)
٧ـ رقم اللجنة رقم متسلسل تلقائي
٨ـ المدرسة  قائمة منسدلة  ( القديمة الجديدة)
٩ـ الدور قائمة منسدلة  ( الأول الثاني الثالث الرابع أخرى)
١٠ـ رقم الفصل رقم متسلسل تلقائي
١١ـ رقم الاطار رقم متسلسل تلقائي
١٢ـ جنس الاطار قائمة منسدلة  (ذكور إناث)
١٢ـ المقعد رقم متسلسل تلقائي
١٣ـ مكانه بالمقعد قائمة منسدلة ( يمين وسط شمال)

👈عمل جنب هذه الفقرة مايلي:-
تصدير لعام ـ تصدير لفصل ـ استرداد من عام استرداد من فصل ـ استرداد الفقرة كاملة ـ استرداد فقرة مختارة ـ استرداد جميع الفقرات

عمل جنب هذه الفقرة بيانات الاضافة:-
ـ تاريخ الإضافة قائمة منسدلة  ( هجري ميلادي) الساعة 
ـ اليوم قائمة منسدلة ( السبت الأحد الإثنين الثلاثاء الأربعاء الخميس الجمعة )
ـ اسم الموظف الذي اضاف : 

عمل جنب هذه الفقرة هذه الخيارات:-
ـ زر ( معاينة ) لعرض العمل والنظر فيه قبل تصديره الخيارات التي يتم تثبيتها ورقة × ورقة ـ أو ورقتين  × ورقة ـ أو ثلاث ورق  × ورقة ـ أو أربع ورق  × ورقة
ـ تصدير التقرير يكون ورقة × ورقة. ـ أو ورقتين  × ورقة ـ  أو ثلاث ورق  × ورقة ـ أو أربع ورق  × ورقة
ـ تصدير عمود أعمدة مختارة جميع الأعمدة قائمة منسدلة  (  Excel PDF )
ـ تصديرها لملف قائمة منسدلة  (  Excel PDF )
ـ تصدير للفصل الثاني ـ تصدير للعام القادم 
ـ استيراد من الفصل الأول استيراد من العام الماضي
ـ مشاركة قائمة منسدلة  (  Excel PDF صورة )
ـ الاحتفاظ بصور الاشعارات للرجوع لها إذا لزم

عمل جنب هذه الفقرة هذه الازرار
🔲 معاينة 🔲 تحرير 🔲 تعديل 🔲 إضافة 🔲 تصدير 🔲 حذف 🔲 فتحPDF 🔲 فتحExcel 🔲 طباعة 🔲 مشاركة
🎄🎄🎄🎄🎄🎄🎄🎄🎄🎄🎄🎄🎄🎄 فاصل بين الفقرات الرئيسية 🎄🎄🎄🎄🎄🎄🎄🎄🎄🎄🎄🎄🎄

🔴 ٦ ـ الدرجات
٦  ـ ١ ـ المحصلات الشهرية
من يكتب هذه الفقرة	الموظف ومسؤل الصف أو من يكلفه المدير من الموظفين في حينه وموافقة المدير على الادخال أو التعديل
ولا يعدلها الموظف أو مسؤل الصف الا بموافقة المدير في حينه
من يشاهد هذه الفقرة	الموظف ومسؤل الصف كل واحد يشاهد عمله فقط
من يشاهد تقارير هذه الفقرة	المدير
من يصدر تقارير هذه الفقرة	المدير
خيارات البحث بهذه الفقرة يكون بواسطة 	مادة موظف فقرة كلمة جملة
التقارير المطلوبة من هذه الفقرة	
تصدير تقرير بالدرجات لطالب لطلاب مختارين لجميع الطلاب
تصدير تقرير بالدرجات لمادة لمواد مختارة لجميع المواد 
تصدير تقرير بالدرجات لصف لصفوف لجميع الصفوف
تصدير تقرير بالدرجات لشهر لشهور مختارة لفصل لعام

تصدير تقرير بالناجحين لطالب لطلاب مختارين لجميع الطلاب
تصدير تقرير بالناجحين لمادة لمواد مختارة لجميع المواد 
تصدير تقرير بالناجحين لصف لصفوف لجميع الصفوف
تصدير تقرير بالناجحين لشهر لشهور مختارة لفصل لعام
تصدير تقرير بالأوائل لصف لصفوف لجميع الصفوف

تصدير تقرير بالراسبين لطالب لطلاب مختارين لجميع الطلاب
تصدير تقرير بالراسبين لمادة لمواد مختارة لجميع المواد 
تصدير تقرير بالراسبين لصف لصفوف لجميع الصفوف
تصدير تقرير بالراسبين لشهر لشهور مختارة لفصل لعام

تصدير تقرير بالغياب لطالب لطلاب مختارين لجميع الطلاب
تصدير تقرير بالغياب لمادة لمواد مختارة لجميع المواد 
تصدير تقرير بالغياب لصف لصفوف لجميع الصفوف
تصدير تقرير بالغياب لشهر لشهور مختارة لفصل لعام

تصدير تقرير بنسبة الناجحين والراسبين والغياب لصف لصفوف مختارة لجميع الصفوف
تصدير تقرير بنسبة الناجحين والراسبين والغياب لمادة لمواد مختارة لجميع المواد
تصدير تقرير بنسبة الناجحين والراسبين والغياب لشهر لشهور مختارة لفصل لعام


١ـ مسلسل رقم متسلسل تلقائي
٢ـ الاسم يكون يكتب تلقائي من بيانات الطلاب
٣ـ الجنس قائمة منسدلة  ( ذكر أنثى)
٤ـ الفصل الدراسي قائمة منسدلة  ( الأول الثاني)
٥ـ الشهر قائمة منسدلة  ( محرم صفر ربيع أول ربيع ثاني جماد أول جماد ثاني رجب شعبان) 
٦ـ اليوم قائمة منسدلة ( السبت الأحد الإثنين الثلاثاء الأربعاء الخميس )
٧ـ التاريخ يكتب تلقائي  ( هجري ميلادي)
٨ـ الصف ( أول أساسي ثاني أساسي ثالث أساسي رابع أساسي خامس أساسي سادس أساسي سابع أساسي ثامن أساسي تاسع أساسي ـ أول ثانوي ـ ثاني ثانوي ـ ثالث ثانوي)
٩ـ الشعبة قائمة منسدلة ( أ ب ج بدون)
١٠ـ المادة قائمة منسدلة القرآن الإسلامية  اللغة العربية اللغة الانجليزية الرياضيات العلوم الاجتماعيات أخرى
ملاحظة : البيانات فصل ......... شهر ......... صف .....مادة .......عام .... تكون تكتب تلقائي
وهذا شكل الكشف وشكل التقرير
م	الاسم	فصل ......... شهر ......... صف .....مادة .......عام ١٤٤٧ ﻫ ٢٠٢٥ ـ ٢٠٢٦ م
		مواظبة	واجب	نشاط	اختبار 	مساهمة	المجموع
		٦	٣	٣	٤	٤	
							
							

ملاحظة كشوفات درجات الأشهر يكون الموقع يصدرها تلقائي


الدرجات الشهرية للفصل الدراسي الأول يقصد بها المحصلة الأولى
الدرجات الشهرية للفصل الدراسي الثاني يقصد بها المحصلة الثانية
١ـ مسلسل رقم متسلسل تلقائي
٢ـ الاسم يكون يكتب تلقائي من بيانات الطلاب
٣ـ الجنس قائمة منسدلة  ( ذكر أنثى)
٤ـ الفصل الدراسي قائمة منسدلة  ( الأول الثاني)
٥ـ الشهر قائمة منسدلة  ( محرم صفر ربيع أول ربيع ثاني جماد أول جماد ثاني رجب شعبان) 
٦ـ المادة قائمة منسدلة القرآن الإسلامية  اللغة العربية اللغة الانجليزية الرياضيات العلوم الاجتماعيات السلوك أخرى
٧ـ الصف ( أول أساسي ثاني أساسي ثالث أساسي رابع أساسي خامس أساسي سادس أساسي سابع أساسي ثامن أساسي تاسع أساسي ـ أول ثانوي ـ ثاني ثانوي ـ ثالث ثانوي)
٨ـ الشعبة قائمة منسدلة ( أ ب ج بدون)
٩ـ الجوانب تثبيت أكثر من خمسة ١ ـ ..... ٢ـ  ..... ٣ـ  ..... ٤ـ  ..... ٥ـ  ..... مثلا مواظبة واجب نشاط اختبار مساهمة أخرى )
١٠ـ درجة كل جانب ادخال
١١ـ جهة استيراد درجة كل جانب 
١٢ـ المجموع يكتب تلقائي وفقا لما كتب بكل جانب مع امكانية التعديل إذا لزم
١٣ـ النسبة ( ٢٠ ) تكتب تلقائي من خلال المجموع 

ملاحظات إذا كانت النسبة أقل من عشرة تعني راسب ويظلل المربع باللون الاحمر تلقائي لتوضيح انه راسب
وهكذا لجميع الصفوف وجميع المواد

ما سبق في حالة تم الاختبار لشهر واحد فقط
أما في حالة تم الاختبار لشهرين يتم قسمة ٤٠ ÷ ٢= ٢٠وتنزل تلقائي في المحصلة الأولى لانه الدرجة المقررة للمحصلة ٢٠
واما في حالة تم الاختبار لثلاثة اشهر يتم قسمة ٦٠÷ ٣= ٢٠ وتنزل في المحصلة الأولى
وأما في حالة تم الاختبار للأربعة الشهور يتم قسمة ٨٠÷٤= ٢٠ وتنزل في المحصلة الأولى

ـــــــــــــــــــــــــــــــــــــــــــ

👈 ملاحظات مستجدة حول تثبيت الدرجات الشهرية
٦= مواظبة وتكتب تلقائي وتفاصيلها ٢٠ يوم ما تأخر و٢٠ يوم ما غاب و٢٠ يوم ما انصرف بالراحة المجموع ٦٠÷ ١٠= ٦  مع امكانية التعديل إذا لزم 
ملاحظة ايام العطل خلوها تحسب لصالح الطالب
٣= واجب ادخال يدوي
٣= نشاط تكتب تلقائي من خلال الوسائل أكثر واحد يمنح ٣ وبعده أقل وهكذا وصفر للي ما جاب وسيلة مع امكانية التعديل إذا لزم
٤= اختبار ادخال يدوي
٤= مساهمة وتكتب تلقائي وتفاصيلها يمنح ٤ من سلم حق الشهر كامل ويمنح ٢ لمن سلم نصف مبلغ الشهر ويمنح ١ لمن سلم ربع مبلغ الشهر ويمنح صفر لمن لم يسلم مبلغ الشهر مع امكانية التعديل إذا لزم


درجة مادة السلوك شهريا
٢٠= تكتب تلقائي وتفاصيلها ٢٠ يوم تأدب و٢٠ يوم ما ازعج و٢٠ يوم ما عمل مشكلة المجموع ٦٠÷٣= ٢٠
وتكتب من حالات الأدب والإزعاج والمشكلات



ملاحظة : درجة مادة السلوك تكون تنزل تلقائي لكل طالب 


٢٠= بالاشهر المحصلة الأولى
٣٠= بالفصل الأول
٢٠= بالاشهر المحصلة الثانية
٣٠= بالفصل الثاني
١٠٠= المجموع
هذا تكون تكتب تلقائي لكل طالب وطالبة مع امكانية التعديل إذا لزم

سؤال في حالة وضع لطالب غ سواء في درجات الشهر او المحصلة او الفصلية كيف شيكون يجمع الموقع
هل سيجمع تلقائي بوجود غ 


عمل جنب هذه الفقرة بيانات الاضافة:-
ـ تاريخ الإضافة قائمة منسدلة  ( هجري ميلادي) الساعة 
ـ اليوم قائمة منسدلة ( السبت الأحد الإثنين الثلاثاء الأربعاء الخميس الجمعة )
ـ اسم الموظف الذي اضاف : 

عمل جنب هذه الفقرة هذه الخيارات:-
ـ زر ( معاينة ) لعرض العمل والنظر فيه قبل تصديره الخيارات التي يتم تثبيتها ورقة × ورقة ـ أو ورقتين  × ورقة ـ أو ثلاث ورق  × ورقة ـ أو أربع ورق  × ورقة
ـ تصدير التقرير يكون ورقة × ورقة. ـ أو ورقتين  × ورقة ـ  أو ثلاث ورق  × ورقة ـ أو أربع ورق  × ورقة
ـ تصدير عمود أعمدة مختارة جميع الأعمدة قائمة منسدلة  (  Excel PDF )
ـ تصديرها لملف قائمة منسدلة  (  Excel PDF )
ـ تصدير للفصل الثاني ـ تصدير للعام القادم 
ـ استيراد من الفصل الأول استيراد من العام الماضي
ـ مشاركة قائمة منسدلة  (  Excel PDF صورة )
ـ الاحتفاظ بصور الاشعارات للرجوع لها إذا لزم

عمل جنب هذه الفقرة هذه الازرار
🔲 معاينة 🔲 تحرير 🔲 تعديل 🔲 إضافة 🔲 تصدير 🔲 حذف 🔲 فتحPDF 🔲 فتحExcel 🔲 طباعة 🔲 مشاركة
➖➖➖➖➖➖➖➖➖➖➖ فاصل بين الفقرات الفرعية ➖➖➖➖➖➖➖➖➖➖➖➖➖
٦  ـ ٢ ـ درجة الفصل الأول

من يكتب هذه الفقرة	الموظف ومسؤل الصف أو من يكلفه المدير من الموظفين في حينه وموافقة المدير على الادخال أو التعديل
ولا يعدلها الموظف أو مسؤل الصف الا بموافقة المدير في حينه
من يشاهد هذه الفقرة	الموظف ومسؤل الصف كل واحد يشاهد عمله فقط
من يشاهد تقارير هذه الفقرة	المدير
من يصدر تقارير هذه الفقرة	المدير
خيارات البحث بهذه الفقرة يكون بواسطة 	مادة موظف فقرة كلمة جملة
التقارير المطلوبة من هذه الفقرة	تصدير تقرير بالدرجات لطالب لطلاب مختارين لجميع الطلاب
تصدير تقرير بالدرجات لمادة لمواد مختارة لجميع المواد 
تصدير تقرير بالدرجات لصف لصفوف لجميع الصفوف
تصدير تقرير بالدرجات لفصل لعام

تصدير تقرير بالناجحين لطالب لطلاب مختارين لجميع الطلاب
تصدير تقرير بالناجحين لمادة لمواد مختارة لجميع المواد 
تصدير تقرير بالناجحين لصف لصفوف لجميع الصفوف
تصدير تقرير بالناجحين لفصل لعام
تصدير تقرير بالتراتيب العشرة لصف لصفوف مختارة لجميع الصفوف
تصدير تقرير بالتراتيب العشرة على مستوى المدرسة

تصدير تقرير بالراسبين لطالب لطلاب مختارين لجميع الطلاب
تصدير تقرير بالراسبين لمادة لمواد مختارة لجميع المواد 
تصدير تقرير بالراسبين لصف لصفوف لجميع الصفوف
تصدير تقرير بالراسبين لفصل لعام

تصدير تقرير بالغياب لطالب لطلاب مختارين لجميع الطلاب
تصدير تقرير بالغياب لمادة لمواد مختارة لجميع المواد 
تصدير تقرير بالغياب لصف لصفوف لجميع الصفوف
تصدير تقرير بالغياب لفصل لعام

تصدير تقرير بنسبة الناجحين والراسبين والغياب لصف لصفوف مختارة لجميع الصفوف
تصدير تقرير بنسبة الناجحين والراسبين والغياب لمادة لمواد مختارة لجميع المواد
تصدير تقرير بنسبة الناجحين والراسبين والغياب لفصل لعام





١ـ مسلسل رقم متسلسل تلقائي
٢ـ الاسم يكون يكتب تلقائي من بيانات الطلاب
٣ـ الجنس قائمة منسدلة  ( ذكر أنثى)
٤ـ الفصل الدراسي قائمة منسدلة  ( الأول الثاني)
٦ـ اليوم قائمة منسدلة ( السبت الأحد الإثنين الثلاثاء الأربعاء الخميس )
٧ـ التاريخ يكتب تلقائي  ( هجري ميلادي)
٨ـ الصف ( أول أساسي ثاني أساسي ثالث أساسي رابع أساسي خامس أساسي سادس أساسي سابع أساسي ثامن أساسي تاسع أساسي ـ أول ثانوي ـ ثاني ثانوي ـ ثالث ثانوي)
٩ـ الشعبة قائمة منسدلة ( أ ب ج بدون)
١٠ـ المادة قائمة منسدلة القرآن الإسلامية  اللغة العربية اللغة الانجليزية الرياضيات العلوم الاجتماعيات أخرى
ملاحظة : البيانات فصل ......... شهر ......... صف .....مادة .......عام .... تكون تكتب تلقائي
وهذا شكل الكشف وشكل التقرير
م	الاسم	فصل ......... صف .....مادة .......عام ١٤٤٧ ﻫ ٢٠٢٥ ـ ٢٠٢٦ م
		المحصلة الأولى	الفصل الدراسي الأول	المجموع
		٢٠	٣٠	٥٠
				
				

مجموع الفصل يكون يكتب تلقائي
درجة المحصلة الأولى تكون تكتب تلقائي وفقا لما كتب في مجموع الشهر مع امكانية التعديل إذا لزم
إذا كان المجموع أقل من ٢٥ تعني راسب ويظلل المربع باللون الاحمر تلقائي لتوضيح انه راسب

بخصوص المجموع الكلي للمواد
المجموع الكلي 	النسبة	النتيجة	الترتيب على مستوى الصف	الترتيب على مستوى المدرسة
				

وفي الكشوفات يكتب 
المجموع الكلي للمواد يكون يكتب تلقائي ويكون محمي من التعديل
النسبة تكون تكتب تلقائي
النتيجة تكون تكتب تلقائي ناجح ناجحة راسب راسبة
الترتيب يكون يكتب تلقائي لجميع الطلاب على مستوى الصف
الترتيب يكون يكتب تلقائي لجميع الطلاب على مستوى المدرسة

ملاحظة كشوفات درجات الفصل الأول يكون الموقع يصدرها تلقائي
👈 تثبيت درجات الفصل الأول بالكشوفات
٢٠= المحصلة الأولى
٣٠= درجة الفصل الأول
٥٠= المجموع

درجة مادة السلوك تكون تنزل تلقائي لكل طالب 
٢٠= بالاشهر المحصلة الأولى
٣٠= بالفصل الأول
٢٠= بالاشهر المحصلة الثانية
٣٠= بالفصل الثاني
١٠٠= المجموع
هذا تكون تكتب تلقائي لكل طالب وطالبة مع امكانية التعديل إذا لزم

عمل جنب هذه الفقرة بيانات الاضافة:-
ـ تاريخ الإضافة قائمة منسدلة  ( هجري ميلادي) الساعة 
ـ اليوم قائمة منسدلة ( السبت الأحد الإثنين الثلاثاء الأربعاء الخميس الجمعة )
ـ اسم الموظف الذي اضاف : 

عمل جنب هذه الفقرة هذه الخيارات:-
ـ زر ( معاينة ) لعرض العمل والنظر فيه قبل تصديره الخيارات التي يتم تثبيتها ورقة × ورقة ـ أو ورقتين  × ورقة ـ أو ثلاث ورق  × ورقة ـ أو أربع ورق  × ورقة
ـ تصدير التقرير يكون ورقة × ورقة. ـ أو ورقتين  × ورقة ـ  أو ثلاث ورق  × ورقة ـ أو أربع ورق  × ورقة
ـ تصدير عمود أعمدة مختارة جميع الأعمدة قائمة منسدلة  (  Excel PDF )
ـ تصديرها لملف قائمة منسدلة  (  Excel PDF )
ـ تصدير للفصل الثاني ـ تصدير للعام القادم 
ـ استيراد من الفصل الأول استيراد من العام الماضي
ـ مشاركة قائمة منسدلة  (  Excel PDF صورة )
ـ الاحتفاظ بصور الاشعارات للرجوع لها إذا لزم

عمل جنب هذه الفقرة هذه الازرار
🔲 معاينة 🔲 تحرير 🔲 تعديل 🔲 إضافة 🔲 تصدير 🔲 حذف 🔲 فتحPDF 🔲 فتحExcel 🔲 طباعة 🔲 مشاركة
➖➖➖➖➖➖➖➖➖➖➖ فاصل بين الفقرات الفرعية ➖➖➖➖➖➖➖➖➖➖➖➖➖
٦  ـ ٣ ـ درجة الفصل الثاني

من يكتب هذه الفقرة	الموظف ومسؤل الصف أو من يكلفه المدير من الموظفين في حينه وموافقة المدير على الادخال أو التعديل
ولا يعدلها الموظف أو مسؤل الصف الا بموافقة المدير في حينه
من يشاهد هذه الفقرة	الموظف ومسؤل الصف كل واحد يشاهد عمله فقط
من يشاهد تقارير هذه الفقرة	المدير
من يصدر تقارير هذه الفقرة	المدير
التقارير المطلوبة من هذه الفقرة	تصدير تقرير بالدرجات لطالب لطلاب مختارين لجميع الطلاب
تصدير تقرير بالدرجات لمادة لمواد مختارة لجميع المواد 
تصدير تقرير بالدرجات لصف لصفوف لجميع الصفوف
تصدير تقرير بالدرجات لفصل لعام

تصدير تقرير بالناجحين لطالب لطلاب مختارين لجميع الطلاب
تصدير تقرير بالناجحين لمادة لمواد مختارة لجميع المواد 
تصدير تقرير بالناجحين لصف لصفوف لجميع الصفوف
تصدير تقرير بالناجحين لفصل لعام
تصدير تقرير بالتراتيب العشرة لصف لصفوف مختارة لجميع الصفوف
تصدير تقرير بالتراتيب العشرة على مستوى المدرسة

تصدير تقرير بالراسبين لطالب لطلاب مختارين لجميع الطلاب
تصدير تقرير بالراسبين لمادة لمواد مختارة لجميع المواد 
تصدير تقرير بالراسبين لصف لصفوف لجميع الصفوف
تصدير تقرير بالراسبين لفصل لعام

تصدير تقرير بالغياب لطالب لطلاب مختارين لجميع الطلاب
تصدير تقرير بالغياب لمادة لمواد مختارة لجميع المواد 
تصدير تقرير بالغياب لصف لصفوف لجميع الصفوف
تصدير تقرير بالغياب لفصل لعام

تصدير تقرير بنسبة الناجحين والراسبين والغياب لصف لصفوف مختارة لجميع الصفوف
تصدير تقرير بنسبة الناجحين والراسبين والغياب لمادة لمواد مختارة لجميع المواد
تصدير تقرير بنسبة الناجحين والراسبين والغياب لفصل لعام
خيارات البحث بهذه الفقرة يكون بواسطة 	مادة موظف فقرة كلمة جملة

١ـ مسلسل رقم متسلسل تلقائي
٢ـ الاسم يكون يكتب تلقائي من بيانات الطلاب
٣ـ الجنس قائمة منسدلة  ( ذكر أنثى)
٤ـ الفصل الدراسي قائمة منسدلة  ( الأول الثاني)
٦ـ اليوم قائمة منسدلة ( السبت الأحد الإثنين الثلاثاء الأربعاء الخميس )
٧ـ التاريخ يكتب تلقائي  ( هجري ميلادي)
٨ـ الصف ( أول أساسي ثاني أساسي ثالث أساسي رابع أساسي خامس أساسي سادس أساسي سابع أساسي ثامن أساسي تاسع أساسي ـ أول ثانوي ـ ثاني ثانوي ـ ثالث ثانوي)
٩ـ الشعبة قائمة منسدلة ( أ ب ج بدون)
١٠ـ المادة قائمة منسدلة القرآن الإسلامية  اللغة العربية اللغة الانجليزية الرياضيات العلوم الاجتماعيات أخرى
ملاحظة : البيانات فصل ......... شهر ......... صف .....مادة .......عام .... تكون تكتب تلقائي
وهذا شكل الكشف وشكل التقرير
م	الاسم	فصل ......... صف .....مادة .......عام ١٤٤٧ ﻫ ٢٠٢٥ ـ ٢٠٢٦ م
		المحصلة الأولى	الفصل الدراسي الأول	المحصلة
الثانية	الفصل الدراسي الثاني	المجموع
		٢٠	٣٠	٢٠	٣٠	١٠٠
						
						

مجموع الفصل يكون يكتب تلقائي
درجة المحصلة الأولى تكون تكتب تلقائي وفقا لما كتب في مجموع الشهر مع امكانية التعديل إذا لزم
إذا كان المجموع أقل من ٥٠ تعني راسب ويظلل المربع باللون الاحمر تلقائي لتوضيح انه راسب

بخصوص المجموع الكلي للمواد
المجموع الكلي 	النسبة	النتيجة	الترتيب على مستوى الصف	الترتيب على مستوى المدرسة
				

وفي الكشوفات يكتب 
المجموع الكلي للمواد يكون يكتب تلقائي ويكون محمي من التعديل
النسبة تكون تكتب تلقائي
النتيجة تكون تكتب تلقائي ناجح ناجحة راسب راسبة
الترتيب يكون يكتب تلقائي لجميع الطلاب على مستوى الصف
الترتيب يكون يكتب تلقائي لجميع الطلاب على مستوى المدرسة

ملاحظة كشوفات درجات الفصل الثاني يكون الموقع يصدرها تلقائي
👈 تثبيت درجات الفصل الثاني بالكشوفات
٢٠= المحصلة الأولى
٣٠= درجة الفصل الأول
٢٠= المحصلة الثانية
٣٠= درجة الفصل الثاني
١٠٠= المجموع

👈 ملاحظات عامة على الدرجات
ـ أن يكون عند تحرير الدرجة كيون ادخل (رقم نص) حتى نتمكن من وضع ( غ ) لمن تغيب
ـ في حالة ادخال درجة أكبر من الدرجة المستحقة. يكون الموقع لا يقبلها و يظهر اشعار 
الدرجة أكبر من المستحقة
مثلا الدرجة المقررة ٢٠ والموظف ادخل ٢٥ هنا يكون الموقع لا يقبلها يظهر اشعار 
الدرجة أكبر من المستحقة
ـ مجموع الدرجة سوأء الشهرية أو الفصل الأول او الثاني
هذا المجموع شيكون يطلع تلقائي وعليه يتم ضبطه بعدم تعدله

درجة مادة السلوك تكون تنزل تلقائي لكل طالب 
٢٠= بالاشهر المحصلة الأولى
٣٠= بالفصل الأول
٢٠= بالاشهر المحصلة الثانية
٣٠= بالفصل الثاني
١٠٠= المجموع
هذا تكون تكتب تلقائي لكل طالب وطالبة مع امكانية التعديل إذا لزم

عمل جنب هذه الفقرة بيانات الاضافة:-
ـ تاريخ الإضافة قائمة منسدلة  ( هجري ميلادي) الساعة 
ـ اليوم قائمة منسدلة ( السبت الأحد الإثنين الثلاثاء الأربعاء الخميس الجمعة )
ـ اسم الموظف الذي اضاف : 

عمل جنب هذه الفقرة هذه الخيارات:-
ـ زر ( معاينة ) لعرض العمل والنظر فيه قبل تصديره الخيارات التي يتم تثبيتها ورقة × ورقة ـ أو ورقتين  × ورقة ـ أو ثلاث ورق  × ورقة ـ أو أربع ورق  × ورقة
ـ تصدير التقرير يكون ورقة × ورقة. ـ أو ورقتين  × ورقة ـ  أو ثلاث ورق  × ورقة ـ أو أربع ورق  × ورقة
ـ تصدير عمود أعمدة مختارة جميع الأعمدة قائمة منسدلة  (  Excel PDF )
ـ تصديرها لملف قائمة منسدلة  (  Excel PDF )
ـ تصدير للفصل الثاني ـ تصدير للعام القادم 
ـ استيراد من الفصل الأول استيراد من العام الماضي
ـ مشاركة قائمة منسدلة  (  Excel PDF صورة )
ـ الاحتفاظ بصور الاشعارات للرجوع لها إذا لزم

عمل جنب هذه الفقرة هذه الازرار
🔲 معاينة 🔲 تحرير 🔲 تعديل 🔲 إضافة 🔲 تصدير 🔲 حذف 🔲 فتحPDF 🔲 فتحExcel 🔲 طباعة 🔲 مشاركة
🎄🎄🎄🎄🎄🎄🎄🎄🎄🎄🎄🎄🎄🎄 فاصل بين الفقرات الرئيسية 🎄🎄🎄🎄🎄🎄🎄🎄🎄🎄🎄🎄🎄

🔴 ٧ ـ النتائج
٧  ـ ١ ـ نتائج الفصل الأول
من يكتب هذه الفقرة	تلقائي
من يشاهد هذه الفقرة	المدير
من يشاهد تقارير هذه الفقرة	المدير
من يصدر تقارير هذه الفقرة	المدير
خيارات البحث بهذه الفقرة يكون بواسطة 	طالب طالبة فقرة كلمة جملة
التقارير المطلوبة من هذه الفقرة	تصدير تقرير بنتائج لصف لصفوف مختارة لجميع الصفوف لفصل
تصدير تقرير بنتائج لطالب لطلاب مختارين لجميع الطلاب لفصل

👈 وهذا شكل النتيجة

الجمهورية اليمنية
وزارة التربية والتعليم
مكتب التربية والتعليم بمحافظة إب
الإدارة التعليمية بمديرية العدين
مدرسة: التعاون الأساسية	نتيجة الفصل الدراسي الأول
للعام ١٤٤٧ ﻫ ٢٠٢٥ ـ ٢٠٢٦ م	


الاسم		الصف	

المادة	الدرجة	القرآن	الإسلامية	اللغة العربية	اللغة الانجليزية	رياضيات	علوم	اجتماعيات	السلوك
المحصلة الأولى	٢٠								
الفصل الدراسي الأول	٣٠								
مجموع المادة	٥٠								

المجموع الكلي		النسبة		النتيجة			مدير المدرسة
أ / عبدالقادر أحمد مرشد مثنى
الترتيب	على مستوى الصف		على مستوى المدرسة		ت / 

الاسم يكون يكتب تلقائي من بيانات الطلاب
الصف يكون يكتب تلقائي من بيانات الطلاب
درجة المحصلة الأولى تكون تكتب تلقائي وفقا لما كتب في مجموع الشهر مع امكانية التعديل إذا لزم
درجة الفصل الدراسي الأول يكون تكتب تلقائي وفقا لما كتب في درجة الفصل الأول مع امكانية التعديل إذا لزم
المجموع الكلي يكون تكتب تلقائي 
النسبة تكون تكتب تلقائي
النتيجة تكون تكتب تلقائي ناجح ناجحة راسب راسبة
الترتيب يكون يكتب تلقائي لجميع الطلاب والطالبات على مستوى الصف
الترتيب يكون يكتب تلقائي لجميع الطلاب والطالبات على مستوى المدرسة
إذا كان المجموع أقل من ٢٥ تعني راسب ويظلل المربع باللون الاحمر تلقائي بالنتيجة لتوضيح انه راسب 

ملاحظة للتراتيب
في حال حصول اثنين على ترتيب مثلا الترتيب الثالث يكون الموقع يرتبهم حسب الأسم ويكتب الاول الترتيب الثالث ويكتب الثاني الترتيب الثالث مكرر مع امكانية التعديل إن لزم

ملاحظة يكون يصدر نتيجة ×ورقة ـ نتيجتين × ورقة ـ ٣ نتائج × ورقة ـ ٤ نتائج ×ورقة  

عمل جنب هذه الفقرة بيانات الاضافة:-
ـ تاريخ الإضافة قائمة منسدلة  ( هجري ميلادي) الساعة 
ـ اليوم قائمة منسدلة ( السبت الأحد الإثنين الثلاثاء الأربعاء الخميس الجمعة )
ـ اسم الموظف الذي اضاف : 

عمل جنب هذه الفقرة هذه الخيارات:-
ـ زر ( معاينة ) لعرض العمل والنظر فيه قبل تصديره الخيارات التي يتم تثبيتها ورقة × ورقة ـ أو ورقتين  × ورقة ـ أو ثلاث ورق  × ورقة ـ أو أربع ورق  × ورقة
ـ تصدير التقرير يكون ورقة × ورقة. ـ أو ورقتين  × ورقة ـ  أو ثلاث ورق  × ورقة ـ أو أربع ورق  × ورقة
ـ تصدير عمود أعمدة مختارة جميع الأعمدة قائمة منسدلة  (  Excel PDF )
ـ تصديرها لملف قائمة منسدلة  (  Excel PDF )
ـ تصدير للفصل الثاني ـ تصدير للعام القادم 
ـ استيراد من الفصل الأول استيراد من العام الماضي
ـ مشاركة قائمة منسدلة  (  Excel PDF صورة )
ـ الاحتفاظ بصور الاشعارات للرجوع لها إذا لزم

عمل جنب هذه الفقرة هذه الازرار
🔲 معاينة 🔲 تحرير 🔲 تعديل 🔲 إضافة 🔲 تصدير 🔲 حذف 🔲 فتحPDF 🔲 فتحExcel 🔲 طباعة 🔲 مشاركة
➖➖➖➖➖➖➖➖➖➖➖ فاصل بين الفقرات الفرعية ➖➖➖➖➖➖➖➖➖➖➖➖➖
٧  ـ ٢ ـ نتائج الفصل الثاني

من يكتب هذه الفقرة	تلقائي
من يشاهد هذه الفقرة	المدير
من يشاهد تقارير هذه الفقرة	المدير
من يصدر تقارير هذه الفقرة	المدير
خيارات البحث بهذه الفقرة يكون بواسطة 	طالب طالبة فقرة كلمة جملة
التقارير المطلوبة من هذه الفقرة	تصدير تقرير بنتائج لصف لصفوف مختارة لجميع الصفوف لفصل
تصدير تقرير بنتائج لطالب لطلاب مختارين لجميع الطلاب لفصل

👈 وهذا شكل النتيجة

الجمهورية اليمنية
وزارة التربية والتعليم
مكتب التربية والتعليم بمحافظة إب
الإدارة التعليمية بمديرية العدين
مدرسة: التعاون الأساسية	نتيجة الفصل الدراسي الثاني
للعام ١٤٤٧ ﻫ ٢٠٢٥ ـ ٢٠٢٦ م	


الاسم		الصف	

المادة	الدرجة	القرآن	الإسلامية	اللغة العربية	اللغة الانجليزية	رياضيات	علوم	اجتماعيات	السلوك
المحصلة الأولى	٢٠								
الفصل الدراسي الأول	٣٠								
المحصلة الثانية	٢٠								
الفصل الدراسي الثاني	٣٠								
مجموع المادة	١٠٠								

المجموع الكلي		النسبة		النتيجة			مدير المدرسة
أ / عبدالقادر أحمد مرشد مثنى
الترتيب	على مستوى الصف		على مستوى المدرسة		ت / 


الاسم يكون يكتب تلقائي من بيانات الطلاب
الصف يكون يكتب تلقائي من بيانات الطلاب
درجة المحصلة الأولى تكون تكتب تلقائي وفقا لما كتب في الفصل الدراسي الأول مع امكانية التعديل إذا لزم
درجة الفصل الدراسي الأول يكون تكتب تلقائي وفقا لما كتب في درجة الفصل الأول مع امكانية التعديل إذا لزم
درجة المحصلة الثانية تكون تكتب تلقائي وفقا لما كتب في مجموع الشهر للفصل الثاني مع امكانية التعديل إذا لزم
درجة الفصل الدراسي الثاني تكون تكتب تلقائي وفقا لما كتب في مجموع درجة الفصل الأول والثاني مع امكانية التعديل إذا لزم

المجموع الكلي يكون تكتب تلقائي 
النسبة تكون تكتب تلقائي
النتيجة تكون تكتب تلقائي ناجح ناجحة راسب راسبة
الترتيب يكون يكتب تلقائي لجميع الطلاب والطالبات على مستوى الصف
الترتيب يكون يكتب تلقائي لجميع الطلاب والطالبات على مستوى المدرسة
إذا كان المجموع أقل من ٥٠ تعني راسب ويظلل المربع باللون الاحمر تلقائي بالنتيجة لتوضيح انه راسب 

ملاحظة للتراتيب
في حال حصول اثنين على ترتيب مثلا الترتيب الثالث يكون الموقع يرتبهم حسب الأسم ويكتب الاول الترتيب الثالث والثاني الترتيب الثالث مكرر مع امكانية التعديل إن لزم

ملاحظة يكون يصدر نتيجة ×ورقة ـ نتيجتين × ورقة ـ ٣ نتائج × ورقة ـ ٤ نتائج ×ورقة  

عمل جنب هذه الفقرة بيانات الاضافة:-
ـ تاريخ الإضافة قائمة منسدلة  ( هجري ميلادي) الساعة 
ـ اليوم قائمة منسدلة ( السبت الأحد الإثنين الثلاثاء الأربعاء الخميس الجمعة )
ـ اسم الموظف الذي اضاف : 

عمل جنب هذه الفقرة هذه الخيارات:-
ـ زر ( معاينة ) لعرض العمل والنظر فيه قبل تصديره الخيارات التي يتم تثبيتها ورقة × ورقة ـ أو ورقتين  × ورقة ـ أو ثلاث ورق  × ورقة ـ أو أربع ورق  × ورقة
ـ تصدير التقرير يكون ورقة × ورقة. ـ أو ورقتين  × ورقة ـ  أو ثلاث ورق  × ورقة ـ أو أربع ورق  × ورقة
ـ تصدير عمود أعمدة مختارة جميع الأعمدة قائمة منسدلة  (  Excel PDF )
ـ تصديرها لملف قائمة منسدلة  (  Excel PDF )
ـ تصدير للفصل الثاني ـ تصدير للعام القادم 
ـ استيراد من الفصل الأول استيراد من العام الماضي
ـ مشاركة قائمة منسدلة  (  Excel PDF صورة )
ـ الاحتفاظ بصور الاشعارات للرجوع لها إذا لزم

عمل جنب هذه الفقرة هذه الازرار
🔲 معاينة 🔲 تحرير 🔲 تعديل 🔲 إضافة 🔲 تصدير 🔲 حذف 🔲 فتحPDF 🔲 فتحExcel 🔲 طباعة 🔲 مشاركة
➖➖➖➖➖➖➖➖➖➖➖ فاصل بين الفقرات الفرعية ➖➖➖➖➖➖➖➖➖➖➖➖➖
٧  ـ ٣ ـ الشهائد
ملاحظة هذه الفقرة عندك  
🎄🎄🎄🎄🎄🎄🎄🎄🎄🎄🎄🎄🎄🎄 فاصل بين الفقرات الرئيسية 🎄🎄🎄🎄🎄🎄🎄🎄🎄🎄🎄🎄🎄

🔴 ٨ ـ الزائرين

من يكتب هذه الفقرة	المدير أو من يكلفه المدير من الموظفين في حينه وموافقة المدير على الادخال أو التعديل
من يشاهد هذه الفقرة	المدير
من يشاهد تقارير هذه الفقرة	المدير
من يصدر تقارير هذه الفقرة	المدير
خيارات البحث بهذه الفقرة يكون بواسطة 	اسم فقرة كلمة جملة
التقارير المطلوبة من هذه الفقرة	تصدير تقرير بالزائرين المختصين ليوم أسبوع لشهر لفصل لعام
تصدير تقرير بالآباء الزائرين ليوم أسبوع لشهر لفصل لعام
تصدير تقرير بالأمهات الزائرات ليوم أسبوع لشهر لفصل لعام
تصدير تقرير بجميع الزائرين ليوم أسبوع لشهر لفصل لعام

١ـ مسلسل رقم متسلسل تلقائي
٢ـ اسم الزائر : ادخال نص
٣ـ الجنس: قائمة منسدلة  ( ذكر أنثى)
٤ـ جهة الزائر قائمة منسدلة  ( جهة مختصة ـ إب ـ العدين ـ المكتب ـ طلاب ـ آباء ـ أمهات ـ أخرى)
٥ـ الصفة قائمة منسدلة  ( نص قائمة)
٦ـ الفصل الدراسي قائمة منسدلة  ( الأول الثاني)
٧ـ الشهر قائمة منسدلة  ( محرم صفر ربيع أول ربيع ثاني جماد أول جماد ثاني رجب شعبان)
٨ـ الأسبوع ( الأول  الثاني الثالث الرابع)  
٩ـ اليوم قائمة منسدلة ( السبت الأحد الإثنين الثلاثاء الأربعاء الخميس )
١٠ـ التاريخ يكتب تلقائي  ( هجري ميلادي)
١١ـ الغرض من الزيارة قائمة منسدلة  ( نص قائمة)
١٢ـ سبب زيارته قائمة منسدلة  ( نص قائمة)
١٣ـ وثق زيارته بسجل المدرسة قائمة منسدلة  ( نعم لا)
١٤ـ صور الوثائق الواردة منه ـ صورة تقرير زيارته ادخال من ملف
١٥ـ ملاحظات ادخال يدوي قائمة منسدلة نص قائمة


عمل جنب هذه الفقرة بيانات الاضافة:-
ـ تاريخ الإضافة قائمة منسدلة  ( هجري ميلادي) الساعة 
ـ اليوم قائمة منسدلة ( السبت الأحد الإثنين الثلاثاء الأربعاء الخميس الجمعة )
ـ اسم الموظف الذي اضاف : 

عمل جنب هذه الفقرة هذه الخيارات:-
ـ زر ( معاينة ) لعرض العمل والنظر فيه قبل تصديره الخيارات التي يتم تثبيتها ورقة × ورقة ـ أو ورقتين  × ورقة ـ أو ثلاث ورق  × ورقة ـ أو أربع ورق  × ورقة
ـ تصدير التقرير يكون ورقة × ورقة. ـ أو ورقتين  × ورقة ـ  أو ثلاث ورق  × ورقة ـ أو أربع ورق  × ورقة
ـ تصدير عمود أعمدة مختارة جميع الأعمدة قائمة منسدلة  (  Excel PDF )
ـ تصديرها لملف قائمة منسدلة  (  Excel PDF )
ـ تصدير للفصل الثاني ـ تصدير للعام القادم 
ـ استيراد من الفصل الأول استيراد من العام الماضي
ـ مشاركة قائمة منسدلة  (  Excel PDF صورة )
ـ الاحتفاظ بصور الاشعارات للرجوع لها إذا لزم

عمل جنب هذه الفقرة هذه الازرار
🔲 معاينة 🔲 تحرير 🔲 تعديل 🔲 إضافة 🔲 تصدير 🔲 حذف 🔲 فتحPDF 🔲 فتحExcel 🔲 طباعة 🔲 مشاركة
🎄🎄🎄🎄🎄🎄🎄🎄🎄🎄🎄🎄🎄🎄 فاصل بين الفقرات الرئيسية 🎄🎄🎄🎄🎄🎄🎄🎄🎄🎄🎄🎄🎄


🔴 ٩ ـ المشكلات
من يكتب هذه الفقرة	المدير أو من يكلفه المدير من الموظفين في حينه وموافقة المدير على الادخال أو التعديل
من يشاهد هذه الفقرة	المدير
من يشاهد تقارير هذه الفقرة	المدير
من يصدر تقارير هذه الفقرة	المدير
خيارات البحث بهذه الفقرة يكون بواسطة 	اسم فقرة كلمة جملة
التقارير المطلوبة من هذه الفقرة	تصدير تقرير مشكلات موظف ـ موظفين مختارين ـ جميع الموظفين  
تصدير تقرير مشكلات الموظفين ليوم أسبوع شهر لفصل لعام

تصدير تقرير بمشكلات الطلاب اليومية الأسبوعية الشهرية لصف لصفوف مختارة لجميع الصفوف
تصدير تقرير مشكلات طالب ـ طلاب مختارين ـ جميع الطلاب 
تصدير تقرير مشكلات لصف لصفوف مختارة لجميع الصفوف

تصدير تقرير مشكلات أب آباء مختارين ـ جميع الآباء 
تصدير تقرير بمشكلات الآباء ليوم أسبوع شهر لفصل لعام

تصدير تقرير مشكلات أم أمهات مختارات ـ جميع الأمهات 
تصدير تقرير بمشكلات الأمهات ليوم أسبوع شهر لفصل لعام

تصدير تقرير بجميع المشكلات ليوم أسبوع شهر لفصل لعام

١ـ مسلسل رقم متسلسل تلقائي
٢ـ الاسم نص
٣ـ الجنس: قائمة منسدلة  ( ذكر أنثى) 
٤ـ جهة المشكلة قائمة منسدلة  ( موظفين ـ طلاب ـ آباء ـ أمهات ـ أخرى)
٥ـ الفصل الدراسي قائمة منسدلة  ( الأول الثاني)
٦ـ الشهر قائمة منسدلة  ( محرم صفر ربيع أول ربيع ثاني جماد أول جماد ثاني رجب شعبان)
٧ـ الأسبوع ( الأول  الثاني الثالث الرابع) 
٨ـ الصف ( أول أساسي ثاني أساسي ثالث أساسي رابع أساسي خامس أساسي سادس أساسي سابع أساسي ثامن أساسي تاسع أساسي ـ أول ثانوي ـ ثاني ثانوي ـ ثالث ثانوي)
٩ـ الشعبة قائمة منسدلة ( أ ب ج بدون)
١٠ـ اليوم قائمة منسدلة ( السبت الأحد الإثنين الثلاثاء الأربعاء الخميس الجمعة)
١١ـ التاريخ يكتب تلقائي  ( هجري ميلادي)
١٢ـ رقم المشكلة يكتب تلقائي في حالة كونها اول مشكلة يكون الرقم ١ وفي حالة كونها الثانية يكتب الرقم ٢
١٣ـ ( اسم أسماء )من تشاكل معه (نص قائمة) لا يوجد
١٤ـ عنوان المشكلة (نص قائمة)
١٥ـ الشهود (نص قائمة) لا يوجد
١٦ـ الاجراءات المنفذة (نص قائمة)
١٧ـ حرر محضرها قائمة منسدلة ( نعم لا )
١٨ـ صورة المحضر ادراج من ملف
١٩ـ ملاحظات ادخال يدوي قائمة منسدلة نص قائمة

عمل جنب هذه الفقرة بيانات الاضافة:-
ـ تاريخ الإضافة قائمة منسدلة  ( هجري ميلادي) الساعة 
ـ اليوم قائمة منسدلة ( السبت الأحد الإثنين الثلاثاء الأربعاء الخميس الجمعة )
ـ اسم الموظف الذي اضاف : 

عمل جنب هذه الفقرة هذه الخيارات:-
ـ زر ( معاينة ) لعرض العمل والنظر فيه قبل تصديره الخيارات التي يتم تثبيتها ورقة × ورقة ـ أو ورقتين  × ورقة ـ أو ثلاث ورق  × ورقة ـ أو أربع ورق  × ورقة
ـ تصدير التقرير يكون ورقة × ورقة. ـ أو ورقتين  × ورقة ـ  أو ثلاث ورق  × ورقة ـ أو أربع ورق  × ورقة
ـ تصدير عمود أعمدة مختارة جميع الأعمدة قائمة منسدلة  (  Excel PDF )
ـ تصديرها لملف قائمة منسدلة  (  Excel PDF )
ـ تصدير للفصل الثاني ـ تصدير للعام القادم 
ـ استيراد من الفصل الأول استيراد من العام الماضي
ـ مشاركة قائمة منسدلة  (  Excel PDF صورة )
ـ الاحتفاظ بصور الاشعارات للرجوع لها إذا لزم

عمل جنب هذه الفقرة هذه الازرار
🔲 معاينة 🔲 تحرير 🔲 تعديل 🔲 إضافة 🔲 تصدير 🔲 حذف 🔲 فتحPDF 🔲 فتحExcel 🔲 طباعة 🔲 مشاركة
🎄🎄🎄🎄🎄🎄🎄🎄🎄🎄🎄🎄🎄🎄 فاصل بين الفقرات الرئيسية 🎄🎄🎄🎄🎄🎄🎄🎄🎄🎄🎄🎄🎄

🔴  ١٠  ـ ١ ـ مجلس الآباء
من يكتب هذه الفقرة	المدير أو من يكلفه المدير من الموظفين في حينه وموافقة المدير على الادخال أو التعديل
من يشاهد هذه الفقرة	المدير
من يشاهد تقارير هذه الفقرة	المدير
من يصدر تقارير هذه الفقرة	المدير
اضافة لهذه الفقرة	تصدير لعام ـ استرداد من عام
خيارات البحث بهذه الفقرة يكون بواسطة 	اسم فقرة كلمة جملة
التقارير المطلوبة من هذه الفقرة	تصدير تقرير بأسماء أعضاء مجلس الآباء 
تصدير تقرير بأعمال مجلس الآباء 
تصدير تقرير بالأعمال التي قام بها خلال العام

👈 بيانات رئيس مجلس الآباء 
الاسم : ادخال نص
الصفة : ادخال نص
عام تعينه : مثلا ١٤٤٧ ﻫ ٢٠٢٥ ـ ٢٠٢٦ م
👈 أعضاء مجلس الآباء الاسم الصفة ملاحظات 
ملاحظة ادخال أكثر من عشرون اسم
م	الاسم	الصفة	ملاحظات
			
			

👈 أعمال مجلس الآباء ملاحظة : يكون يدخل الاعمال بلا حد لها
م	العمل	ملاحظات
		
		

👈 الأعمال التي قام بها خلال العام ملاحظة : يكون يدخل الاعمال بلا حد لها
م	العمل	 اليوم	 التاريخ	ملاحظات
				
				

👈 وثائق خاصة بمجلس الآباء : الخيارات التي يتم تثبيتها ( فيديو صور ملف نص ملف صوتي أخرى )

👈عمل جنب هذه الفقرة مايلي:-
تصدير لعام ـ تصدير لفصل ـ استرداد من عام استرداد من فصل ـ استرداد الفقرة كاملة ـ استرداد فقرة مختارة ـ استرداد جميع الفقرات

عمل جنب هذه الفقرة بيانات الاضافة:-
ـ تاريخ الإضافة قائمة منسدلة  ( هجري ميلادي) الساعة 
ـ اليوم قائمة منسدلة ( السبت الأحد الإثنين الثلاثاء الأربعاء الخميس الجمعة )
ـ اسم الموظف الذي اضاف : 

عمل جنب هذه الفقرة هذه الخيارات:-
ـ زر ( معاينة ) لعرض العمل والنظر فيه قبل تصديره الخيارات التي يتم تثبيتها ورقة × ورقة ـ أو ورقتين  × ورقة ـ أو ثلاث ورق  × ورقة ـ أو أربع ورق  × ورقة
ـ تصدير التقرير يكون ورقة × ورقة. ـ أو ورقتين  × ورقة ـ  أو ثلاث ورق  × ورقة ـ أو أربع ورق  × ورقة
ـ تصدير عمود أعمدة مختارة جميع الأعمدة قائمة منسدلة  (  Excel PDF )
ـ تصديرها لملف قائمة منسدلة  (  Excel PDF )
ـ تصدير للفصل الثاني ـ تصدير للعام القادم 
ـ استيراد من الفصل الأول استيراد من العام الماضي
ـ مشاركة قائمة منسدلة  (  Excel PDF صورة )
ـ الاحتفاظ بصور الاشعارات للرجوع لها إذا لزم

عمل جنب هذه الفقرة هذه الازرار
🔲 معاينة 🔲 تحرير 🔲 تعديل 🔲 إضافة 🔲 تصدير 🔲 حذف 🔲 فتحPDF 🔲 فتحExcel 🔲 طباعة 🔲 مشاركة
🎄🎄🎄🎄🎄🎄🎄🎄🎄🎄🎄🎄🎄🎄 فاصل بين الفقرات الرئيسية 🎄🎄🎄🎄🎄🎄🎄🎄🎄🎄🎄🎄🎄

🔴 ١١  ـ ١ ـ النظافة المدرسية

من يكتب هذه الفقرة	المدير أو من يكلفه المدير من الموظفين في حينه وموافقة المدير على الادخال أو التعديل
من يشاهد هذه الفقرة	الموظف اطلاع فقط دون تعديل
من يشاهد تقارير هذه الفقرة	المدير
من يصدر تقارير هذه الفقرة	المدير
اضافة لهذه الفقرة	تصدير لعام ـ استرداد من عام
خيارات البحث بهذه الفقرة يكون بواسطة 	اسم فقرة كلمة جملة
التقارير المطلوبة من هذه الفقرة	تصدير تقرير مخالفة موظف موظفين مختارين جميع الموظفين ويكون ليوم أسبوع شهر فصل لعام
تصدير تقرير لمرافق المدرسة القديمة الجديدة

مسؤل النظافة العامة بالمدرسة الموظف تلقائي 
مشرف النظافة بالمدرسة القديمة الموظف تلقائي
مشرف النظافة بالمدرسة الجديدة الموظف تلقائي

مرافق المدرسة القديمة الخيارات التي يتم تثبيتها
نظافة داخل الفصل رقم ١ دور ١
نظافة ممر الدور الأول
نظافة ساحة المدرسة القديمة فقط
نظافة داخل الفصل رقم ٢ دور ١
نظافة خلف المدرسة القديمة
نظافة داخل الفصل رقم ١ دور ٢
نظافة داخل الفصل رقم ٢ دور ٢
نظافة داخل الفصل رقم ٣ دور ٢
نظافة داخل الفصل رقم ٤ دور ٣ 
نظافة ممر الدور ٣ مع التربيعة
نظافة ممر الدور ٢ والدرج وتحت الدرج إلى عند البوابة
نظافة سطوح المدرسة القديمة 
أخرى

الموظف المكلف يكتب تلقائي 
القائم بالتنظيف الخيارات التي يتم تثبيتها طلاب ـ طالبات الصف الخيارات التي يتم تثبيتها من ١ ل ٣/ث 
مخالف نعم لا


مرافق المدرسة الجديدة الخيارات التي يتم تثبيتها
نظافة داخل الفصل رقم ١ دور ١
نظافة داخل الفصل رقم ٢ دور ١ 
نظافة داخل الفصل رقم ٣ دور ٢
نظافة ممر الدور الثاني مع التربيعة وتحت الدرج
حوش المدرسة من الخلف
نظافة ممر الدور الأول مع درج المنصة
نظافة داخل إدارة المدرسة
نظافة ساحة المدرسة الجديدة والتركيز على أمام المنصة
نظافة مدخل المدرسة مع أمام الدكان للخلف
احراق القمامة في السائلة والتخلص منها
نظافة سطوح المدرسة الجديدة
أخرى 

نظافة دورة المياه 
الأيام الخيارات التي يتم تثبيتها ( السبت الأحد الإثنين ـ الثلاثاء الأربعاء الخميس )
المشرف الخيارات التي يتم تثبيتها الموظف اخرى مخالف نعم لا
نظافة سطح دورة المياه

ـ مسلسل رقم متسلسل تلقائي
ـ اسم الموظف
ـ الفصل الدراسي قائمة منسدلة  ( الأول الثاني)
ـ الشهر قائمة منسدلة  ( محرم صفر ربيع أول ربيع ثاني جماد أول جماد ثاني رجب شعبان)
ـ الأسبوع ( الأول  الثاني الثالث الرابع)  
ـ اليوم قائمة منسدلة ( السبت الأحد الإثنين الثلاثاء الأربعاء الخميس )
ـ التاريخ يكتب تلقائي  ( هجري ميلادي)
المدرسة قائمة منسدلة ( القديمة الجديدة
ـ الموقع :
ـ القائم بالتنظيف الخيارات التي يتم تثبيتها طلاب طالبات الصف الخيارات التي يتم تثبيتها من ١ ل ٣/ث ـ الشعبة قائمة منسدلة ( أ ب ج بدون)
مخالف نعم لا

👈عمل جنب هذه الفقرة مايلي:-
تصدير لعام ـ تصدير لفصل ـ استرداد من عام استرداد من فصل ـ استرداد الفقرة كاملة ـ استرداد فقرة مختارة ـ استرداد جميع الفقرات

عمل جنب هذه الفقرة بيانات الاضافة:-
ـ تاريخ الإضافة قائمة منسدلة  ( هجري ميلادي) الساعة 
ـ اليوم قائمة منسدلة ( السبت الأحد الإثنين الثلاثاء الأربعاء الخميس الجمعة )
ـ اسم الموظف الذي اضاف : 

عمل جنب هذه الفقرة هذه الخيارات:-
ـ زر ( معاينة ) لعرض العمل والنظر فيه قبل تصديره الخيارات التي يتم تثبيتها ورقة × ورقة ـ أو ورقتين  × ورقة ـ أو ثلاث ورق  × ورقة ـ أو أربع ورق  × ورقة
ـ تصدير التقرير يكون ورقة × ورقة. ـ أو ورقتين  × ورقة ـ  أو ثلاث ورق  × ورقة ـ أو أربع ورق  × ورقة
ـ تصدير عمود أعمدة مختارة جميع الأعمدة قائمة منسدلة  (  Excel PDF )
ـ تصديرها لملف قائمة منسدلة  (  Excel PDF )
ـ تصدير للفصل الثاني ـ تصدير للعام القادم 
ـ استيراد من الفصل الأول استيراد من العام الماضي
ـ مشاركة قائمة منسدلة  (  Excel PDF صورة )
ـ الاحتفاظ بصور الاشعارات للرجوع لها إذا لزم

عمل جنب هذه الفقرة هذه الازرار
🔲 معاينة 🔲 تحرير 🔲 تعديل 🔲 إضافة 🔲 تصدير 🔲 حذف 🔲 فتحPDF 🔲 فتحExcel 🔲 طباعة 🔲 مشاركة
🎄🎄🎄🎄🎄🎄🎄🎄🎄🎄🎄🎄🎄🎄 فاصل بين الفقرات الرئيسية 🎄🎄🎄🎄🎄🎄🎄🎄🎄🎄🎄🎄🎄

🔴 ١٢  ـ ١ ـ الإذاعة الصباحية

من يكتب هذه الفقرة	المدير أو من يكلفه المدير من الموظفين في حينه وموافقة المدير على الادخال أو التعديل
من يشاهد هذه الفقرة	الموظف اطلاع فقط
من يشاهد تقارير هذه الفقرة	المدير
من يصدر تقارير هذه الفقرة	المدير
خيارات البحث بهذه الفقرة يكون بواسطة 	اسم فقرة كلمة جملة
التقارير المطلوبة من هذه الفقرة	تصدير تقرير جدول الاذاعة ليوم لأسبوع لشهر لفصل للعام كامل
تصدير تقرير عنوان الاذاعة ليوم لأسبوع لشهر لفصل للعام كامل
تصدير تقرير جدول الاذاعة لصف لصفوف مختارة لجميع الصفوف
تصدير تقرير عنوان الاذاعة لصف لصفوف مختارة لجميع الصفوف

١ـ الفصل الدراسي قائمة منسدلة  ( الأول الثاني)
٢ـ الشهر قائمة منسدلة  ( محرم صفر ربيع أول ربيع ثاني جماد أول جماد ثاني رجب شعبان)
٣ـ الأسبوع ( الأول  الثاني الثالث الرابع)  
٤ـ اليوم قائمة منسدلة ( السبت الأحد الإثنين الثلاثاء الأربعاء الخميس )
٥ـ التاريخ يكتب تلقائي  ( هجري ميلادي)
٦ـ المكلف بالقاء طابور الصباح قائمة منسدلة ( طلاب الصف ـ طالبات الصف ـ عام ـ أخرى
٧ـ الرآئد المربي اسم الموظف تلقائي قائمة منسدلة نص أخرى   
٨ـ المسؤل المدرب اسم الموظف تلقائي قائمة منسدلة نص أخرى   
٩ـ مسؤل تنظيم الطلاب اسم الموظف تلقائي قائمة منسدلة نص أخرى   
١٠ـ مسؤل تنظيم الطالبات اسم الموظف تلقائي قائمة منسدلة نص أخرى   
١١ـ مسؤل عقاب المتأخرين اسم الموظف تلقائي قائمة منسدلة نص أخرى   
١٢ـ مسؤل تقييم الطابور اسم الموظف تلقائي قائمة منسدلة نص أخرى   
١٣ـ أخرى اسم الموظف تلقائي قائمة منسدلة نص أخرى   
١٤ـ موضوع الإذاعة  ادخال يدوي ـ قائمة منسدلة نص بدون 



جدول الإذاعة الصباحية الشهرية
 للعام الدراسي.         ـ بمدرسة التعاون الأساسية
رقم 
أسبوع	الأيام	المكلف بإلقاء طابور الصباح	الرآئد
المربي   	المسؤل
المدرب
 	مسؤل تنظيم
الطلاب	مسؤل تنظيم الطالبات	مسؤل عقاب المتأخرين 	مسؤل تقييم الطابور 	أخرى
و
اسم المكلف

١	السبت	طلاب صف سادس أخرى							
	الأحد	طلاب صف ثاني أخرى							
	الإثنين	طالبات صف ثامن أخرى							
	الثلاثاء	طالبات صف رابع أخرى							
	الأربعاء	عام أخرى							
									

٢	السبت	طلاب صف سابع أخرى							
	الأحد	طلاب صف خامس أخرى							
	الإثنين	طالبات صف ثالث أخرى							
	الثلاثاء	طالبات صف أول أخرى							
	الأربعاء	عام أخرى							
									

٣	السبت	طلاب صف ثامن أخرى							
	الأحد	طلاب صف رابع أخرى							
	الإثنين	طالبات صف سادس أخرى							
	الثلاثاء	طالبات صف ثاني أخرى							
	الأربعاء	عام أخرى							
									

٤	السبت	طلاب صف ثالث أخرى							
	الأحد	طلاب صف أول أخرى							
	الإثنين	طالبات صف سابع أخرى							
	الثلاثاء	طالبات صف خامس أخرى							
	الأربعاء	عام أخرى							


 










































مواضيع الإذاعة الصباحية للعام كامل بمدرسة التعاون  

الأيام	
المكلف بالإذاعة	الفصل الدراسي الأول		الفصل الدراسي الثاني
		محرم	صفر	ربيع أول	ربيع ثاني		جماد أول	جماد ثاني	رجب	شعبان
										
السبت	طلاب صف سادس	ــــــــ	المذاكرة	الابداع والابتكار			ــــــــ	بر الوالدين	شكر المعروف	
الأحد	طلاب صف ثاني	ــــــــ	ــــــــ	ــــــــ			ــــــــ	ــــــــ	ــــــــ	
الإثنين	طالبات صف ثامن	ــــــــ	الاختبارات	الطموح			ــــــــ	الحياء	الاقتصاد	
الثلاثاء	طالبات صف رابع	ــــــــ	التفوق	اتقان العمل			ــــــــ	الرفق بالحيوان	صلة الأرحام	
										
السبت	طلاب صف سابع	الإخلاص	الإيمان	طاعة الرسول			التعاون	الشتاء	الايثار	
الأحد	طلاب صف خامس	العلم	الإحسان	الإقتداء بالرسول			الصحة والوقاية	الصبر	الكرم	
الإثنين	طالبات صف ثالث	ــــــــ	ــــــــ	ــــــــ			ــــــــ	ــــــــ	ــــــــ	
الثلاثاء	طالبات صف أول	ــــــــ	ــــــــ	ــــــــ			ــــــــ	ــــــــ	ــــــــ	
										
السبت	طلاب صف ثامن	الصلاة	المدرسة	الصدق			يوم الجمعة	الهمة	الاستقامة	
الأحد	طلاب صف رابع	النظافة	المعلم	الغيبة			احترام الكبير	التوكل	القناعة	
الإثنين	طالبات صف سادس	أهمية القراءة	التقوى	القرآن الكريم			آداب الأكل والشرب	الوفاء	التصدق	
الثلاثاء	طالبات صف ثاني	ــــــــ	ــــــــ	ــــــــ			ــــــــ	ــــــــ	ــــــــ	
										
السبت	طلاب صف ثالث	ــــــــ	ــــــــ	ــــــــ			ــــــــ	ــــــــ	ــــــــ	
الأحد	طلاب صف أول	ــــــــ	ــــــــ	ــــــــ			ــــــــ	ــــــــ	ــــــــ	
الإثنين	طالبات صف سابع	الأب	حسن الخلق	الوقت			التسامح	الشجاعة	رعاية الأشجار	
الثلاثاء	طالبات صف خامس	الأم	احترام الاخرين	النظام			الأمانة	الرحمة	حب الوطن	
										


👈عمل جنب هذه الفقرة مايلي:-
تصدير لعام ـ تصدير لفصل ـ استرداد من عام استرداد من فصل ـ استرداد الفقرة كاملة ـ استرداد فقرة مختارة ـ استرداد جميع الفقرات

عمل جنب هذه الفقرة بيانات الاضافة:-
ـ تاريخ الإضافة قائمة منسدلة  ( هجري ميلادي) الساعة 
ـ اليوم قائمة منسدلة ( السبت الأحد الإثنين الثلاثاء الأربعاء الخميس الجمعة )
ـ اسم الموظف الذي اضاف : 

عمل جنب هذه الفقرة هذه الخيارات:-
ـ زر ( معاينة ) لعرض العمل والنظر فيه قبل تصديره الخيارات التي يتم تثبيتها ورقة × ورقة ـ أو ورقتين  × ورقة ـ أو ثلاث ورق  × ورقة ـ أو أربع ورق  × ورقة
ـ تصدير التقرير يكون ورقة × ورقة. ـ أو ورقتين  × ورقة ـ  أو ثلاث ورق  × ورقة ـ أو أربع ورق  × ورقة
ـ تصدير عمود أعمدة مختارة جميع الأعمدة قائمة منسدلة  (  Excel PDF )
ـ تصديرها لملف قائمة منسدلة  (  Excel PDF )
ـ تصدير للفصل الثاني ـ تصدير للعام القادم 
ـ استيراد من الفصل الأول استيراد من العام الماضي
ـ مشاركة قائمة منسدلة  (  Excel PDF صورة )
ـ الاحتفاظ بصور الاشعارات للرجوع لها إذا لزم

عمل جنب هذه الفقرة هذه الازرار
🔲 معاينة 🔲 تحرير 🔲 تعديل 🔲 إضافة 🔲 تصدير 🔲 حذف 🔲 فتحPDF 🔲 فتحExcel 🔲 طباعة 🔲 مشاركة
🎄🎄🎄🎄🎄🎄🎄🎄🎄🎄🎄🎄🎄🎄 فاصل بين الفقرات الرئيسية 🎄🎄🎄🎄🎄🎄🎄🎄🎄🎄🎄🎄🎄


🔴 ١٣  ـ ١ ـ المسابقات المدرسية
من يكتب هذه الفقرة	المدير أو من يكلفه المدير من الموظفين في حينه وموافقة المدير على الادخال أو التعديل
من يشاهد هذه الفقرة	يطلع الموظف على فقرات منها 
ويمنع اطلاع الموظف على تراتيب أي مسابقة
من يشاهد تقارير هذه الفقرة	المدير
من يصدر تقارير هذه الفقرة	المدير
خيارات البحث بهذه الفقرة يكون بواسطة 	اسم فقرة كلمة جملة
التقارير المطلوبة من هذه الفقرة	اشكال التقارير الموجودة بالفقرة




ـ نوع المسابقة قائمة منسدلة ( مسابقة الطلاب والطالبات ـ مسابقة المعلمين والمعلمات ـ مسابقة الصفوف الدراسية ـ أخرى )
ـ فترة المسابقة قائمة منسدلة ( أسبوعية ـ شهرية ـ فصلية ـ أخرى
ـ الفصل الدراسي قائمة منسدلة  ( الأول الثاني)
ـ الشهر قائمة منسدلة  ( محرم صفر ربيع أول ربيع ثاني جماد أول جماد ثاني رجب شعبان)
ـ الأسبوع ( الأول  الثاني الثالث الرابع)  
ـ اليوم قائمة منسدلة ( السبت الأحد الإثنين الثلاثاء الأربعاء الخميس )
ـ التاريخ يكتب تلقائي  ( هجري ميلادي)

👈 اولا مسابقة الطلاب والطالبات
مسؤل تقييم المسابقة قائمة منسدلة ( مربي مربية رائد رائدة الصف) اخرى
نقاط تقييم المسابقة ملاحظة يكتبها المدير فقط ويطلع عليها الموظف اطلاع فقط
تراتيب المسابقة 

طلاب الصف الأول
الترتيب الأول 
الترتيب الثاني
الترتيب الثالث

طالبات الصف الأول
الترتيب الأول 
الترتيب الثاني
الترتيب الثالث

طلاب الصف الثاني
الترتيب الأول 
الترتيب الثاني
الترتيب الثالث

طالبات الصف الثاني
الترتيب الأول 
الترتيب الثاني
الترتيب الثالث

وهكذا إلى صف ٣/ ث
ملاحظة التراتيب يكتبها المدير فقط ولا يطلع عليها الموظف

شكل تقارير المسابقة

محضر تسليم جائزة مسابقة الطلاب/الطالبات الشهرية
             الجمهورية اليمنية                                                                                                    
وزارة التربية والتعليم.   
مكتب التربية والتعليم بمحافظة إب   
مكتب التربية والتعليم بمديرية العدين	رقم المسابقة	١	مدرسة : التعاون الأساسية
العام الدراسي : ١٤٤٧ ﻫ ٢٠٢٥  ـ ٢٠٢٦ م
مسؤل التقييم : مربي / رائد الصف
	الشهر	صفر	
	الفصل الدراسي	الأول	
        ==========================================================================
الصف	الترتيب	أسماء الطلاب	المبلغ	التوقيع		الترتيب	أسماء الطالبات	المبلغ	التوقيع

الأول	الأول	وجيه بندر محمد محمد عبدالله	١٥٠٠			الأول	أمر محمد عبدالقادر نعمان مسعد	١٥٠٠	
	الثاني	عبدالرحمن فؤاد محمد عبدالله سعيد 	١٠٠٠			الثاني	إيلاف صلاح أحمد مرشد مثنى	١٠٠٠	
	الثالث	أيمن مقبل حمود مثنى	٥٠٠			الثالث	وداد محمد حمود هزاع عثمان	٥٠٠	
									

الثاني	الأول	اياد محمد عبدالله سيف محمد	١٥٠٠			الأول	هدية محمد علي مرشد الأحمدي	١٥٠٠	
	الثاني	الياس ياسر احمد عبده مهدي	١٠٠٠			الثاني	ليان ياسر علي عبده محمد	١٠٠٠	
	الثالث	شهاب محمد عبدالكريم احمد	٥٠٠			الثالث	مودة فؤاد امين طه المساوى	٥٠٠	
									

الثالث	الأول	يونس أكرم محمد محسن زيد	١٥٠٠			الأول	روان ياسر علي عبده محمد	١٥٠٠	
	الثاني	سليمان عبدالعزيز أحمد مرشد أحمد	١٠٠٠			الثاني	أثير عبدالله أحمد مرشد الهمداني	١٠٠٠	
	الثالث	حمزة أمين مطيع همام الصيري	٥٠٠			الثالث	نوف محمد محمد أحمد سيف	٥٠٠	
									

الرابع	الأول	إبراهيم أحمد مهدي عبدالوهاب الجبرتي	١٥٠٠			الأول	ريماس محمد أحمد ملهي قاسم	١٥٠٠	
	الثاني	محمد مطهر عبدالعزيز سيف أحمد	١٠٠٠			الثاني	جنى محمدمحمدعلي مثنى	١٠٠٠	
	الثالث	مالك مطهر عبدالواحد محمد عبدالله	٥٠٠			الثالث	سونيا محمد أحمد شعلان	٥٠٠	
									

الخامس	الأول	عبد العالم منصور محمد طه شرف	١٥٠٠			الأول	حماس طه علي عبدالله أحمد	١٥٠٠	
	الثاني	جمال محمد حمود هزاع عثمان	١٠٠٠			الثاني	وفاء عبد العزيز سيف احمد عبده	١٠٠٠	
	الثالث	عوض محسن عوض محمد طه	٥٠٠			الثالث	رواية مصلح قاسم احمد محمد	٥٠٠	
									

السادس	الأول	محمد هاشم عوض محمد طه	١٥٠٠			الأول	رهف عبدالرحمن عبدالله عبده عنان	١٥٠٠	
	الثاني	خالد عبدالرحمن عبدالله عبده عنان	١٠٠٠			الثاني	زهور عبدالله سيف محمد عبده	١٠٠٠	
	الثالث	عبدالخالق عبدالله محمد مرشد محمد	٥٠٠			الثالث	جنئ فيصل عبدالله عبده عنان	٥٠٠	
									

السابع	الأول	عمار ياسر علي قايد	١٥٠٠			الأول	صفاء فيصل طه محمد أحمد	١٥٠٠	
	الثاني	أمير بلال قاسم عبدالله غالب	١٠٠٠			الثاني	فرح أحمد حسين منصور العواضي	١٠٠٠	
	الثالث	ريدان أنور أحمد مرشد مثنى	٥٠٠			الثالث	مراسيل راشد صادق مرشد أحمد	٥٠٠	
									

الثامن	الأول	ريان محمد أحمد سعيد	١٥٠٠			الأول	رقية محمد عبدالكريم أحمد عمر	١٥٠٠	
	الثاني	شاكر أمين عابد عامر مسعد	١٠٠٠			الثاني	داليا أحمد منصور محمد طه	١٠٠٠	
	الثالث	ريان فؤاد صادق أحمد سيف	٥٠٠			الثالث	شيماء محمد قاسم عبدالله عبدالعزيز	٥٠٠	
									
نحن مربين ورواد الفصول بمدرسة التعاون ومن خلال عملنا ودورنا بالفصل واطلاعنا على نقاط المسابقة الشهرية تم اختيارنا للمتفوقين والمتفوقات داخل الفصول لهذا الشهر ورفعهم إلى إدارة المدرسة وهم المذكورين أعلا هذا ومسؤلين عنهم أمام الله وعليه نوقع
    
م	أسماء المعلمين والمعلمات	الصف المكلف به	التوقيع
	سرور فيصل عبدالله عبده عنان	الأول	
	محمد عبدالقادر أحمد مرشد مثنى	الثاني	
	رفاق خالد أحمد مرشد مثنى	الثالث	
	رغد بشير نعمان مسعد راشد	الرابع	
	شيماء أحمد محمد صالح شبيطه	الخامس	
	رحمة صادق عبدالله علي العمري	السادس	
	سعاد عبد العزيز أحمد مرشد أحمد	السابع	
	رنا فيصل أحمد حميد شبيطه	الثامن	
                إدارة المدرسة
مدير المدرسة	القائم بأعمال المدير
الأستاذ / عبدالقادر أحمد مرشد مثنى	الأستاذ / خالد أحمد مرشد مثنى

	


توثيق المسابقة قائمة منسدلة ( فيديو صور ملف نص ملف صوتي أخرى )
 
 ــــــــــــــــــــــــــــــــــــــــــــــــــــــــــــــــــــــــــــــــــــــــــــــــــــــــ
ثانيا مسابقة المعلمين والمعلمات
مسؤل تقييم المسابقة قائمة منسدلة ( إدارة المدرسة اخرى)
نقاط تقييم المسابقة ملاحظة يكتبها المدير فقط ويطلع عليها الموظف اطلاع فقط
تراتيب المسابقة
الترتيب الأول ـ الترتيب الأول مكرر ـ الترتيب الأول مكرر 
الترتيب الثاني ـ الترتيب الثاني مكرر ـ الترتيب الثاني مكرر
الترتيب الثالث ـ الترتيب الثالث مكرر ـ الترتيب الثالث مكرر
الترتيب الرابع ـ الترتيب الرابع مكرر ـ الترتيب الرابع مكرر
الترتيب الخامس ـ الترتيب الخامس مكرر ـ الترتيب الخامس مكرر
الترتيب السادس ـ الترتيب السادس مكرر ـ الترتيب السادس مكرر
الترتيب السابع ـ الترتيب السابع مكرر ـ الترتيب السابع مكرر

ملاحظة التراتيب يكتبها المدير فقط ولا يطلع عليها الموظف
شكل تقارير المسابقة
محضر تسليم الجائزة الشهرية للمتفوقين والمتفوقات بمسابقة المعلمين والمعلمات بمدرسة التعاون الأساسية
لشهر صفر للعام الدراسي ١٤٤٧ ﻫ ٢٠٢٥ ـ ٢٠٢٦ م
 شهر صفر
الترتيب	الاسم	المبلغ	التوقيع
الأول	إيمان محمد أحمد سعد الجماعي	١١٠٠٠	
الثاني	رفاق خالد أحمد مرشد	٨٠٠٠	
الثالث	رنا فيصل أحمد حميد	٧٠٠٠	
الرابع	سرور فيصل عبدالله عبده	٦٠٠٠	
الخامس	سعاد عبد العزيز أحمد مرشد	٤٠٠٠	



         إدارة المدرسة
مدير المدرسة	القائم بأعمال المدير
الأستاذ
عبدالقادر أحمد مرشد مثنى	الأستاذ
خالد أحمد مرشد مثنى


	



توثيق المسابقة قائمة منسدلة ( فيديو صور ملف نص ملف صوتي أخرى )
➖➖➖➖➖➖➖➖➖➖➖ فاصل بين الفقرات الفرعية ➖➖➖➖➖➖➖➖➖➖➖➖➖
ثالثا مسابقة الصفوف الدراسية
مسؤل تقييم المسابقة قائمة منسدلة ( إدارة المدرسة اخرى)
نقاط تقييم المسابقة ملاحظة يكتبها المدير فقط ويطلع عليها الموظف اطلاع فقط
تراتيب المسابقة
الترتيب الأول ـ الترتيب الأول مكرر ـ الترتيب الأول مكرر 
الترتيب الثاني ـ الترتيب الثاني مكرر ـ الترتيب الثاني مكرر
الترتيب الثالث ـ الترتيب الثالث مكرر ـ الترتيب الثالث مكرر
الترتيب الرابع ـ الترتيب الرابع مكرر ـ الترتيب الرابع مكرر

ملاحظة التراتيب يكتبها المدير فقط ولا يطلع عليها الموظف

شكل تقارير المسابقة
محضر تسليم جائزة مسابقة الفصول الشهرية

ومن خلال عمل ودور إدارة المدرسة بالتقييم لشهر صفر 
للعام الدراسي ١٤٤٧ ﻫ ٢٠٢٥ ـ ٢٠٢٦ م تم تفوق الصفوف التالية:
الترتيب	١	٢	٣
الصف	الرابع	الثالث	الثاني

إدارة المدرسة
مدير المدرسة	القائم بأعمال المدير
الأستاذ / عبدالقادر أحمد مرشد مثنى	الأستاذ / خالد أحمد مرشد مثنى

	

نحن مربين ورواد الفصول بمدرسة التعاون المتفوقين صفوفنا داخل المدرسة لشهر ــــــــــــ 
وقد استلمنا الجائزة مع الإكرامية وشهادة التقدير لهذا الشهر من إدارة المدرسة وعليه نوقع

م	إسم المربي / الرائد 	الصف المكلف به	الترتيب	مبلغ الصف	اكرامية	شهادة	التوقيع
	رغد بشير نعمان مسعد	الرابع	الأول	٦٠٠٠	٤٠٠٠		
	رفاق خالد أحمد مرشد	الثالث	الثاني	٥٠٠٠	٣٠٠٠		
	محمد عبدالقادر أحمد مرشد	الثاني	الثالث	٤٠٠٠	٢٠٠٠		



توثيق المسابقة قائمة منسدلة ( فيديو صور ملف نص ملف صوتي أخرى )
 
عمل جنب هذه الفقرة بيانات الاضافة:-
ـ تاريخ الإضافة قائمة منسدلة  ( هجري ميلادي) الساعة 
ـ اليوم قائمة منسدلة ( السبت الأحد الإثنين الثلاثاء الأربعاء الخميس الجمعة )
ـ اسم الموظف الذي اضاف : 

عمل جنب هذه الفقرة هذه الخيارات:-
ـ زر ( معاينة ) لعرض العمل والنظر فيه قبل تصديره الخيارات التي يتم تثبيتها ورقة × ورقة ـ أو ورقتين  × ورقة ـ أو ثلاث ورق  × ورقة ـ أو أربع ورق  × ورقة
ـ تصدير التقرير يكون ورقة × ورقة. ـ أو ورقتين  × ورقة ـ  أو ثلاث ورق  × ورقة ـ أو أربع ورق  × ورقة
ـ تصدير عمود أعمدة مختارة جميع الأعمدة قائمة منسدلة  (  Excel PDF )
ـ تصديرها لملف قائمة منسدلة  (  Excel PDF )
ـ تصدير للفصل الثاني ـ تصدير للعام القادم 
ـ استيراد من الفصل الأول استيراد من العام الماضي
ـ مشاركة قائمة منسدلة  (  Excel PDF صورة )
ـ الاحتفاظ بصور الاشعارات للرجوع لها إذا لزم

عمل جنب هذه الفقرة هذه الازرار
🔲 معاينة 🔲 تحرير 🔲 تعديل 🔲 إضافة 🔲 تصدير 🔲 حذف 🔲 فتحPDF 🔲 فتحExcel 🔲 طباعة 🔲 مشاركة
🎄🎄🎄🎄🎄🎄🎄🎄🎄🎄🎄🎄🎄🎄 فاصل بين الفقرات الرئيسية 🎄🎄🎄🎄🎄🎄🎄🎄🎄🎄🎄🎄🎄
🔴 ١٤  ـ ١ ـ الاجتماعات المدرسية
من يكتب هذه الفقرة	المدير أو من يكلفه المدير من الموظفين في حينه وموافقة المدير على الادخال أو التعديل
من يشاهد هذه الفقرة	المدير
من يشاهد تقارير هذه الفقرة	المدير
من يصدر تقارير هذه الفقرة	المدير
اضافة لهذه الفقرة	استيراد من الفصل الأول ـ من العام الماضي
خيارات البحث بهذه الفقرة يكون بواسطة 	اسم فقرة كلمة جملة
التقارير المطلوبة من هذه الفقرة	تصدير تقرير الاجتماع ليوم أسبوع شهر فصل لعام
تصدير تقرير نقاط الاجتماع ليوم أسبوع شهر فصل لعام

١ـ الفصل الدراسي قائمة منسدلة  ( الأول الثاني)
٢ـ الشهر قائمة منسدلة  ( محرم صفر ربيع أول ربيع ثاني جماد أول جماد ثاني رجب شعبان)
٣ـ الأسبوع ( الأول  الثاني الثالث الرابع)  
٤ـ اليوم قائمة منسدلة ( السبت الأحد الإثنين الثلاثاء الأربعاء الخميس )
٥ـ التاريخ يكتب تلقائي  ( هجري ميلادي)
٦ـ الوقت
٧ـ نوع الاجتماع قائمة منسدلة ثابت دوري 
٨ـ موضوع الاجتماع قائمة منسدلة نص ـ بدون 
٩ـ مقر الاجتماع قائمة منسدلة إدارة المدرسة اخرى
١٠ـ نقاط الاجتماع مسلسل النقاط ملاحظات يكون يدخل نقاط بلا حد لها
م	النقاط	ملاحظات
		
		

١١ـ حضور وغياب اسم الموظف تلقائي 
١٢ـ العلامات متأخر حاضر غائب بعذر غائب بدون عذر
👈عمل جنب هذه الفقرة مايلي:-
تصدير لعام ـ تصدير لفصل ـ استرداد من عام استرداد من فصل ـ استرداد الفقرة كاملة ـ استرداد فقرة مختارة ـ استرداد جميع الفقرات

عمل جنب هذه الفقرة بيانات الاضافة:-
ـ تاريخ الإضافة قائمة منسدلة  ( هجري ميلادي) الساعة 
ـ اليوم قائمة منسدلة ( السبت الأحد الإثنين الثلاثاء الأربعاء الخميس الجمعة )
ـ اسم الموظف الذي اضاف : 

عمل جنب هذه الفقرة هذه الخيارات:-
ـ زر ( معاينة ) لعرض العمل والنظر فيه قبل تصديره الخيارات التي يتم تثبيتها ورقة × ورقة ـ أو ورقتين  × ورقة ـ أو ثلاث ورق  × ورقة ـ أو أربع ورق  × ورقة
ـ تصدير التقرير يكون ورقة × ورقة. ـ أو ورقتين  × ورقة ـ  أو ثلاث ورق  × ورقة ـ أو أربع ورق  × ورقة
ـ تصدير عمود أعمدة مختارة جميع الأعمدة قائمة منسدلة  (  Excel PDF )
ـ تصديرها لملف قائمة منسدلة  (  Excel PDF )
ـ تصدير للفصل الثاني ـ تصدير للعام القادم 
ـ استيراد من الفصل الأول استيراد من العام الماضي
ـ مشاركة قائمة منسدلة  (  Excel PDF صورة )
ـ الاحتفاظ بصور الاشعارات للرجوع لها إذا لزم

عمل جنب هذه الفقرة هذه الازرار
🔲 معاينة 🔲 تحرير 🔲 تعديل 🔲 إضافة 🔲 تصدير 🔲 حذف 🔲 فتحPDF 🔲 فتحExcel 🔲 طباعة 🔲 مشاركة
🎄🎄🎄🎄🎄🎄🎄🎄🎄🎄🎄🎄🎄🎄 فاصل بين الفقرات الرئيسية 🎄🎄🎄🎄🎄🎄🎄🎄🎄🎄🎄🎄🎄
🔴 ١٥  ـ ١ ـ الأنشطة المدرسية
من يكتب هذه الفقرة	المدير والموظف 
من يشاهد هذه الفقرة	المدير والموظف لنشاطه فقط ومن يسمح له المدير في حينه
من يشاهد تقارير هذه الفقرة	المدير
من يصدر تقارير هذه الفقرة	المدير
خيارات البحث بهذه الفقرة يكون بواسطة 	اسم نشاط فقرة كلمة جملة
التقارير المطلوبة من هذه الفقرة	تصدير تقرير بيانات مسؤل النشاط لموظف لموظفين مختارين لجميع الموظفين
تصدير تقرير بيانات المشاركين لنشاط واحد لأنشطة مختارة لجميع الأنشطة
تصدير تقرير بيانات المشاركين لصف واحد لصفوف مختارة لجميع الصفوف
تصدير تقرير بيانات النشاط لنشاط واحد لأنشطة مختارة لجميع الأنشطة
تصدير تقرير الأنشطة لفصل لعام

👈 بيانات مسؤل النشاط
١ـ اسم مسؤل النشاط :
٢ـ أسم النشاط المسؤل عنه : قائمة منسدلة  ( القرآن ـ الخط ـ الصحة ـ الرياضة ـ الكشافة ـ المسرح ـ الإنشاد ـ الرسم ـ الإلقاء ـالأشبال ـ  أخرى )
٣ـ العلامات :
حاضر	نعم		
	لا		
غائب 	نعم	بإذن ـ بدون إذن	خيارات بدون إذن (بعذر ـ بدون عذر)
	لا		

👈 بيانات المشارك
١ـ مسلسل رقم متسلسل تلقائي
٢ـ اسم الطالب / الطالبة يكتب تلقائي
٣ـ الجنس : قائمة منسدلة  ( ذكر أنثى)
٤ـ الصف : ( أول أساسي ثاني أساسي ثالث أساسي رابع أساسي خامس أساسي سادس أساسي سابع أساسي ثامن أساسي تاسع أساسي ـ أول ثانوي ـ ثاني ثانوي ـ ثالث ثانوي)
٥ـ الشعبة : قائمة منسدلة ( أ ب ج بدون)
٦ـ اليوم : قائمة منسدلة ( السبت الأحد الإثنين الثلاثاء الأربعاء الخميس )
٧ـ التاريخ : يكتب تلقائي  ( هجري ميلادي)
٨ـ أسم النشاط الملتحق به : قائمة منسدلة  ( القرآن ـ الخط ـ الصحة ـ الرياضة ـ الكشافة ـ المسرح ـ الإنشاد ـ الرسم ـ الإلقاء  ـالأشبال ـ  أخرى )
٩ـ العلامات :
حاضر	نعم		
	لا		
غائب 	نعم	بإذن ـ بدون إذن	خيارات بدون إذن (بعذر ـ بدون عذر)
	لا		

👈 بيانات النشاط
١ـ الفصل الدراسي : قائمة منسدلة  ( الأول الثاني)
٢ـ الشهر : قائمة منسدلة  ( محرم صفر ربيع أول ربيع ثاني جماد أول جماد ثاني رجب شعبان)
٣ـ الأسبوع : ( الأول  الثاني الثالث الرابع)  
٤ـ اليوم : قائمة منسدلة ( السبت الأحد الإثنين الثلاثاء الأربعاء الخميس )
٥ـ التاريخ : يكتب تلقائي  ( هجري ميلادي)
٦ـ موعد النشاط قائمة منسدلة  ( أسبوعي ـ دوري ـ أخرى )
٧ـ الوقت : 16:5 صباحا مساء
٨ـ مدة النشاط : ادخال يدوي نص
٩ـ مقر النشاط : ادخال يدوي نص 
١٠ـ برنامج النشاط : ادخال يدوي قائمة منسدلة نص قائمة
١١ـ نفذ : قائمة منسدلة ( نعم لا)
١٢ـ توثيق النشاط : ادخال (ملف فيديوا صور ملف صوتي أخرى)
١٣ ـ ملاحظات : ادخال يدوي قائمة منسدلة نص قائمة

👈عمل جنب هذه الفقرة مايلي:-
تصدير لعام ـ تصدير لفصل ـ استرداد من عام استرداد من فصل ـ استرداد الفقرة كاملة ـ استرداد فقرة مختارة ـ استرداد جميع الفقرات

عمل جنب هذه الفقرة بيانات الاضافة:-
ـ تاريخ الإضافة قائمة منسدلة  ( هجري ميلادي) الساعة 
ـ اليوم قائمة منسدلة ( السبت الأحد الإثنين الثلاثاء الأربعاء الخميس الجمعة )
ـ اسم الموظف الذي اضاف : 

عمل جنب هذه الفقرة هذه الخيارات:-
ـ زر ( معاينة ) لعرض العمل والنظر فيه قبل تصديره الخيارات التي يتم تثبيتها ورقة × ورقة ـ أو ورقتين  × ورقة ـ أو ثلاث ورق  × ورقة ـ أو أربع ورق  × ورقة
ـ تصدير التقرير يكون ورقة × ورقة. ـ أو ورقتين  × ورقة ـ  أو ثلاث ورق  × ورقة ـ أو أربع ورق  × ورقة
ـ تصدير عمود أعمدة مختارة جميع الأعمدة قائمة منسدلة  (  Excel PDF )
ـ تصديرها لملف قائمة منسدلة  (  Excel PDF )
ـ تصدير للفصل الثاني ـ تصدير للعام القادم 
ـ استيراد من الفصل الأول استيراد من العام الماضي
ـ مشاركة قائمة منسدلة  (  Excel PDF صورة )
ـ الاحتفاظ بصور الاشعارات للرجوع لها إذا لزم

عمل جنب هذه الفقرة هذه الازرار
🔲 معاينة 🔲 تحرير 🔲 تعديل 🔲 إضافة 🔲 تصدير 🔲 حذف 🔲 فتحPDF 🔲 فتحExcel 🔲 طباعة 🔲 مشاركة
🎄🎄🎄🎄🎄🎄🎄🎄🎄🎄🎄🎄🎄🎄 فاصل بين الفقرات الرئيسية 🎄🎄🎄🎄🎄🎄🎄🎄🎄🎄🎄🎄🎄
🔴 ١٦  ـ ١ ـ المساهمة المجتمعية
من يكتب هذه الفقرة	المدير أو من يكلفه المدير في حينه موافقة المدير على الادخال أو التعديل
من يشاهد هذه الفقرة	المدير ومن يسمح له المدير في حينه
من يشاهد تقارير هذه الفقرة	المدير
من يصدر تقارير هذه الفقرة	المدير
خيارات البحث بهذه الفقرة يكون بواسطة 	اسم فقرة كلمة جملة
التقارير المطلوبة من هذه الفقرة	تصدير تقرير شهري بالمسلمين لصف لصفوف مختارة ـ لجميع الصفوف لفصل لعام 
تصدير تقرير شهري بالمبلغ المتبقي لصف لصفوف مختارة ـ لجميع الصفوف لفصل لعام
تصدير تقرير شهري بالذين لم يسلموا لصف لصفوف مختارة ـ لجميع الصفوف لفصل لعام

تصدير تقرير بجملة المبلغ الوارد يومي لصف لصفوف مختارة ـ لجميع الصفوف لفصل لعام
تصدير تقرير بجملة المبلغ الوارد أسبوعي لصف لصفوف مختارة ـ لجميع الصفوف لفصل لعام
تصدير تقرير بجملة المبلغ الوارد شهري لصف لصفوف مختارة ـ لجميع الصفوف لفصل لعام

تصدير تقرير بجملة المبلغ المتبقي يومي لصف لصفوف مختارة ـ لجميع الصفوف لفصل لعام
تصدير تقرير بجملة المبلغ المتبقي أسبوعي لصف لصفوف مختارة ـ لجميع الصفوف لفصل لعام
تصدير تقرير بجملة المبلغ المتبقي شهري لصف لصفوف مختارة ـ لجميع الصفوف لفصل لعام

تصدير تقرير بجملة المبلغ الغير مسلم يومي لصف لصفوف مختارة ـ لجميع الصفوف لفصل لعام
تصدير تقرير بجملة المبلغ الغير مسلم أسبوعي لصف لصفوف مختارة ـ لجميع الصفوف لفصل لعام
تصدير تقرير بجملة المبلغ الغير مسلم شهري لصف لصفوف مختارة ـ لجميع الصفوف لفصل لعام

تصدير تقرير بأسماء وبيانات المعفين لصف لصفوف مختارة ـ لجميع الصفوف لفصل لعام
تصدير تقرير بجملة مبلغ المعفين لصف لصفوف مختارة ـ لجميع الصفوف لفصل لعام
تصدير تقرير لطالب لطلاب لشهر لشهور مختارة لفصل لعام

١ـ مسلسل رقم متسلسل تلقائي
٢ـ الاسم 
٣ـ الجنس قائمة منسدلة  ( ذكر أنثى)
٤ـ الفصل الدراسي قائمة منسدلة  ( الأول الثاني)
٥ـ الشهر قائمة منسدلة  ( محرم صفر ربيع أول ربيع ثاني جماد أول جماد ثاني رجب شعبان)
٦ـ الأسبوع ( الأول  الثاني الثالث الرابع)  
٧ـ اليوم قائمة منسدلة ( السبت الأحد الإثنين الثلاثاء الأربعاء الخميس )
٨ـ التاريخ يكتب تلقائي  ( هجري ميلادي)
٩ـ الصف ( أول أساسي ثاني أساسي ثالث أساسي رابع أساسي خامس أساسي سادس أساسي سابع أساسي ثامن أساسي تاسع أساسي ـ أول ثانوي ـ ثاني ثانوي ـ ثالث ثانوي)
١٠ـ الشعبة قائمة منسدلة ( أ ب ج بدون)
١١ـ مبلغ الشهر المقرر قائمة منسدلة ( ١٥٠٠ ـ ٢٠٠٠ ـ اخرى )
١٢ـ المبلغ المستلم ادخال
١٣ـ المبلغ الباقي هذا يكون يكتب تلقائي وفقا للمبلغ المقرر والمبلغ المستلم
١٤ـ اسم المستلم
١٥ـ اسم موصل المبلغ
١٦ـ الحالة قائمة منسدلة ( معفي غير معفي )
١٧ـ سبب الاعفاء قائمة منسدلة (  يتيم ـ ابن تربوي ـ ابن موظف ـ  أحفاد بلال ـ له أكثر من أخ ـ حالة متعسرة ـ  اخرى)
١٨ـ مبلغ الاعفاء قائمة منسدلة ( كامل ـ ٥٠٠ ـ ١٠٠٠ ـ ١٥٠٠ ـ ٢٠٠٠ ـ اخرى)
١٩ـ جهة الاعفاء قائمة منسدلة ( تعميم ـ مدير ـ أخرى)
٢٠ـ ملاحظات ادخال يدوي قائمة منسدلة نص قائمة

عمل جنب هذه الفقرة بيانات الاضافة:-
ـ تاريخ الإضافة قائمة منسدلة  ( هجري ميلادي) الساعة 
ـ اليوم قائمة منسدلة ( السبت الأحد الإثنين الثلاثاء الأربعاء الخميس الجمعة )
ـ اسم الموظف الذي اضاف : 

عمل جنب هذه الفقرة هذه الخيارات:-
ـ زر ( معاينة ) لعرض العمل والنظر فيه قبل تصديره الخيارات التي يتم تثبيتها ورقة × ورقة ـ أو ورقتين  × ورقة ـ أو ثلاث ورق  × ورقة ـ أو أربع ورق  × ورقة
ـ تصدير التقرير يكون ورقة × ورقة. ـ أو ورقتين  × ورقة ـ  أو ثلاث ورق  × ورقة ـ أو أربع ورق  × ورقة
ـ تصدير عمود أعمدة مختارة جميع الأعمدة قائمة منسدلة  (  Excel PDF )
ـ تصديرها لملف قائمة منسدلة  (  Excel PDF )
ـ تصدير للفصل الثاني ـ تصدير للعام القادم 
ـ استيراد من الفصل الأول استيراد من العام الماضي
ـ مشاركة قائمة منسدلة  (  Excel PDF صورة )
ـ الاحتفاظ بصور الاشعارات للرجوع لها إذا لزم

عمل جنب هذه الفقرة هذه الازرار
🔲 معاينة 🔲 تحرير 🔲 تعديل 🔲 إضافة 🔲 تصدير 🔲 حذف 🔲 فتحPDF 🔲 فتحExcel 🔲 طباعة 🔲 مشاركة
🎄🎄🎄🎄🎄🎄🎄🎄🎄🎄🎄🎄🎄🎄 فاصل بين الفقرات الرئيسية 🎄🎄🎄🎄🎄🎄🎄🎄🎄🎄🎄🎄🎄

🔴  ١٧  ـ ١ ـ النشاط المرافق لكل صف لإستغلال عطلة نهاية الفصل الدراسي الأول
من يكتب هذه الفقرة	المدير ومن يحدده المدير في حينه 
من يشاهد هذه الفقرة	المدير ومسؤال الصف فقط ومن يسمح له المدير في حينه
من يشاهد تقارير هذه الفقرة	المدير
من يصدر تقارير هذه الفقرة	المدير
خيارات البحث بهذه الفقرة يكون بواسطة 	اسم نشاط فقرة كلمة جملة
التقارير المطلوبة من هذه الفقرة	تصدير تقرير للنشاط المرافق لصف لصفوف مختارة لجميع الصفوف
تصدير تقرير درجة النشاط لصف لصفوف مختارة لجميع الصفوف
تصدير تقرير درجة النشاط لطالب لطلاب مختارين

١ـ مسلسل رقم متسلسل تلقائي
٢ـ الصف : ( أول أساسي ثاني أساسي ثالث أساسي رابع أساسي خامس أساسي سادس أساسي سابع أساسي ثامن أساسي تاسع أساسي ـ أول ثانوي ـ ثاني ثانوي ـ ثالث ثانوي)
٣ـ الشعبة : قائمة منسدلة ( أ ب ج بدون)
٤ـ برنامج النشاط : ادخال يدوي قائمة منسدلة نص قائمة
٥ـ نفذ : قائمة منسدلة ( نعم لا)
٦ـ توبع : قائمة منسدلة ( نعم لا)
٧ـ صحح : قائمة منسدلة ( نعم لا)
٨ـ  توثيق لنماذخ مختارة : ادخال (ملف فيديوا صور ملف صوتي أخرى)
٩ ـ ملاحظات : ادخال يدوي قائمة منسدلة نص قائمة

👈 بيانات المنفذين للنشاط
١ـ مسلسل رقم متسلسل تلقائي
٢ـ الاسم يكتب تلقائي
٣ـ الجنس : قائمة منسدلة  ( ذكر أنثى)
٤ـ الصف : ( أول أساسي ثاني أساسي ثالث أساسي رابع أساسي خامس أساسي سادس أساسي سابع أساسي ثامن أساسي تاسع أساسي ـ أول ثانوي ـ ثاني ثانوي ـ ثالث ثانوي)
٥ـ الشعبة : قائمة منسدلة ( أ ب ج بدون)
٦ـ سلم : قائمة منسدلة ( نعم لا)
٧ــ اليوم : قائمة منسدلة ( السبت الأحد الإثنين الثلاثاء الأربعاء الخميس )
٨ـ التاريخ : يكتب تلقائي  ( هجري ميلادي)
٩ـ الدرجة المقررة : (  رقم متسلسل ) 
١٠ـ الدرجة المستحقة : ( رقم متسلسل  )
١١ـ الترتيب : يكتب تلقائي وفقا لدرجته المستحقة
١٢ـ ملاحظات : ادخال يدوي قائمة منسدلة نص قائمة

👈عمل جنب هذه الفقرة مايلي:-
تصدير لعام ـ تصدير لفصل ـ استرداد من عام استرداد من فصل ـ استرداد الفقرة كاملة ـ استرداد فقرة مختارة ـ استرداد جميع الفقرات

عمل جنب هذه الفقرة بيانات الاضافة:-
ـ تاريخ الإضافة قائمة منسدلة  ( هجري ميلادي) الساعة 
ـ اليوم قائمة منسدلة ( السبت الأحد الإثنين الثلاثاء الأربعاء الخميس الجمعة )
ـ اسم الموظف الذي اضاف : 

عمل جنب هذه الفقرة هذه الخيارات:-
ـ زر ( معاينة ) لعرض العمل والنظر فيه قبل تصديره الخيارات التي يتم تثبيتها ورقة × ورقة ـ أو ورقتين  × ورقة ـ أو ثلاث ورق  × ورقة ـ أو أربع ورق  × ورقة
ـ تصدير التقرير يكون ورقة × ورقة. ـ أو ورقتين  × ورقة ـ  أو ثلاث ورق  × ورقة ـ أو أربع ورق  × ورقة
ـ تصدير عمود أعمدة مختارة جميع الأعمدة قائمة منسدلة  (  Excel PDF )
ـ تصديرها لملف قائمة منسدلة  (  Excel PDF )
ـ تصدير للفصل الثاني ـ تصدير للعام القادم 
ـ استيراد من الفصل الأول استيراد من العام الماضي
ـ مشاركة قائمة منسدلة  (  Excel PDF صورة )
ـ الاحتفاظ بصور الاشعارات للرجوع لها إذا لزم

عمل جنب هذه الفقرة هذه الازرار
🔲 معاينة 🔲 تحرير 🔲 تعديل 🔲 إضافة 🔲 تصدير 🔲 حذف 🔲 فتحPDF 🔲 فتحExcel 🔲 طباعة 🔲 مشاركة
🎄🎄🎄🎄🎄🎄🎄🎄🎄🎄🎄🎄🎄🎄 فاصل بين الفقرات الرئيسية 🎄🎄🎄🎄🎄🎄🎄🎄🎄🎄🎄🎄🎄
🔴 ١٨ ـ صندوق
١٨  ـ ١ ـ الصادر للموظفين
من يكتب هذه الفقرة	المدير أو من يكلفه المدير من الموظفين في حينه وموافقة المدير على الادخال أو التعديل
من يشاهد هذه الفقرة	الموظف
من يشاهد تقارير هذه الفقرة	المدير
من يصدر تقارير هذه الفقرة	المدير
اضافة لهذه الفقرة	استيراد من الفصل الأول ـ من العام الماضي ـ استرداد الفقرة كاملة ـ استرداد فقرة مختارة ـ استرداد فقرات مختارة ـ 
خيارات البحث بهذه الفقرة يكون بواسطة 	اسم فقرة كلمة جملة
التقارير المطلوبة من هذه الفقرة	ـ تقرير ليوم أسبوع شهر ـ لفصل ـ لعام

١ـ الفصل الدراسي قائمة منسدلة  ( الأول الثاني)
٢ـ الشهر قائمة منسدلة  ( محرم صفر ربيع أول ربيع ثاني جماد أول جماد ثاني رجب شعبان)
٣ـ الأسبوع ( الأول  الثاني الثالث الرابع)  
٤ـ اليوم قائمة منسدلة ( السبت الأحد الإثنين الثلاثاء الأربعاء الخميس )
٥ـ التاريخ قائمة منسدلة  ( هجري ميلادي)
٦ـ الساعة 
٧ـ العنوان
٨ـ مرفقات التوثيق فيديو صور ملفات نصية ملف صوتي نص اخرى
ملاحظة : اضافة جوار صندوق الصادر خيارات ( تعديل ـ حذف )

👈عمل جنب هذه الفقرة مايلي:-
تصدير لعام ـ تصدير لفصل ـ استرداد من عام استرداد من فصل ـ استرداد الفقرة كاملة ـ استرداد فقرة مختارة ـ استرداد جميع الفقرات

عمل جنب هذه الفقرة بيانات الاضافة:-
ـ تاريخ الإضافة قائمة منسدلة  ( هجري ميلادي) الساعة 
ـ اليوم قائمة منسدلة ( السبت الأحد الإثنين الثلاثاء الأربعاء الخميس الجمعة )
ـ اسم الموظف الذي اضاف : 

عمل جنب هذه الفقرة هذه الخيارات:-
ـ زر ( معاينة ) لعرض العمل والنظر فيه قبل تصديره الخيارات التي يتم تثبيتها ورقة × ورقة ـ أو ورقتين  × ورقة ـ أو ثلاث ورق  × ورقة ـ أو أربع ورق  × ورقة
ـ تصدير التقرير يكون ورقة × ورقة. ـ أو ورقتين  × ورقة ـ  أو ثلاث ورق  × ورقة ـ أو أربع ورق  × ورقة
ـ تصدير عمود أعمدة مختارة جميع الأعمدة قائمة منسدلة  (  Excel PDF )
ـ تصديرها لملف قائمة منسدلة  (  Excel PDF )
ـ تصدير للفصل الثاني ـ تصدير للعام القادم 
ـ استيراد من الفصل الأول استيراد من العام الماضي
ـ مشاركة قائمة منسدلة  (  Excel PDF صورة )
ـ الاحتفاظ بصور الاشعارات للرجوع لها إذا لزم

عمل جنب هذه الفقرة هذه الازرار
🔲 معاينة 🔲 تحرير 🔲 تعديل 🔲 إضافة 🔲 تصدير 🔲 حذف 🔲 فتحPDF 🔲 فتحExcel 🔲 طباعة 🔲 مشاركة
➖➖➖➖➖➖➖➖➖➖➖ فاصل بين الفقرات الفرعية ➖➖➖➖➖➖➖➖➖➖➖➖➖
١٨  ـ ٢ ـ الوارد من الموظفين
من يكتب هذه الفقرة	الموظف وموافقة المدير على الادخال أو التعديل
من يشاهد هذه الفقرة	الموظف يشاهد ما ارسله
من يشاهد تقارير هذه الفقرة	المدير
من يصدر تقارير هذه الفقرة	المدير
اضافة لهذه الفقرة	استيراد من الفصل الأول ـ من العام الماضي استرداد الفقرة كاملة ـ استرداد فقرة مختارة ـ استرداد فقرات مختارة
خيارات البحث بهذه الفقرة يكون بواسطة 	اسم فقرة كلمة جملة
التقارير المطلوبة من هذه الفقرة	ـ تقرير موظف ـ موظفين مختارين ـ جميع الموظفين ليوم أسبوع شهر ـ لفصل ـ لعام

١ـ الفصل الدراسي قائمة منسدلة  ( الأول الثاني)
٢ـ الشهر قائمة منسدلة  ( محرم صفر ربيع أول ربيع ثاني جماد أول جماد ثاني رجب شعبان)
٣ـ الأسبوع ( الأول  الثاني الثالث الرابع)  
٤ـ اليوم قائمة منسدلة ( السبت الأحد الإثنين الثلاثاء الأربعاء الخميس )
٥ـ التاريخ قائمة منسدلة  ( هجري ميلادي)
٦ـ الساعة
٧ـ العنوان
٨ـ مرفقات التوثيق فيديو صور ملفات نصية ملف صوتي نص اخرى
ملاحظة : اضافة جوار صندوق الوارد خيارات ( تعديل ـ حذف )

👈عمل جنب هذه الفقرة مايلي:-
تصدير لعام ـ تصدير لفصل ـ استرداد من عام استرداد من فصل ـ استرداد الفقرة كاملة ـ استرداد فقرة مختارة ـ استرداد جميع الفقرات

عمل جنب هذه الفقرة بيانات الاضافة:-
ـ تاريخ الإضافة قائمة منسدلة  ( هجري ميلادي) الساعة 
ـ اليوم قائمة منسدلة ( السبت الأحد الإثنين الثلاثاء الأربعاء الخميس الجمعة )
ـ اسم الموظف الذي اضاف : 

عمل جنب هذه الفقرة هذه الخيارات:-
ـ زر ( معاينة ) لعرض العمل والنظر فيه قبل تصديره الخيارات التي يتم تثبيتها ورقة × ورقة ـ أو ورقتين  × ورقة ـ أو ثلاث ورق  × ورقة ـ أو أربع ورق  × ورقة
ـ تصدير التقرير يكون ورقة × ورقة. ـ أو ورقتين  × ورقة ـ  أو ثلاث ورق  × ورقة ـ أو أربع ورق  × ورقة
ـ تصدير عمود أعمدة مختارة جميع الأعمدة قائمة منسدلة  (  Excel PDF )
ـ تصديرها لملف قائمة منسدلة  (  Excel PDF )
ـ تصدير للفصل الثاني ـ تصدير للعام القادم 
ـ استيراد من الفصل الأول استيراد من العام الماضي
ـ مشاركة قائمة منسدلة  (  Excel PDF صورة )
ـ الاحتفاظ بصور الاشعارات للرجوع لها إذا لزم

عمل جنب هذه الفقرة هذه الازرار
🔲 معاينة 🔲 تحرير 🔲 تعديل 🔲 إضافة 🔲 تصدير 🔲 حذف 🔲 فتحPDF 🔲 فتحExcel 🔲 طباعة 🔲 مشاركة
🎄🎄🎄🎄🎄🎄🎄🎄🎄🎄🎄🎄🎄🎄 فاصل بين الفقرات الرئيسية 🎄🎄🎄🎄🎄🎄🎄🎄🎄🎄🎄🎄🎄
🔴 ١٩ ـ توثيق أعمال العام
من يكتب هذه الفقرة	المدير أو من يكلفه المدير من الموظفين في حينه وموافقة المدير على الادخال أو التعديل
من يشاهد هذه الفقرة	المدير والموظف بعد طلب الاذن من المدير
من يشاهد تقارير هذه الفقرة	المدير
من يصدر تقارير هذه الفقرة	المدير
اضافة لهذه الفقرة	استيراد من الفصل الأول ـ من العام الماضي
خيارات البحث بهذه الفقرة يكون بواسطة 	اسم فقرة كلمة جملة
التقارير المطلوبة من هذه الفقرة	ـ تقرير ليوم أسبوع شهر ـ لفصل ـ لعام

١ـ الفصل الدراسي قائمة منسدلة  ( الأول الثاني)
٢ـ الشهر قائمة منسدلة  ( محرم صفر ربيع أول ربيع ثاني جماد أول جماد ثاني رجب شعبان)
٣ـ الأسبوع ( الأول  الثاني الثالث الرابع)  
٤ـ اليوم قائمة منسدلة ( السبت الأحد الإثنين الثلاثاء الأربعاء الخميس )
٥ـ التاريخ قائمة منسدلة  ( هجري ميلادي)
٦ـ الساعة
٧ـ العنوان
٨ـ مرفقات التوثيق فيديو صور ملفات نصية ملف صوتي نص اخرى


عمل جنب هذه الفقرة بيانات الاضافة:-
ـ تاريخ الإضافة قائمة منسدلة  ( هجري ميلادي) الساعة 
ـ اليوم قائمة منسدلة ( السبت الأحد الإثنين الثلاثاء الأربعاء الخميس الجمعة )
ـ اسم الموظف الذي اضاف : 

عمل جنب هذه الفقرة هذه الخيارات:-
ـ زر ( معاينة ) لعرض العمل والنظر فيه قبل تصديره الخيارات التي يتم تثبيتها ورقة × ورقة ـ أو ورقتين  × ورقة ـ أو ثلاث ورق  × ورقة ـ أو أربع ورق  × ورقة
ـ تصدير التقرير يكون ورقة × ورقة. ـ أو ورقتين  × ورقة ـ  أو ثلاث ورق  × ورقة ـ أو أربع ورق  × ورقة
ـ تصدير عمود أعمدة مختارة جميع الأعمدة قائمة منسدلة  (  Excel PDF )
ـ تصديرها لملف قائمة منسدلة  (  Excel PDF )
ـ تصدير للفصل الثاني ـ تصدير للعام القادم 
ـ استيراد من الفصل الأول استيراد من العام الماضي
ـ مشاركة قائمة منسدلة  (  Excel PDF صورة )
ـ الاحتفاظ بصور الاشعارات للرجوع لها إذا لزم

عمل جنب هذه الفقرة هذه الازرار
🔲 معاينة 🔲 تحرير 🔲 تعديل 🔲 إضافة 🔲 تصدير 🔲 حذف 🔲 فتحPDF 🔲 فتحExcel 🔲 طباعة 🔲 مشاركة
🎄🎄🎄🎄🎄🎄🎄🎄🎄🎄🎄🎄🎄🎄 فاصل بين الفقرات الرئيسية 🎄🎄🎄🎄🎄🎄🎄🎄🎄🎄🎄🎄🎄
🔴 ٢٠ ـ الملخصات الدراسية المتوفرة
من يكتب هذه الفقرة	المدير ومن يحدده المدير في حينه 
من يشاهد هذه الفقرة	المدير والموظف 
من يشاهد تقارير هذه الفقرة	المدير
من يصدر تقارير هذه الفقرة	المدير
خيارات البحث بهذه الفقرة يكون بواسطة 	اسم نشاط فقرة كلمة جملة
التقارير المطلوبة من هذه الفقرة	تصدير تقرير للملخصات الموجودة لصف لصفوف مختارة لجميع الصفوف
تصدير تقرير الملخصات الموجودة لمادة لمواد مختارة لجميع المواد
تصدير تقرير بمن حمل الملخصات الاسم الملخص اليوم التاريخ

١ـ مسلسل رقم متسلسل تلقائي
٢ـ الصف : ( أول أساسي ثاني أساسي ثالث أساسي رابع أساسي خامس أساسي سادس أساسي سابع أساسي ثامن أساسي تاسع أساسي ـ أول ثانوي ـ ثاني ثانوي ـ ثالث ثانوي)
٣ـ الفصل الدراسي : قائمة منسدلة  ( الأول الثاني مشترك)
٤ـ اسم مادة الملخص قائمة منسدلة  ( القرآن الإسلامية  اللغة العربية E حصة E واجب الرياضيات العلوم تاريخ جغرافيا وطنية أخرى )
٥ـ اليوم : قائمة منسدلة ( السبت الأحد الإثنين الثلاثاء الأربعاء الخميس )
٦ـ التاريخ : يكتب تلقائي  ( هجري ميلادي)
٧ـ ملاحظات : ادخال يدوي قائمة منسدلة نص قائمة

👈عمل جنب هذه الفقرة مايلي:-
تصدير لعام ـ تصدير لفصل ـ استرداد من عام استرداد من فصل ـ استرداد الفقرة كاملة ـ استرداد فقرة مختارة ـ استرداد جميع الفقرات

عمل جنب هذه الفقرة بيانات الاضافة:-
ـ تاريخ الإضافة قائمة منسدلة  ( هجري ميلادي) الساعة 
ـ اليوم قائمة منسدلة ( السبت الأحد الإثنين الثلاثاء الأربعاء الخميس الجمعة )
ـ اسم الموظف الذي اضاف : 

عمل جنب هذه الفقرة هذه الخيارات:-
ـ زر ( معاينة ) لعرض العمل والنظر فيه قبل تصديره الخيارات التي يتم تثبيتها ورقة × ورقة ـ أو ورقتين  × ورقة ـ أو ثلاث ورق  × ورقة ـ أو أربع ورق  × ورقة
ـ تصدير التقرير يكون ورقة × ورقة. ـ أو ورقتين  × ورقة ـ  أو ثلاث ورق  × ورقة ـ أو أربع ورق  × ورقة
ـ تصدير عمود أعمدة مختارة جميع الأعمدة قائمة منسدلة  (  Excel PDF )
ـ تصديرها لملف قائمة منسدلة  (  Excel PDF )
ـ تصدير للفصل الثاني ـ تصدير للعام القادم 
ـ استيراد من الفصل الأول استيراد من العام الماضي
ـ مشاركة قائمة منسدلة  (  Excel PDF صورة )
ـ الاحتفاظ بصور الاشعارات للرجوع لها إذا لزم

عمل جنب هذه الفقرة هذه الازرار
🔲 معاينة 🔲 تحرير 🔲 تعديل 🔲 إضافة 🔲 تصدير 🔲 حذف 🔲 فتحPDF 🔲 فتحExcel 🔲 طباعة 🔲 مشاركة
🎄🎄🎄🎄🎄🎄🎄🎄🎄🎄🎄🎄🎄🎄 فاصل بين الفقرات الرئيسية 🎄🎄🎄🎄🎄🎄🎄🎄🎄🎄🎄🎄🎄

المشاهد
ملاحظة يكون المشاهد يشاهد نظام الموقع فقط دون الاطلاع على أي بيانات 
ولا يسمح له باضافة أي شي
سوى يفتح له مربع نص تلقائي ويوجه اليه
فضلا : أكتب انطباعك هنا
ليكتب فيه ما يريد
وما كتبه لا يطلع عليه أحد سوى المدير فقط

من يكتب هذه الفقرة	تلقائي
من يشاهد هذه الفقرة	المدير ومن يسمح له المدير في حينه
من يشاهد تقارير هذه الفقرة	المدير
من يصدر تقارير هذه الفقرة	المدير
خيارات البحث بهذه الفقرة يكون بواسطة 	اسم فقرة كلمة جملة
التقارير المطلوبة من هذه الفقرة	تصدير تقرير يومي بالمشاهدين 
تصدير تقرير أسبوعي بالمشاهدين
تصدير تقرير شهري بالمشاهدين
تصدير تقرير فصلي بالمشاهدين
تصدير تقرير للعام بالمشاهدين



١ـ مسلسل رقم متسلسل تلقائي
٢ـ الاسم 
٣ـ رقم جواله
٤ـ الجنس قائمة منسدلة  ( ذكر أنثى)
٥ـ الفصل الدراسي قائمة منسدلة  ( الأول الثاني)
٦ـ الشهر قائمة منسدلة  ( محرم صفر ربيع أول ربيع ثاني جماد أول جماد ثاني رجب شعبان)
٧ـ الأسبوع ( الأول  الثاني الثالث الرابع)  
٨ـ اليوم قائمة منسدلة ( السبت الأحد الإثنين الثلاثاء الأربعاء الخميس الجمعة )
٩ـ التاريخ يكتب تلقائي  ( هجري ميلادي)
١٠ـ ماذا شاهد
١١ـ مدة الجلسة 
١٢ ـ ملاحظات ادخال يدوي قائمة منسدلة نص قائمة
🎄🎄🎄🎄🎄🎄🎄🎄🎄🎄🎄🎄🎄🎄 فاصل بين الفقرات الرئيسية 🎄🎄🎄🎄🎄🎄🎄🎄🎄🎄🎄🎄🎄
أولياء الأمور

من يكتب هذه الفقرة	ولي الأمر
من يشاهد هذه الفقرة	المدير  
من يشاهد تقارير هذه الفقرة	المدير
من يصدر تقارير هذه الفقرة	المدير
خيارات البحث بهذه الفقرة يكون بواسطة 	اسم فقرة كلمة جملة
التقارير المطلوبة من هذه الفقرة	تصدير تقرير يومي بالوارد من أولياء الأمور 
تصدير تقرير أسبوعي بالوارد من أولياء الأمور
تصدير تقرير شهري بالوارد من أولياء الأمور
تصدير تقرير فصلي بالوارد من أولياء الأمور
تصدير تقرير للعام بالوارد من أولياء الأمور


١ـ مسلسل يكتب تلقائي وفقا لعدد ما ارسله إذا كان أول مرة يكتب تلقائي ١ وإذا كان قد ارسل مرة سابقة يكتب تلقائي ٢ وهكذا
٢ـ فضلا سجل اسمك : ادخال نص
٣ـ رقم جوالك : ادخال رقم
٤ـ عنوانك : ادخال نص
٥ـ صفة القرابة : قائمة منسدلة( أب أم أخ أخت عم عمة خال خالة جد جدة أخرى)
٦ ـ موضوع طلبك قائمة منسدلة ( طلب إذن لطالب / طالبة ـ رد على إشعار ارسل لكم ـ طلب حل مشكلة طالب / طالبة ـ طلب درجات طالب / طالبة ـ طلب نتيجة طالب / طالبة ـ طلب معرفة المساهمة المجتمعية لطالب / طالبة ـ أخرى )
٧ـ اسم الطالب / الطالبة
٨ـ الجنس: قائمة منسدلة  ( ذكر أنثى)
٩ـ الصف : قائمة منسدلة ( أول أساسي ثاني أساسي ثالث أساسي رابع أساسي خامس أساسي سادس أساسي سابع أساسي ثامن أساسي تاسع أساسي ـ أول ثانوي ـ ثاني ثانوي ـ ثالث ثانوي)
١٠ـ اليوم : يكتب تلقائي
١١ـ التأريخ : يكتب تلقائي
🎄🎄🎄🎄🎄🎄🎄🎄🎄🎄🎄🎄🎄🎄 فاصل بين الفقرات الرئيسية 🎄🎄🎄🎄🎄🎄🎄🎄🎄🎄🎄🎄🎄


التقارير
🎄🎄🎄🎄🎄🎄🎄🎄🎄🎄🎄🎄🎄🎄 فاصل بين الفقرات الرئيسية 🎄🎄🎄🎄🎄🎄🎄🎄🎄🎄🎄🎄🎄

الاشعارات

🎄🎄🎄🎄🎄🎄🎄🎄🎄🎄🎄🎄🎄🎄 فاصل بين الفقرات الرئيسية 🎄🎄🎄🎄🎄🎄🎄🎄🎄🎄🎄🎄🎄






